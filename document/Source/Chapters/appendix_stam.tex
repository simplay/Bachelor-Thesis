% how about removing this
\chapter{Summary of Stam's Derivations}
\label{chap:stamsderivations}

For his derivations Stam uses the Kirchhoff integral$\footnote{See \texttt{http://en.wikipedia.org/wiki/Kirchhoff\textunderscore integral\textunderscore theorem} for further information.}$, which is relating the reflected field to the incoming field. This equation is a formalization of Huygen’s well-known principle that states that if one knows the wavefront at a given moment, the wave at a later time can be deduced by considering each point on the first wave as the source of a new disturbance. Once the wave $\psi_1 =  e^{ik\mathbf{x} \cdot \mathbf{s}\mathbf{s}}$ on the surface is known, the emitted wave $\psi_2$ can be computed.

% instead want to compute use idea behind kirchhof or something sim
We want to compute an emitted wave $\psi_2$ which is the reflected wave of an incoming planar monochromatic wave 

\begin{equation}
\psi_1 = e^{ik \omega_i * x}
\end{equation}


where $\omega_i$ denotes its propagation direction. Using the Kirchhoff integral we get:

\begin{equation}
\psi_{2}(\omega_i, \omega_r) = \frac{i k e^{i K R}}{4 \pi R} (F(-\omega_i-\omega_r)-(-\omega_i+\omega_r)) \cdot I_{1}(\omega_i, \omega_r) 
\label{eq:kirchhoff}
\end{equation}

with

\begin{equation}
I_{1}(\omega_i, \omega_r) = \int_{S} \hat{\mathbf{n}} e^{ik(-\omega_i-\omega_{r}) \cdot \mathbf{s} d\mathbf{s}}
\label{eq:IBase}
\end{equation}

In applied optics, when dealing with scattered waves, one does use differential scattering cross-section rather than defining a BRDF which has the following identity: 

\begin{equation}
    \sigma^0 = 4 \pi \lim_{R \to \infty} R^2 \frac{\langle \left|\psi_2\right|^2\rangle}{\langle \left|\psi_1\right|^2\rangle}
\end{equation}

where R is the distance from the center of the patch to the receiving point $x_p$, $\hat{\mathbf{n}}$ is the normal of the surface at s and the vectors:

The relationship between the BRDF and the scattering cross section can be shown to be equal to 

\begin{equation}
 BRDF = \frac{1}{4\pi}\frac{1}{A}\frac{\sigma^0}{cos(\theta_i)cos(\theta_r)}
 \label{fig:crossscateringbrdfrelationship} 
\end{equation}

where $\theta_i$ and $\theta_r$ are the angles of incident and reflected directions on the surface with the surface normal $n$ according to figure ~\ref{fig:geometricsetup}.

The components of vector resulting by the difference between these direction vectors:
In order to simplify the calculations involved in his vectorized integral equations, Stam considers the components of vector 
\begin{equation}
  (u,v,w) = -\omega_i - \omega_r 
\label{eq:uvwappendix}
\end{equation}

explicitly and introduces the equation: 
\begin{equation}
  I(ku,kv) = \int_{S} \hat{\mathbf{n}} e^{ik(u,v,w) \cdot \mathbf{s} d\mathbf{s}} 
\label{eq:Istart}
\end{equation}

which is a first simplification of $\ref{eq:IBase}$. Note that the scalar $w$ is the third component of ~\ref{eq:uvw} and can be written as $w = -(cos(\theta_i)+cos(\theta_r))$ using spherical coordinates. The scalar $k=\frac{2\pi}{\lambda}$ represent the wavenumber.


During his derivations, Stam provides a analytical representation for the Kirchhoff integral assuming that each surface point $s(x,y)$ can be parameterized by $(x,y,h(x,y))$ where $h$ is the height at the position $(x,y)$ on the given $(x,y)$ surface plane. Using the tangent plane approximation for the parameterized surface and plugging it into $\ref{eq:Istart}$ he will end up with: 

\begin{equation}
    \mathbf{I}(ku, kv) = \int \int (-h_{x}(x,y), -h_{y}(x,y), 1) e^{ikwh(x,y)} e^{ik(ux + vy)} dx dy
\label{eq:I1}
\end{equation}

For further simplification Stam formulates auxiliary function which depends on the provided height field: 
\begin{equation}
  p(x,y) = e^{iwkh(x,y)} 
\label{eq:pxappendix}
\end{equation}

which will allow him to further simplify his equation $\ref{eq:I1}$ to:

\begin{equation}
    \mathbf{I}(ku, kv) = \int \int \frac{1}{ikw}(-p_x, -p_y, ikwp) dx dy
\label{eq:I2}
\end{equation}

where he used that $(-h_{x}(x,y), -h_{y}(x,y), 1)e^{kwh(x,y)}$ is equal to $\frac{(-p_x, -p_y, ikwp)}{ikw}$ using the definition of the partial derivatives applied to the function $\ref{eq:px}$.

Let $P(x,y)$ denote the Fourier Transform (FT) of $p(x,y)$. Then, the differentiation with respect to x respectively to y in the Fourier domain is equivalent to a multiplication of the Fourier transform by $-iku$ or $-ikv$ respectively. This leads him to the following simplification for $\ref{eq:I1}$:

\begin{equation}
    \mathbf{I}(ku, kv) = \frac{1}{w}P(ku, kv) \cdot (u,v,w)
\label{eq:I3}
\end{equation}

Let us consider the term $g = (F(-\omega_i - \omega_r)-(-\omega_i + \omega_r))$, which is a scalar factor of $\ref{eq:kirchhoff}$. The dot product with $g$ and $(-\omega_i - \omega_r)$ is equal $2F(1 + \omega_i \cdot \omega_r)$. Putting this finding and the identity $\ref{eq:I3}$ into $\ref{eq:kirchhoff}$ he will end up with:

\begin{equation}
\psi_{2}(\omega_i, \omega_r) = \frac{i k e^{i K R}}{4 \pi R} \frac{2F(1 + \omega_i \cdot \omega_r)}{w} P(ku, kv)
\label{eq:kirchhoffFinding}
\end{equation}

By using the identity $\ref{fig:crossscateringbrdfrelationship}$, this will lead us to his main finding:
\begin{equation} 
  BRDF_{\lambda}(\omega_i, \omega_r) = \frac{k^2 F^2 G}{4\pi^2 A w^2} \langle \left|P(ku, kv)\right|^2\rangle
\label{eq:mainstamappendix}
\end{equation}

where $G$ is the so called geometry term which is equal: 

\begin{equation}
  G =\frac{(1 + \omega_i \cdot \omega_r)^2}{cos(\theta_i)cos(\theta_r)}
\label{eq:geometrictermappendix}
\end{equation}