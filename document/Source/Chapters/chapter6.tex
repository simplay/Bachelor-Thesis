\chapter{Conclusion}
\section{Summary}
The aim of this thesis was to simulate structural colors produced by a directional light source upon diffraction by quasyperiodic nanostructures in nature. Our goal also included to use explicitly measured nano-scaled patches to simulate these structural colors at interactive rates. For this purpose we developed a BRDF model based on J. Stam's work described in his paper about diffraction shaders. Furthermore, we implemented our model as a rendering program. 

Rendering the effect of diffraction for a measured nanostructure, described as a discrete height field was previously not done and may therefore be considered as a novelty. \\

Our BRDF model is based on a Taylor series approximation and a Gaussian windowing approach, formulated in a way such that many computationally expensive terms$\footnote{Such as computing different powers of the two dimensional Fourier transformation of the given height field.}$ in our model can be precomputed. These precomputed terms can be accessed by performing a lookup during the shading stages, which reduce the overall runtime complexity of our shader. Nevertheless, our basic implementation (FLSS) is not interactive for most of the present GPU platforms. Therefore, we also devised some optimization into my NMM shading approach, to render scenes interactively. We also formulated a completely different approach, the PQ shading approach, which performs sinc-interpolation, assuming the given height field patch is periodically distributed on a surface. Since the height field of natural diffraction gratings is only a quasiperiodic function, the PQ approach does not appropriately model the reality and thus does not produce reliable results. In addition, we evaluated the quality of all our shading approaches applied on different surface gratings. Finally, we produced some renderings BRDF maps and renderings on a snake mesh using all our shading approaches applied using different surface gratings. \\

For future work we could think of extending our model in a way such that it can handle multilayer diffraction grating. This would allow us to simulate structural colors due to multilayer interferrence wave-effects, which can be observed on wings of certain butterfly species. \\
 
This thesis contributed also to a follow up paper which discusses a fast implementation of the FLSS shading approach. This paper also contains a complete comparison with Stam's Shading approach. For further information, please have a look at the paper $\cite{daljitpaper}$.

\section{Personal Experiences}
I always knew that eventually I will write a thesis in the field of computer graphics. After having attended the $\emph{Computer Graphics}$ class held by Mr. Zwicker, I was certain that I will definitely write my thesis at the Computer Graphics Group at the University of Bern. Since I already acquired a Minor degree in Mathematics I and am very interested in topics like numerics, differential equations and differential geometry, I asked for a thesis subject involving some Mathematics. I have to admit that, after having read J. Stam's paper about diffraction shaders at the very beginning of my thesis, I thought this topic would exceed my knowledge in Mathematics and Physics. And it actually did, but I decided not to give up. \\

During working on my thesis I learnt new concepts in Physics relating to wave-interference, diffraction of waves, diffraction gratings and deepened my knowledge of wave theory for light. Regarding Mathematics I learned quite a lot about the different kind of Fourier transforms, about Windowing techniques for signal processing and about BRDF models. I felt it as a satisfactory experience to use and apply all the knowledge which I have acquired during the time as a bachelor student. Furthermore, this thesis gave me the opportunity to program quite a lot. Hence I could strengthen my programming skills in Java and OpenGL's shading language GLSL. Altogether it was a rewarding experience for me to write a Bachelor thesis at the Computer Graphics Group.

\section{Acknowledgment}
First, I would like to thank Mr. Zwicker for giving me the opportunity to write a Bachelor thesis at the Computer Graphics Group at the University of Bern. \\

Foremost, I would like to express my sincere gratitude to my advisor Mr. Daljit Singh Dhillon for his continuous support of my study, his patience, motivation, enthusiasm, and knowledge. His guidance and active support helped me quite a lot while deriving our BRDF model, developing it and evaluating the shaders and while writing this thesis. \\
  
Last but not least, I would like to thank my mother, Manuela Single and my brother Patrik Single and also to my close friend, Radischa Iyadurai for supporting me morally throughout during this thesis.
