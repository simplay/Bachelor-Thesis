\chapter{Conclusion}
\section{Review}
The goal of this thesis was to simulate structural color produced when a directional light source is diffracted on the nanostructure of a surface, when explicitly being provided by a nano-scaled surface patch. For this purpose we developed BRDF model based on J. Stam's BRDF model described in his paper about diffraction shaders and implemented it in a rendering program. Rendering the effect of diffraction for explicit defined surfaces, described as a discrete height field, was previously not possible and may therefore be considered as a kind of novelty. \\

Our BRDF model is based on a Taylor Series Approximation and a Gaussian Windowing Approach, formulated in a way such that many computationally expensive terms$\footnote{Such as computing different powers of the two dimensional Fourier transformation of the given height field.}$ in our model can be precomputed. These precomputed terms can be accessed by performing a lookup during the shading stages, which enhance the overall runtime complexity of our shading approach. Nevertheless, this basic implementation (FLSS) cannot denoted as being interactive at all. Therefore we also discussed some optimizations for the FLSS approach, like the NMM shading approach, to enhance the run-time complexity of the whole shading process even more. Furthermore, we also formulated a complete different approach, the PQ shading approach, which is performing Sinc-Interpolation, assuming the given height field patch is periodically distributed on a surface. In addition, we evaluated the quality of all our shading approaches applied on different surface gratings. Last, we produced some renderings BRDF maps and renderings on a snake mesh using all our shading approaches applied for different surface gratings. \\
 
This thesis contributed also to a follow up paper which discuesses a fast implementation of the FLSS shading approach. This paper also contains a complete comparison with Stam's Shading approach. For furher information please have a look at the paper $\cite{daljitpaper}$.

\section{Personal Experiences}
I always knew that eventually I will write a thesis in the field of computer graphics. After having attended the $\emph{Computer Graphics}$ class held by Mr. Zwicker, I was certain that I will definitely write my thesis at the Computer Graphics Group at the University of Bern. Since I already acquired a Minor degree in Mathematics and am very interested in this kind of topics, I asked for a thesis subject involving some Mathematics. I have to admit that, after having read the Paper of J. Stam about Diffraction shaders, in the very beginning of my thesis, I thought this topic would exceed my knowledge in Mathematics and Physics. And It actually did, but decided not to give up. \\

Despite of this thesis I could acquire new and even deepen knowledge in various concepts of Physics such as waves theory, wave-interference, diffraction of waves and diffraction gratings. Regarding Mathematics I learned quite a lot about the different kind of Fourier transforms, about Windowing Approaches and about BRDF models. I felt it as a satisfactory experience to use and apply all the knowledge which I have acquired during the time as a bachelor student. Furthermore, this thesis gave me the opportunity to program quite a lot. Hence I could strengthen my programming skills in Java and OpenGL's shading language GLSL. Altogether it was a rewarding experience for me to write a Bachelor Thesis at the Computer Graphics Group.

\section{Acknowledgment}
First, I would like to thank Mr. Zwicker for giving me the opportunity to write a Bachelor Thesis at the Computer Graphics Group at the University of Bern. \\

Foremost, I would like to express my sincere gratitude to my advisor Mr. Daljit Singh for the continuous support of my study, for his patience, motivation, enthusiasm, and knowledge. His guidance and active support helped me quite a lot deriving the BRDF formulation, during developing and evaluating the shaders and writing this thesis. \\
  
Last but not least, I would like to thank my mother and brother, Manuela and Patrik Single and also my close friend, Radischa Iyadurai, for supporting me mentally and spiritually throughout during this thesis.
