\chapter{Conclusion}

can we do better?

brief overwiew pf results achieved, what was the most important in the work, appropriate to provide an introduction to possible future work in this field. reflect the emotions associated with the work, what was especially difficult or particularly interesting, one may elaborate on open questions within subjects related to the thesis without giving any answer. discuss follow-ups.

statment what you've researched and what your original contribution of the fild is
explain why our approach is a good idea
explain how the straight foreward approach would behave compared to our approach, computing the fourier transformations straight away.
explain what we achieved, summary
say something about draw-backs and about limitations of current apporach
say something about the ongoing paper future work
maybe say something about runtime complexity

\section{Further Work}
\subsection{References}
% \begin{itemize}
refs
% chapter1:
% chapter2:
% http://www.cambridgeincolour.com/tutorials/diffraction-photography.htm
% chapter3:
% chapter4:
% chapter5:
% http://h2physics.org/?cat=49 
% www.tau.ac.il/~phchlab/experiments_new/SemB01_Hydrogen/02TheoreticalBackground.html
% www.itp.uni-hannover.de/~zawischa/ITP/multibeam.html 
%http://geneva-physics-of-biology-2013.ch

% TODO: SHOW IMAGE: SOLID angle
% $WIKI: http://en.wikipedia.org/wiki/Radiance$
% $CG Slides 2012 - 6.shading$
% $Book: Fundamentals of computer graphics$


% \item \textbf{{[}1{]}} http://en.wikipedia.org/wiki/Ratio\_test
% \item \textbf{{[}2{]}} http://math.jasonbhill.com/courses/fall-2010-math-2300-005/lectures/taylor-polynomial-error-bounds\end{itemize}
\section{Acknowledgment}
% ty to abcdefg
