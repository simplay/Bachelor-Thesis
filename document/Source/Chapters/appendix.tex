\chapter{Appendix}
\section{Schlick's approximation}
The specular reflection coefficient $R$ can be approximated by:

\begin{equation}
 R(\theta) = R_0 + (1 - R_0)(1 - \cos \theta)^5
\label{eq:schlickapprox}
\end{equation}

and

\begin{equation*}
  R_0 = \left(\frac{n_1-n_2}{n_1+n_2}\right)^2
\end{equation*}

where $\theta$ is the angle between the viewing direction and the half-angle direction, which is halfway between the incident 
light direction and the viewing direction, hence $\cos\theta=(H\cdot V)$. And $n_1,\,n_2$ are the indices of refraction of the two medias at the interface and $R_0$ is the reflection coefficient for light incoming parallel to the normal (i.e., the value of the Fresnel term when $\theta = 0$ or minimal reflection). In computer graphics, one of the interfaces is usually air, meaning that $n_1$ very well can be approximated as 1.

\section{Spherical Coordinates}
$\forall \colvec[x]{y}{z} \in \mathbb{R}^3 : \exists r \in [0,\infty) \exists \phi \in [0,2\pi] \exists \theta \in [0,\pi] $ s.t.
\begin{equation*}
\colvec[x]{y}{z} = \colvec[r sin(\theta)cos(\phi)]{r sin(\theta)sin(\phi)}{r cos(\theta)}
\label{eq:sphericalcoordinates}
\end{equation*}





