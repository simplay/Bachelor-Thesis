\chapter{Results}

show comparision between all shader variants: pq, adaptive, all, nvidiaGem.
mention system specs and how long it took in order to precompute
mention GEM results.
show nminmax renderings

\begin{figure}[H]
  \centering
  \subfigure[A]{
    \includegraphics[scale=0.1118]{ElapheGuttata/1.png}
    \label{fig:realImageElapheGuttata1}
  }
~
  \subfigure[B]{
    \includegraphics[scale=0.189]{ElapheGuttata/4.png}
    \label{fig:realImageElapheGuttata2}
  }
  \label{realImageElapheGuttata}
  \caption{Species Elaphe Guttata}
\end{figure}

\begin{figure}[H]
  \centering
  \subfigure[A]{
    \includegraphics[scale=0.369]{XenopeltisUnicolor/2.png}
    \label{fig:realImageXeno1}
  }
~
  \subfigure[B]{
    \includegraphics[scale=0.40]{XenopeltisUnicolor/3.png}
    \label{fig:realImageXeno2}
  }
  \label{realImageXenopeltis}
  \caption{Species Xenopeltis Unicolor}
\end{figure}

\section{BRDF maps}
Mention brdf map basic idea figure + text


brdf maps for different nanoscale surface gratings
% brdf maps patches
\begin{figure}[H]
  \centering
  \subfigure[Blaze grating]{
    \includegraphics[scale=0.12]{results/diffPatches/fftBlazeHeight_0.25Microns_allL_weak_scale.png}
    \label{fig:brdfmapBlaze}
  }
~
  \subfigure[elaphe grating]{
    \includegraphics[scale=0.12]{results/diffPatches/elaph65.png}
    \label{fig:brdfmapElaphe}
  }
~
  \subfigure[xeno grating]{
    \includegraphics[scale=0.12]{results/diffPatches/xeno65.png}
    \label{fig:brdfmapXeno}
  }
  \label{brdfmapsDiffPatches}
  \caption{BRDF maps for different patches}
\end{figure}


for blaze grating different wavelength step sizes in fragment shader
wider steps fewer iterations, loss of qulatiy. for elaphe step bigger than 10nm there are visible artifacts. for blaze grating bigger step sizes possible. 
% blaze steps
\begin{figure}[H]
  \centering
  \subfigure[$\lambda_{step=1 nm}$]{
    \includegraphics[scale=0.12]{results/different_lambda_steps/blaze/dl=1.png}
    \label{fig:brdfmapsDiffLambdaStepsL1Blaze}
  }
~
  \subfigure[$\lambda_{step=5 nm}$]{
    \includegraphics[scale=0.12]{results/different_lambda_steps/blaze/dl=5.png}
    \label{fig:brdfmapsDiffLambdaStepsL5Blaze}
  }
~
  \subfigure[$\lambda_{step=10 nm}$]{
    \includegraphics[scale=0.12]{results/different_lambda_steps/blaze/dl=10.png}
    \label{fig:brdfmapsDiffLambdaStepsL10Blaze}
  }
  
  \subfigure[$\lambda_{step=25 nm}$]{
    \includegraphics[scale=0.12]{results/different_lambda_steps/blaze/dl=25.png}
    \label{fig:brdfmapsDiffLambdaStepsL25Blaze}
  }
~
  \subfigure[$\lambda_{step=50 nm}$]{
    \includegraphics[scale=0.12]{results/different_lambda_steps/blaze/dl=50.png}
    \label{fig:brdfmapsDiffLambdaStepsL50Blaze}
  }
~ 
  \subfigure[$\lambda_{step=100 nm}$]{
    \includegraphics[scale=0.12]{results/different_lambda_steps/blaze/dl=100.png}
    \label{fig:brdfmapsDiffLambdaStepsL100Blaze}
  }
  
  \label{brdfmapsDiffLambdaStepsBlaze}
  \caption{Blaze grating at $2.5 \mu m$: Different $\lambda$ step sizes}
\end{figure}

% elpahe steps
\begin{figure}[H]
  \centering
  \subfigure[$\lambda_{step=1 nm}$]{
    \includegraphics[scale=0.12]{results/different_lambda_steps/elaphe65/dl=1.png}
    \label{fig:brdfmapsDiffLambdaStepsL1Elaphe65}
  }
~
  \subfigure[$\lambda_{step=5 nm}$]{
    \includegraphics[scale=0.12]{results/different_lambda_steps/elaphe65/dl=5.png}
    \label{fig:brdfmapsDiffLambdaStepsL5Elaphe65}
  }
~
  \subfigure[$\lambda_{step=10 nm}$]{
    \includegraphics[scale=0.12]{results/different_lambda_steps/elaphe65/dl=10.png}
    \label{fig:brdfmapsDiffLambdaStepsL10Elaphe65}
  }
  
  \subfigure[$\lambda_{step=25 nm}$]{
    \includegraphics[scale=0.12]{results/different_lambda_steps/elaphe65/dl=25.png}
    \label{fig:brdfmapsDiffLambdaStepsL25Elaphe65}
  }
~
  \subfigure[$\lambda_{step=50 nm}$]{
    \includegraphics[scale=0.12]{results/different_lambda_steps/elaphe65/dl=50.png}
    \label{fig:brdfmapsDiffLambdaStepsL50Elaphe65}
  }
~ 
  \subfigure[$\lambda_{step=100 nm}$]{
    \includegraphics[scale=0.12]{results/different_lambda_steps/elaphe65/dl=100.png}
    \label{fig:brdfmapsDiffLambdaStepsL100Elaphe65}
  }
  
  \label{brdfmapsDiffLambdaStepsElaphe65}
  \caption{Elaphe grating at $65 \mu m$: Different $\lambda$ step sizes}
\end{figure}


compare pq qith full lambda sampling. see similarities but also differences 
%pq blaze
\begin{figure}[H]
  \centering
  \subfigure[Full Lambda Sampling: Blaze grating]{
    \includegraphics[scale=0.12]{results/PQapproach_vs_sampleAll/blaze/fftBlazeHeight_0.25Microns_allL_weak_scale_g=1.1.png}
    \label{fig:fullLambdaBlaze}
  }
~
  \subfigure[Full Lambda Sampling brightened: Blaze grating]{
    \includegraphics[scale=0.12]{results/PQapproach_vs_sampleAll/blaze/fftBlazeHeight_0.25Microns_allL_weak_scale_g=2.2_scale=100.png}
    \label{fig:fullLambdaBrightenedBlaze}
  }
~
  \subfigure[PQ Approach: Blaze grating]{
    \includegraphics[scale=0.12]{results/PQapproach_vs_sampleAll/blaze/pq.png}
    \label{fig:pqBlaze}
  }
  \label{pqBlaze}
  \caption{Blaze grating: PQ approach vs full lambda space sampling}
\end{figure}


%pq xeno
\begin{figure}[H]
  \centering
  \subfigure[Full Lambda Sampling: Xeno grating $\theta_i = 0 degree$]{
    \includegraphics[scale=0.19]{results/PQapproach_vs_sampleAll/xeno/a_xeno_t_i=0.png}
    \label{fig:fullLambdaXenoti0}
  }
~
  \subfigure[PQ Approach: Xeno grating $\theta_i = 0 degree$]{
    \includegraphics[scale=0.19]{results/PQapproach_vs_sampleAll/xeno/a_pq_t_i=0.png}
    \label{fig:pqXenoti0}
  }

  \subfigure[Full Lambda Sampling: Xeno grating $\theta_i = 10 degree$]{
    \includegraphics[scale=0.19]{results/PQapproach_vs_sampleAll/xeno/b_xeno_t_i=10.png}
    \label{fig:fullLambdaXenoti10}
  }
~
  \subfigure[PQ Approach: Xeno grating $\theta_i = 10 degree$]{
    \includegraphics[scale=0.19]{results/PQapproach_vs_sampleAll/xeno/b_pq_t_i=10.png}
    \label{fig:pqXenoti10}
  }
  
  \subfigure[Full Lambda Sampling: Xeno grating $\theta_i = 20 degree$]{
    \includegraphics[scale=0.19]{results/PQapproach_vs_sampleAll/xeno/c_xeno_t_i=20.png}
    \label{fig:fullLambdaXenoti20}
  }
~
  \subfigure[PQ Approach: Xeno grating $\theta_i = 20 degree$]{
    \includegraphics[scale=0.19]{results/PQapproach_vs_sampleAll/xeno/c_pq_t_i=20.png}
    \label{fig:pqXenoti20}
  }
  \label{pqXeno}
  \caption{Xeno grating: PQ approach vs full lambda space sampling}
\end{figure}

%pq elaphe
\begin{figure}[H]
  \centering
  \subfigure[Full Lambda Sampling: Elaphe grating]{
    \includegraphics[scale=0.2]{results/PQapproach_vs_sampleAll/elaphe/elaph65.png}
    \label{fig:fullLambdaElaphe}
  }
~
  \subfigure[PQ Approach: Elaphe grating]{
    \includegraphics[scale=0.2]{results/PQapproach_vs_sampleAll/elaphe/pq.png}
    \label{fig:pqElaphe}
  }

  \label{pqElaphe}
  \caption{Elaphe grating: PQ approach vs full lambda space sampling}
\end{figure}

blaze grating lower coherence length, fewer interacting grating periods, produce blurred diffraction bands for different lambda which overlap to produce poorly resolved colors.

%sigma var
\begin{figure}[H]
  \centering
  \subfigure[$\sigma_{s=3.25 \mu m}$]{
    \includegraphics[scale=0.12]{results/sigma_sVariation/blaze/sigma_s=3.25.png}
    \label{fig:brdfmapsDiffSigmaStepsL1Blaze}
  }
~
  \subfigure[$\sigma_{s=6.5 \mu m}$]{
    \includegraphics[scale=0.12]{results/sigma_sVariation/blaze/sigma_s=6.5.png}
    \label{fig:brdfmapsDiffSigmaStepsL5Blaze}
  }
~
  \subfigure[$\sigma_{s=15 \mu m}$]{
    \includegraphics[scale=0.12]{results/sigma_sVariation/blaze/sigma_s=15.png}
    \label{fig:brdfmapsDiffSigmaStepsL10Blaze}
  }
  
  \subfigure[$\sigma_{s=30 \mu m}$]{
    \includegraphics[scale=0.12]{results/sigma_sVariation/blaze/sigma_s=30.png}
    \label{fig:brdfmapsDiffSigmaStepsL25Blaze}
  }
~
  \subfigure[$\sigma_{s=45 \mu m}$]{
    \includegraphics[scale=0.12]{results/sigma_sVariation/blaze/sigma_s=45.png}
    \label{fig:brdfmapsDiffSigmaStepsL50Blaze}
  }
~ 
  \subfigure[$\sigma_{s=65 \mu m}$]{
    \includegraphics[scale=0.12]{results/sigma_sVariation/blaze/sigma_s=65.png}
    \label{fig:brdfmapsDiffSigmaStepsL100Blaze}
  }
  
  \label{brdfmapsDiffSigmaSizeBlaze}
  \caption{Blaze grating at $2.5 \mu m$: Different $\sigma_s$ sizes}
\end{figure}

visually convergence taylor series for higher values for N.

%taylor var elaphe
\begin{figure}[H]
  \centering
  \subfigure[$N=0$]{
    \includegraphics[scale=0.06]{results/taylorStepsVar/elaphe65/0.png}
    \label{fig:brdfmapsTaylorN0Elaphe65}
  }
~
  \subfigure[$N=1$]{
    \includegraphics[scale=0.06]{results/taylorStepsVar/elaphe65/1.png}
    \label{fig:brdfmapsTaylorN1Elaphe65}
  }
~
  \subfigure[$N=2$]{
    \includegraphics[scale=0.06]{results/taylorStepsVar/elaphe65/2.png}
    \label{fig:brdfmapsTaylorN2Elaphe65}
  }
~  
  \subfigure[$N=3$]{
    \includegraphics[scale=0.06]{results/taylorStepsVar/elaphe65/3.png}
    \label{fig:brdfmapsTaylorN3Elaphe65}
  }
~
  \subfigure[$N=4$]{
    \includegraphics[scale=0.06]{results/taylorStepsVar/elaphe65/4.png}
    \label{fig:brdfmapsTaylorN4Elaphe65}
  }

  \subfigure[$N=5$]{
    \includegraphics[scale=0.06]{results/taylorStepsVar/elaphe65/5.png}
    \label{fig:brdfmapsTaylorN5Elaphe65}
  }
~  
  \subfigure[$N=6$]{
    \includegraphics[scale=0.06]{results/taylorStepsVar/elaphe65/6.png}
    \label{fig:brdfmapsTaylorN6Elaphe65}
  }
~  
  \subfigure[$N=7$]{
    \includegraphics[scale=0.06]{results/taylorStepsVar/elaphe65/7.png}
    \label{fig:brdfmapsTaylorN7Elaphe65}
  }
~  
  \subfigure[$N=8$]{
    \includegraphics[scale=0.06]{results/taylorStepsVar/elaphe65/8.png}
    \label{fig:brdfmapsTaylorN8Elaphe65}
  }
~ 
  \subfigure[$N=9$]{
    \includegraphics[scale=0.06]{results/taylorStepsVar/elaphe65/9.png}
    \label{fig:brdfmapsTaylorN9Elaphe65}
  }
  
  \label{brdfmapsTaylorIterationsElaphe65}
  \caption{Elaphe grating at $65 \mu m$: $N$ Taylor Iterations}
\end{figure}


%taylor var blaze
\begin{figure}[H]
  \centering
  \subfigure[$N=0$]{
    \includegraphics[scale=0.06]{results/taylorStepsVar/blaze/0.png}
    \label{fig:brdfmapsTaylorN0Blaze}
  }
~
  \subfigure[$N=1$]{
    \includegraphics[scale=0.06]{results/taylorStepsVar/blaze/1.png}
    \label{fig:brdfmapsTaylorN1Blaze}
  }
~
  \subfigure[$N=2$]{
    \includegraphics[scale=0.06]{results/taylorStepsVar/blaze/2.png}
    \label{fig:brdfmapsTaylorN2Blaze}
  }
~  
  \subfigure[$N=3$]{
    \includegraphics[scale=0.06]{results/taylorStepsVar/blaze/3.png}
    \label{fig:brdfmapsTaylorN3Blaze}
  }
~
  \subfigure[$N=4$]{
    \includegraphics[scale=0.06]{results/taylorStepsVar/blaze/4.png}
    \label{fig:brdfmapsTaylorN4Blaze}
  }

  \subfigure[$N=5$]{
    \includegraphics[scale=0.06]{results/taylorStepsVar/blaze/5.png}
    \label{fig:brdfmapsTaylorN5Blaze}
  }
~  
  \subfigure[$N=6$]{
    \includegraphics[scale=0.06]{results/taylorStepsVar/blaze/6.png}
    \label{fig:brdfmapsTaylorN6Blaze}
  }
~  
  \subfigure[$N=7$]{
    \includegraphics[scale=0.06]{results/taylorStepsVar/blaze/7.png}
    \label{fig:brdfmapsTaylorN7Blaze}
  }
~  
  \subfigure[$N=8$]{
    \includegraphics[scale=0.06]{results/taylorStepsVar/blaze/8.png}
    \label{fig:brdfmapsTaylorN8Blaze}
  }
~ 
  \subfigure[$N=9$]{
    \includegraphics[scale=0.06]{results/taylorStepsVar/blaze/9.png}
    \label{fig:brdfmapsTaylorN9Blaze}
  }
  
  \label{brdfmapsTaylorIterationsBlaze}
  \caption{Blaze grating at $2.5 \mu m$: $N$ Taylor Iterations}
\end{figure}

% brdf maps xeno angles
\begin{figure}[H]
  \centering
  \subfigure[Xeno grating $\theta_i=0$]{
    \includegraphics[scale=0.12]{results/different_theta_i_angles/xenopeltis/xeno_t_i=0.png}
    \label{fig:brdfmapXenoti0}
  }
~
  \subfigure[Xeno grating $\theta_i=10$]{
    \includegraphics[scale=0.12]{results/different_theta_i_angles/xenopeltis/xeno_t_i=10.png}
    \label{fig:brdfmapXenoti10}
  }
~
  \subfigure[Xeno grating $\theta_i=20$]{
    \includegraphics[scale=0.12]{results/different_theta_i_angles/xenopeltis/xeno_t_i=20.png}
    \label{fig:brdfmapXenoti20}
  }
  \label{brdfmapsXenoDiffThetaIAngles}
  \caption{BRDF maps for Xeno grating: different $\theta_i$ angles}
\end{figure}


\section{Snake surface geometries}
initially using fully lambda shader for those renderings, slow but accurate.

\begin{figure}[H]
  \centering
  \subfigure[Blaze grating]{
    \includegraphics[scale=0.12]{results/snakerenderings/compars/blaze.png}
    \label{fig:renderingBlazeGrating}
  }
~
  \subfigure[Elaphe grating]{
    \includegraphics[scale=0.12]{results/snakerenderings/compars/elaphe65.png}
    \label{fig:renderingElapheGrating}
  }
~
  \subfigure[Xeno grating]{
    \includegraphics[scale=0.12]{results/snakerenderings/compars/xeno65.png}
    \label{fig:renderingXenoGrating}
  }
  \label{renderingDifferentSnankeGratings}
  \caption{Diffraction of different snake skin gratings rendered on a snake geometry}
\end{figure}


\begin{figure}[H]
  \centering
  \subfigure[$zoom = 0.1$]{
    \includegraphics[scale=0.2]{results/snakerenderings/zoomIn/elaphe65/0.1.png}
    \label{fig:renderingZoomElaphe01}
  }
~
  \subfigure[$zoom = 0.2$]{
    \includegraphics[scale=0.2]{results/snakerenderings/zoomIn/elaphe65/0.2.png}
    \label{fig:renderingZoomElaphe02}
  }
  
  \subfigure[$zoom = 0.5$]{
    \includegraphics[scale=0.2]{results/snakerenderings/zoomIn/elaphe65/0.5.png}
    \label{fig:renderingZoomElaphe05}
  }
~
  \subfigure[$zoom = 1.0$]{
    \includegraphics[scale=0.2]{results/snakerenderings/zoomIn/elaphe65/1.png}
    \label{fig:renderingZoomElaphe1}
  }
  
  \subfigure[$zoom = 1.5$]{
    \includegraphics[scale=0.2]{results/snakerenderings/zoomIn/elaphe65/1.5.png}
    \label{fig:renderingZoomElaphe15}
  }
~
  \subfigure[$zoom = 2.0$]{
    \includegraphics[scale=0.2]{results/snakerenderings/zoomIn/elaphe65/2.png}
    \label{fig:renderingZoomElaphe2}
  }
  \label{renderingDifferentZoomLevelsElaphe}
  \caption{Diffraction on Elaphe snake skin grating: Different camera zoom levels}
\end{figure}

% first 3 move along x, next 3 move along y
\begin{figure}[H]
  \centering
  \subfigure[$(-3.3130, 0.0, -0.9999)$]{
    \includegraphics[scale=0.12]{results/snakerenderings/rotateLight/elaphe65/moveX/2.png}
    \label{fig:renderingElapheRotX2}
  }
~
  \subfigure[$(-0.1989, 0.0, -0.9799)$]{
    \includegraphics[scale=0.12]{results/snakerenderings/rotateLight/elaphe65/moveX/4.png}
    \label{fig:renderingElapheRotX4}
  }
~
  \subfigure[$(-0.3897, 0.0, -0.9208)$]{
    \includegraphics[scale=0.12]{results/snakerenderings/rotateLight/elaphe65/moveX/6.png}
    \label{fig:renderingElapheRotX6}
  }
  
  \subfigure[$(0.0995, 0.0993, -0.9900)$]{
    \includegraphics[scale=0.12]{results/snakerenderings/rotateLight/elaphe65/moveY/2.png}
    \label{fig:renderingElapheRotY2}
  }
~
  \subfigure[$(0.0995, 0.2940, -0.9505)$]{
    \includegraphics[scale=0.12]{results/snakerenderings/rotateLight/elaphe65/moveY/4.png}
    \label{fig:renderingElapheRotY4}
  }
~
  \subfigure[$(0.0995, 0.4770, -0.8731)$]{
    \includegraphics[scale=0.12]{results/snakerenderings/rotateLight/elaphe65/moveY/6.png}
    \label{fig:renderingElapheRotY6}
  }
  
  \label{renderingElapheLightRotations6}
  \caption{Diffraction on Elaphe snake skin grating: Different light directions}
\end{figure}

\begin{figure}[H]
  \centering
  \subfigure[Diffraction Patten]{
    \includegraphics[scale=0.2]{results/snakerenderings/elaphe65/1.png}
    \label{fig:renderingElaphe65DP}
  }
~
  \subfigure[Diffraction + Texture]{
    \includegraphics[scale=0.2]{results/snakerenderings/elaphe65/2.png}
    \label{fig:renderingElaphe65DT}
  }

  \subfigure[Texture + Lightdir]{
    \includegraphics[scale=0.12]{results/snakerenderings/elaphe65/3.png}
    \label{fig:renderingElaphe65TL}
  }
~
  \subfigure[Nanostructure]{
    \includegraphics[scale=0.10]{results/snakerenderings/elaphe65/4.png}
    \label{fig:renderingElaphe65NS}
  }
~ 
  \subfigure[Fourier Transform]{
    \includegraphics[scale=0.52]{results/snakerenderings/elaphe65/5.png}
    \label{fig:renderingElaphe65FT}
  }
  
  
  \label{renderingElaphe65}
  \caption{Diffraction for Elaphe snake skin}
\end{figure}


\begin{figure}[H]
  \centering
  \subfigure[Diffraction Patten]{
    \includegraphics[scale=0.2]{results/snakerenderings/xeno65/1.png}
    \label{fig:renderingXeno65DP}
  }
~
  \subfigure[Diffraction + Texture]{
    \includegraphics[scale=0.2]{results/snakerenderings/xeno65/2.png}
    \label{fig:renderingXeno65DT}
  }

  \subfigure[Texture + Lightdir]{
    \includegraphics[scale=0.12]{results/snakerenderings/xeno65/3.png}
    \label{fig:renderingXeno65TL}
  }
~
  \subfigure[Nanostructure]{
    \includegraphics[scale=0.075]{results/snakerenderings/xeno65/4.png}
    \label{fig:renderingXeno65NS}
  }
~ 
  \subfigure[Fourier Transform]{
    \includegraphics[scale=0.52]{results/snakerenderings/xeno65/5.png}
    \label{fig:renderingXeno65FT}
  }
  
  
  \label{renderingXeno65}
  \caption{Diffraction for Xeno snake skin}
\end{figure}


\section{Snake surface geometries}

\begin{figure}[H]
  \centering
  \subfigure[Simulation]{
    \includegraphics[scale=0.2]{results/experiment/elaphe/g2.png}
    \label{fig:renderingElaphe65Good}
  }

  \subfigure[Experiment]{
    \includegraphics[scale=0.32]{results/experiment/elaphe/e1.png}
    \label{fig:experimentElaphe65}
  }

  \label{renderingVsExperimentElaphe65}
  \caption{Diffraction Elpahe: simulation vs. experimental setup}
\end{figure}


