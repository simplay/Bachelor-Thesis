\chapter{Theoretical Background}
\section{Basics in Modeling Light in Computer Graphics}

\subsection{Radiometry}
One purpose of Computer Graphics is to simulate the interaction of light on a surface and how a real-world observer, such as a human eye, will perceive this. These visual sensations of an eye are modeled relying on a virtual camera which captures the emitted light from the surface. The physical basis to measure such reflected light depicts radiometry which is about measuring the electromagnetic radiation transfered from a soruce to a receiver. 

Fundamentally, light is a form of energy propagation, consisting of a large collection of photons, whereat each photon can be considered as a quantum of light that has a position, direction of propagation and a wavelength $\lambda$. A photon travels at a certain speed $v = \frac{c}{n}$, that depends only the speed of light $c$ and the refractive index $n$ through which it progrates. Its frequency is defined by $f = \frac{v}{\lambda}$ and its carried amount of energy $q$, mearsured in the SI unit Joule, is given by $q = hf= \frac{hv}{\lambda n}$ where $h$ is the Plank's constant. The total energy of a large collection of photons is hence $Q = \sum_i q_i$.

\subsection{Spectral Energy}
It is important to understand that the human eye is not equally sensitive to all wavelength of the spectrum of light and therefore responds differently to specific wavelengths. Remember that our goal is to model the human visual perception. This is why we consider the energy distribution of a light spectrum rather than considering the total energry of a photon collection since then we could weight the distribution according the human visual system. So the question we want to answer is: How is the energy distributed across wavelengths of light?

The idea is to make an energy histrogram from a given photon collection. For this we have to order all photons by their associated wavelength, discretize wavelength spectrum, count all photons which then will fall in same wavelength-interval, and then, finally, normalize each interval by the total energy $Q$. This will give us a histogram which tells us the relative energy $Q_{\lambda}$ for a given discrete $\lambda$ interval and thus models an energy distribution $\footnote{Intensive quantities can be thought of as density functions that tell the density of an extensive quantity at an infinitesimal point.}$.

\subsection{Spectral Power}
\subsection{Spectral Irradiance}
\subsection{Spectral Radiance}
\subsection{BRDF}
\subsection{Colorspace}
\subsection{Spectral Rendering}

\section{Wave Theory for Light and Diffraction}
\section{Stam's BRDF formulation}