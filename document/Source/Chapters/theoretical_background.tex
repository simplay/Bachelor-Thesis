\chapter{Theoretical Background}
\section{Basics in Modeling Light in Computer Graphics}

\subsection{Radiometry}
One purpose of Computer Graphics is to simulate the interaction of light on a surface and how a real-world observer, such as a human eye, will perceive this. These visual sensations of an eye are modeled relying on a virtual camera which captures the emitted light from the surface. The physical basis to measure such reflected light depicts radiometry which is about measuring the electromagnetic radiation transfered from a soruce to a receiver. 

Fundamentally, light is a form of energy propagation, consisting of a large collection of photons, whereat each photon can be considered as a quantum of light that has a position, direction of propagation and a wavelength $\lambda$. A photon travels at a certain speed $v = \frac{c}{n}$, that depends only the speed of light $c$ and the refractive index $n$ through which it progrates. Its frequency is defined by $f = \frac{v}{\lambda}$ and its carried amount of energy $q$, mearsured in the SI unit Joule, is given by $q = hf= \frac{hv}{\lambda n}$ where $h$ is the Plank's constant. The total energy of a large collection of photons is hence $Q = \sum_i q_i$.

\subsection{Spectral Energy}

\subsection{Spectral Power}
\subsection{Spectral Irradiance}
\subsection{Spectral Radiance}
\subsection{BRDF}
\subsection{Colorspace}
\subsection{Spectral Rendering}

\section{Wave Theory for Light and Diffraction}
\section{Stam's BRDF formulation}