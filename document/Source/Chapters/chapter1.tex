\chapter{Introduction}
\section{Motivation}
As human beings, we visually perceive and experience our whole world in terms of colors resulting from various physical phenomena that involve interaction between light and matter. Particularly in nature, there are basically two main causes for color production. Firstly, due to pigmentation, which occurs since certain molecules in a biological structure selectively absorb or reflect specific wavelengths from an incident light source. And secondly because of structural colors, which are the result of physical interaction of light with a nanostructure, exclusively relying on the structuring of the material and not on any other property. A natural diffraction grating is a semitransparent layer of biological nano-structures that exhibits a certain degree of regularity to produce structural colors by diffracting an incident light beam. One particular example for such biological color production includes the spectacular colors that we can see while having a closer look at the illuminated skin of snakes, as shown in figure $\ref{fig:snakespecies}$.

\begin{figure}[H]
  \centering
  \subfigure[Xenopeltis snake]{
    \includegraphics[scale=0.47]{introduction/xenosnake.png}
    \label{fig:xenospeicies}
  }
~
  \subfigure[Elaphe Guttata snake]{
    \includegraphics[scale=0.47]{introduction/elaphesnake.png}
    \label{fig:elpahespecies}
  }
  \caption[Example of Biological Color Production]{Examples of pigmentation color (green boxes) and structural color (blue boxes) on different snake species$\footnotemark$.}
  \label{fig:snakespecies}
\end{figure}
\footnotetext{The image source for figure $\ref{fig:xenospeicies}$ is \texttt{http://www.snakes-alive.co.uk/gallery\textunderscore 5.html} and for figure $\ref{fig:elpahespecies}$ is \texttt{http://www.the-livingrainforest.co.uk/living/view\textunderscore price.php?id=464}} 
\noindent
Some species, like the Xenopeltis, express structural colors in form of iridescent patterns along their scales way stronger than others such as the Elaphe species. The reason lies in the nanostructure of their skins. There are a vast amount of additional reasons for structural color occurrences in nature, such as thin film interference, intra-cellular photonic crystals or diffraction gratings. More detailed examples are shown in figure $\ref{fig:structuralcolorexamples}$. 

\begin{figure}[H]
  \centering
  \subfigure[Thin film interference on a soap bubble]{
    \includegraphics[scale=0.6]{background/soapbubble.png}
    \label{fig:soapbubble}
  }
~
  \subfigure[Multilayer interference on abdomen of a beetle]{
    \includegraphics[scale=0.6]{background/beetle.png}
    \label{fig:beetle}
  }
~
  \subfigure[Photonic crystals in the wings of a butterfly]{
    \includegraphics[scale=0.6]{background/butterflypc.png}
    \label{fig:butterflyphotoniccristals}
  }
~
  \subfigure[Structural colors on peacock feathers]{
    \includegraphics[scale=1.0]{introduction/peacock.jpg}
    \label{fig:peacockfeather}
  }


  \subfigure[Scattering of light from a butterfly's wings]{
    \includegraphics[scale=0.6]{background/butterflyscattering.png}
    \label{fig:butterflyscattering}
  }
~
  \subfigure[Synthetic diffraction grating on a CD]{
    \includegraphics[scale=0.6]{background/cd.png}
    \label{fig:cddiffractiongrating}
  } 
~
  \subfigure[Natural diffraction grating on a snake skin]{
    \includegraphics[scale=0.6]{background/snakeskin.png}
    \label{fig:snakediffractiongrating}
  }   
  
  \caption[Structural color examples]{Examples$\footnotemark$ for structural colors on wings and abdomen of insects, liquids, synthetic structures, and on scales on the skin of reptiles.}
  \label{fig:structuralcolorexamples}
\end{figure}
\footnotetext{image source for figures:
\begin{itemize}
  \item \ref{fig:soapbubble}: \texttt{http://www.ualberta.ca/\textasciitilde pogosyan/teaching/PHYS\textunderscore 130/FALL\textunderscore 2010/lectures/lect33/lecture33.html}
  \item \ref{fig:beetle}: \texttt{http://www.itp.uni-hannover.de/\textasciitilde zawischa/ITP/multibeam.html}
  \item \ref{fig:butterflyphotoniccristals}: \texttt{http://upload.wikimedia.org/wikipedia/commons/a/a4/Parides\textunderscore sesostris\textunderscore MHNT\textunderscore dos.jpg}
  \item \ref{fig:peacockfeather}: \texttt{http://en.wikipedia.org/wiki/Structural\textunderscore coloration}
  \item \ref{fig:butterflyscattering}: From paper \cite{struccolor}, figure 6.
  \item \ref{fig:cddiffractiongrating}: \texttt{http://cnx.org/content/m42496/latest/?collection=col11428/latest}
  \item \ref{fig:snakediffractiongrating}: \texttt{http://www.snakes-alive.co.uk/gallery\textunderscore 5.html}
  \end{itemize}
}
\noindent
As far back as in the 17th century, Robert Hooke was able to relate the cause of structural colors to the microstructure of a material. During his examinations of peacock feathers (see figure $\ref{fig:peacockfeather}$), he found that by wetting the feathers, their colors disappeared. Further, he observed that the feathers have tiny ridges. Later, building on top of the knowledge about interference at this times, Newton related structural colors to wave interference. Recently, in the field of computer graphics, many researchers have developed models to render structural colors. Most of these currently available models are not able to perform interactive rendering or are oversimplified and thus cannot accurately model the effects of diffraction from natural gratings. \\

This thesis investigates this particular problem in detail and provides a solution for rendering structural colors that are caused by diffraction on natural gratings at interactive rates.

\section{Goals}
The purpose of this thesis is, to simulate physically accurate structural colors caused by the effect of diffraction on various biological structures and then implement this simulation as a renderer with interactive behaviour. We mainly focus on structural colors generated by natural diffraction gratings. In particular the approach presented in this thesis applies to surfaces with quasiperiodic structures at the nanometer scale, which can be represented as height fields and are visualized as grayscale images. \\

Natural gratings are found on the scale of reptiles, wings of butterflies or the bodies of various insects, but we restrict ourselves and focus on snake skins. The data of our discrete valued height fields, which are representing the surface of a measured snake skin, is acquired using atomic force microscopy (AFM)$\footnote{All data is provided by the Laboratory of Artificial and Natural Evolution in Geneva. See their website:\texttt{www.lanevol.org}}$. Figure $\ref{fig:xenopeltisafm}$ shows a measured height field of a Xenopeltis snake as a grayscale image. The surface of its skin is composed of many finger like structures. Locally, these fingers seem to be very regularly aligned (red box). However, globally, we observe that the alignment of the fingers is curved irregularly (indicated by green curves) along the whole surface. This kind of global irregularity is what makes it hard to model the structural complexity of natural gratings$\footnote{E.g. by relying on statistical methods, caputuring surface details by introducing an appropriate distribution function of the finger strucures.}$.

\begin{figure}[H]
  \centering
  \includegraphics[scale=0.5]{background/samplenaturalgrating.png}
  \caption[Xenopeltis AFM image]{Height field of a Xenopeltis snake$\footnotemark$ skin taken using AFM. Locally, this natural grating consists of regularly aligned (red box) finger-like substructures, but globally we observe a curved alignment of these structures (green curves).}
  \label{fig:xenopeltisafm}
\end{figure}
\footnotetext{This image was provided by the LANE lab in Geneva}
\noindent
Our renderer established in this thesis is based on the pioneering work of J. Stam about diffraction shaders $\cite{diffstam}$. Stam formulated a BRDF modelling the effect of diffraction based on certain statistical properties of height fields. Nevertheless we have to adapt his BRDF model, since his model assumes that a given surface of a grating can either be formulated by an analytical function and therefore has a closed form solution, or it is modelled effectively by relying on statistical methods. However, we are dealing with natural diffraction gratings (represented as explicitly formulated height fields), which unfortunately are neither known analytically, nor do they fit into simple statistical models. This thesis thus proposes an extension to J. Stam's work for the complex case of explicitly defined, discrete and quasi-periodic height field structures. \\

In the following section a brief overview of relevant previous work and related to this thesis will be presented.

\section{Previous work}
The first scientific descriptions of structural colors were provided by Hooke in 1665 in his book Micrographia$\cite{hookemicro}$. Hooke investigated feathers of peacocks using one of the first microscopes from his time and found out that the colors on the feather were canceled out whenever a drop of water moistened the feather. He proposed the speculation that a layer of thin plates and air were responsible for reflecting the light and thus he related the structure of the feather to colors. In Newton's book Opticks$\cite{newtonopticks}$ he described that the colors of the peacock feather are related to the thinness of the transparent part of the feathers. Around 1800, T. Young demonstrated wave interference using his double-slit experiment$\footnote{See \texttt{http://en.wikipedia.org/wiki/Double-slit\textunderscore experiment}}$$\cite{doublslit}$. In his work Young explained light as a result of wave interference. He also related structural colors to the concept of wave interference. He published his findings in the journal Philosophical Transactions of the Royal Society. \\

In the field of computer graphics, J.Stam$\cite{diffstam}$ developed a popular reflection model that could produce colorful diffraction effects, based on certain statistical properties of a given height field. His model is an approximation of far field diffraction$\footnote{See \texttt{http://en.wikipedia.org/wiki/Fraunhofer\textunderscore diffraction}}$ effects relying on the Kirchhof integral$\footnote{See \texttt{http://en.wikipedia.org/wiki/Kirchhoff\textunderscore integral\textunderscore theorem}}$. For a certain class of surfaces which can be modelled as a height field he provides an analytical solution of the BRDF model. He assumes homogeneity of the structure and the main idea of his paper is to formulate the BRDF as a function of the Fourier Transform of certain correlation functions relating to the given height field. However, the height fields that Stam is dealing with are either extremely regular or can be considered as a superposition of randomly distributed bumps forming a periodic like structure relying on probabilistic distribution theory$\footnote{See \texttt{http://en.wikipedia.org/wiki/Probability\textunderscore distribution}}$. Either of assumptions allow him to derive an analytical solution using statistical models. However, the height field we are dealing with are measured, complex, biological nano-structures and thus they do not exhibit regularity at a global scale as demonstrated in figure $\ref{fig:xenopeltisafm}$. Superimposing one particular nano finger (considering it as a bump) cannot capture the complexity of the measured structure since this structure is globally highly irregular aligned. Thus, using only one finger poses a non-trivial problem of modelling the distribution of nano fingers statistically. Therefore, we cannot directly use Stam's BRDF model when we want to enable interactive rendering for diffraction effects of natural gratings.\\

In 2012 Cuypers et all $\cite{reflectancediffmodel}$ proposed a wave based Bidirectional scattering distribution function (BSDF$\footnote{See \texttt{http://en.wikipedia.org/wiki/Bidirectional\textunderscore scattering\textunderscore distribution\textunderscore function}}$) denoted as WBSDF.
Using the rendering equation and Wigner Distribition Functions$\footnote{See \texttt{http://en.wikipedia.org/wiki/Wigner\textunderscore distribution\textunderscore function}}$ (WDF) they related their WBSDF model to the incoming wavefront and hence, their model can be adapted such that it can be rendered by a Monte Carlo renderer. The advantage of their model over Stam's is that their models also captures near field diffraction effects. A disadvantage of their model is that it is computational expensive since the WDF of a two dimensional surface is a four dimensional function and therefore can hardly be used in order to perform interactive rendering. \\

Linday and Agu $\cite{reflectancediffmodel}$ proposed an approach in order to perform interactive rendering diffraction effect by precomputing and storing their BRDF model using spherical harmonics. Nonetheless, for complex natural gratings their BRDF may be insufficient accurate since their approach is using low order spherical harmonics.

\section{Thesis Structure}
The reminder of this thesis is organised as follows: Due to the fact that this thesis has a rather advanced mathematical complexity, chapter 2 introduces some important definitions about modelling light in computer graphics and some wave theory. These concept are required in order to be able to follow our later derivations. This is followed by a brief summary of J. Stam's Paper about diffraction shaders, since his BRDF formulation is the basis of our derivations. \\

In chapter 3 we adapt Stam's BRDF model step-wise in a way that we derive a representation which can be implemented as an interactive diffraction renderer for natural diffraction gratings. We also propose an alternative formulation, which we refer to as the PQ approach in this chapter and discuss its short-comings. \\

Chapter 4 addresses the practical part of this thesis, the implementation of our diffraction model, explaining all precomputation steps and how rendering is preformed in our reference framework for this thesis. \\

Chapter 5 deals with the evaluation of our models. It first provides some insights about diffraction gratings. Then, within this chapter we evaluate the qualitative validity of our BRDF model when applied on different surface gratings by computing their reflectance and comparing the results to the grating equation, under similar conditions. \\

Chapter 6 presents our rendered results, first the BRDF maps for all our gratings and shading approaches under various shading parameters and then the actual renderings on a snake skin. And finally Chapter 7 contains the conclusion of this thesis discussing what has been achieved in this thesis and the drawbacks of the proposed method. It also contains a note about some of my personal experiences during this thesis.
