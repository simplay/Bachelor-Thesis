\chapter{Summary of Stam's Derivations}
\section{Taylor Series Approximation}
\label{chap:taylorseriesapproxappendix}

For an $N\mathbb{\in N}$ such that 
\begin{equation}
 \sum_{n=0}^{N}\frac{(ikwh)^{n}}{n!}\mathcal{F}\left\{ h{}^{n}\right\} (\alpha,\beta) \approx P(\alpha,\beta) 
\end{equation}
 we have to prove:
 
\begin{enumerate}
\item Show that there exist such an $N\mathbb{\in N}$s.t the approximation
holds true.
\item Find a value for B s.t. this approximation is below a certain error
bound, for example machine precision $\epsilon$. 
\end{enumerate}

\subsection{Proof Sketch of 1.}

By the \textbf{ratio test} (see \textbf{{[}1{]}}) 
It is possible to show that the series $\sum_{n=0}^{N}\frac{(ikwh)^{n}}{n!}\mathcal{F}\left\{ h{}^{n}\right\} (\alpha,\beta)$ converges absolutely:

\textbf{Proof}: Consider $\sum_{k=0}^{\infty}\frac{y^{n}}{n!}$ where
$a_{k}=\frac{y^{k}}{k!}$. By applying the definition of the ratio test for this series it follows: 

\begin{equation}
 \forall y:limsup_{k\rightarrow\infty}|\frac{a_{k+1}}{a_{k}}|=limsup_{k\rightarrow\infty}\frac{y}{k+1}=0 
\end{equation}

Thus this series converges absolutely, no matter what value we will
pick for y.

\subsection{Part 2: Find such an N}
Let $f(x)=e^{x}$. We can formulate its Taylor-Series, stated above.
Let $P_{n}(x)$denote the n-th Taylor polynom, 

\begin{equation}
 P_{n}(x)=\sum_{k=0}^{n}\frac{f^{(k)}(a)}{k!}(x-a)^{k}
\end{equation}

where $a$ is our developing point (here a is equal zero). 

We can define the error of the n-th Taylor polynom to be $E_{n}(x)=f(x)-P_{n}(x)$.
the error of the n-th Taylor polynom is difference between the value of the function and the Taylor polynomial
This directly implies $|E_{n}(x)|=|f(x)-P_{n}(x)|$. By using the Lagrangian Error Bound it follows: 

\begin{equation}
 |E_{n}(x)|\leq\frac{M}{(n+1)!}|x-a|^{n+1} 
\end{equation}

with $a=0$, where \textbf{M} is some value satisfying $|f^{(n+1)}(x)|\leq M$ on the interval $I=[a,x]$. Since we are interested in an upper bound of the error and since \textbf{a} is known, we can reformulate the interval as $I=[0,x_{max}]$, where 

\begin{equation}
 x_{max} = \|i\| k_{max} w_{max} h_{max}
\end{equation}

We are interested in computing an error bound for $e^{ikwh(x,y)}$. Assuming the following parameters and facts used within Stam's Paper: 

\begin{itemize}
\item Height of bump: 0.15micro meters
\item Width of a bump: 0.5micro meters
\item Length of a bump: 1micro meters
\item $k=\frac{2\pi}{\lambda}$ is the wavenumber, $\lambda\in[\lambda_{min,}\lambda_{max}]$ and
thus $k_{max}=\frac{2\pi}{\lambda_{min}}$. Since $(u,v,w) = -\omega_i - \omega_r$ and both are unit direction vectors, 
each component can have a value in range {[}-2, 2{]}.
\item for simplification, assume$[\lambda_{min,}\lambda_{max}]=[400nm,700nm].$

\end{itemize}

We get:  

\begin{align}
x_{max}
 &= \|i\|*k_{max}*w_{max}*h_{max} \nonumber \\
 &= k_{max}*w_{max}*h_{max} \nonumber \\
 &=2*(\frac{2\pi}{4*10^{-7}m})*1.5*10^{-7} \nonumber \\
 &=1.5\pi
\end{align}

and it follows for our interval $I=[0,1.5\pi]$. 

Next we are going to find the value for $M$. Since the exponential function is monotonically growing (on the interval I) and the derivative of the \textbf{exp} function is the exponential function itself, we can find such an $M$: 
\begin{align*}
 M
 &=e^{x_{max}} \nonumber \\
 &=exp(1.5\pi)
\end{align*}

and $|f^{(n+1)}(x)|\leq M$ holds. With 

\begin{align}
|E_{n}(x_{max})|
 &\leq\frac{M}{(n+1)!}|x_{max}-a|^{n+1} \nonumber \\
 &= \frac{exp(1.5\pi)*(1.5\pi)^{n+1}}{(n+1)!}
\end{align}

we now can find a value of $n$ for a given bound, i.e. we can find an value of $N\mathbb{\in N}$ s.t. $\frac{exp(1.5\pi)*(1.5\pi)^{N+1}}{(N+1)!}\leq\epsilon$.
With Octave/Matlab we can see: 

\begin{itemize}
\item if N=20 then $\epsilon\approx2.9950*10^{-4}$
\item if N=25 then $\epsilon\approx8.8150*10^{-8}$
\item if N=30 then $\epsilon\approx1.0050*10^{-11}$
\end{itemize}

With this approach we have that $\sum_{n=0}^{25}\frac{(ikwh)^{n}}{n!}\mathcal{F}\left\{ h{}^{n}\right\} (\alpha,\beta)$ is
an approximation of $P(u,v)$ with error $\epsilon\approx8.8150*10^{-8}$. This means we can precompute 25 Fourier Transformations in order to approximate P(u,v) having an error $\epsilon\approx8.8150*10^{-8}$. 

\section{PQ approach}
\subsection{One dimensional case}
\label{sec:pqonedimappendix}

Since our series is bounded, we can simplify the right-hand-side of equation $\ref{eq:pqgeometricseries}$.

Note that $e^{-ix}$ is a complex number. Every complex number can be written in its polar form, i.e. 

\begin{equation}
e^{-ix} = cos(x) + i sin(x) 
\label{eq:polarform}
\end{equation}

Using the following trigonometric identities
\begin{gather}
cos(-x) = cos(x) \nonumber \\
sin(-x) = -sin(x)
\end{gather}

combined with $\ref{eq:polarform}$ we can simplify the series $\ref{eq:pqgeometricseries}$ even further to:

\begin{align}
\frac{1-e^{iwT(N+1)}}{1-e^{-iwT}}
& =\frac{1-cos(wT(N+1)) + i sin(wT(N+1)) }{1-cos(wT) + i sin(wT)}
\label{eq:pq1minusexp}
\end{align}

Equation $\ref{eq:pq1minusexp}$ is still a complex number, denoted as $(p+iq)$. Generally, every complex number can be written as a fraction of two complex numbers. This implies that the complex number $(p+iq)$ can be written as $(p+iq) = \frac{(a+ib)}{(c+id)}$ for any $(a+ib), (c+id) \neq 0$. Let us use the following substitutions: 

\begin{align}
a& := 1 - cos(wT(N+1))&
b& =sin(wT(N+1)) \nonumber \\
c& =1-cos(wT)&
d& =sin(wT)
\label{eq:pqabcdsubstitudes}
\end{align}

Hence, using $\ref{eq:pqabcdsubstitudes}$, it follows 

\begin{equation}
  \frac{1-e^{iwT(N+1)}}{1-e^{-iwT}} = \frac{(a+ib)}{(c+id)}
\end{equation}

By rearranging the terms, it follows $(a+ib) = (c+id)(p+iq)$ and by multiplying its right hand-side out we get the following system of equations:

\begin{align}
(cp-dq)& =a \nonumber \\
(dp + cq)& =b
\label{eq:cdadcn}
\end{align}

After multiplying the first equation of $\ref{eq:cdadcn}$ by $c$ and the second by $d$ and then adding them together, we get using the law of distributivity new identities for $p$ and $q$:

\begin{align}
p& =\frac{(ac+bd)}{c^2 + d^2} \nonumber \\
q& =\frac{(bc+ad)}{c^2 + d^2}
\label{eq:pq1}
\end{align}

Using some trigonometric identities and putting our substitution from $\ref{eq:pqabcdsubstitudes}$ for $a$, $b$, $c$, $d$ back into the current representation $\ref{eq:pq1}$ of $p$ and $q$ we will get:

\begin{align}
p& =\frac{1}{2}+\frac{1}{2}\left(\frac{cos(wTN)-cos(wT(N+1))}{1-cos(wT)}\right) \nonumber \\
q& =\frac{sin(wT(N+1))-sin(wTN)-sin(wT)}{2(1-cos(wT))}
\end{align}

Since we have seen, that $\sum_{n=0}^N e^{-uwnT}$ is a complex number and can be written as $(p+iq)$, we now know an explicit expression for $p$ and $q$. Therefore, the one dimensional inverse Fourier transform of $S$ is equal:

\begin{align}
\mathcal{F}^{-1}\{S\}(w)
& =\mathcal{F}^{-1}\{f\}(w) \sum_{n=0}^{N} e^{-iwnT} \nonumber \\
& = (p+iq) \mathcal{F}^{-1}\{f\}(w)  
\label{eq:mainfinding1dappendix}
\end{align}


\subsection{Two dimensional case}
\label{sec:pqtwodimappendix}

\begin{align}
\mathcal{F}^{-1}\{S\}(w_1, w_2)
& = \int_{-\infty}^{\infty}\int_{-\infty}^{\infty} \sum_{n_2=0}^{N_1} \sum_{n_2=0}^{N_2} h(x_1 + n_1 T_1, x_2 + n_2 T_2) e^{iw(x_1 + x_2)}dx_1 dx_2 \nonumber \\
& = \int_{-\infty}^{\infty}\int_{-\infty}^{\infty} \sum_{n_2=0}^{N_1} \sum_{n_2=0}^{N_2} h(y_1, y_2) e^{iw((y_1 - n_1 T_1) + (y_2 + n_2 T_2))}dx_1 dx_2 \nonumber \\
& =\sum_{n_2=0}^{N_1} \sum_{n_2=0}^{N_2} \int_{-\infty}^{\infty}\int_{-\infty}^{\infty} h(y_1, y_2) e^{iw(y_1 + y_2)} e^{-iw(n_1 T_1 + n_2 T_2)}dy_1 dy_2 \nonumber \\
& =\sum_{n_2=0}^{N_1} \sum_{n_2=0}^{N_2} e^{-iw(n_1 T_1 + n_2 T_2)} \int_{-\infty}^{\infty}\int_{-\infty}^{\infty} Box(y_1, y_2) e^{iw(y_1 + y_2)} dy_1 dy_2 \nonumber \\
& =\left(\sum_{n_2=0}^{N_1} \sum_{n_2=0}^{N_2} e^{-iw(n_1 T_1 + n_2 T_2)}\right) \mathcal{F}^{-1}\{h\}(w_1,w_2) \nonumber \\
& =\left(\sum_{n_2=0}^{N_1} e^{-iw n_1 T_1}\right) \left(\sum_{n_2=0}^{N_2} e^{-iw n_2 T_2}\right) \mathcal{F}^{-1}\{h\}(w_1,w_2) \nonumber \\
& =(p_1 + i q_1)(p_2 + i q_2) \mathcal{F}^{-1}\{h\}(w_1,w_2) \nonumber \\
& =((p_1 p_2 - q_1 q_2) + i(p_1 p_2 + q_1 q_2)) \mathcal{F}^{-1}\{h\}(w_1,w_2) \nonumber \\
& =(p + iq) \mathcal{F}_{DTFT}\{h\}(w_1,w_2)
\label{eq:pqmainfindingappendix}
\end{align}

Where we have defined 

\begin{align}
p := (p_1 p_2 - q_1 q_2) \nonumber \\ 
q := (p_1 p_2 + q_1 q_2)
\label{eq:pqsubst2dappendix}
\end{align}