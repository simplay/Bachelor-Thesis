\documentclass[a4paper]{report}
\usepackage{natbib}
\bibpunct{[}{]}{,}{a}{}{;}
\usepackage{fancyheadings}
\usepackage{iamdip}
\usepackage[pdftex]{graphicx}
\usepackage{amsmath}
\usepackage{amsthm}
\usepackage{amssymb}
\usepackage{amsfonts}
\usepackage{url}
\usepackage{hyperref}
\usepackage{listings}
\usepackage{pdfpages}
\usepackage{lingmacros}
\usepackage{tree-dvips}
\usepackage{mathpazo}
\usepackage[english]{babel}
\usepackage{fontspec}
\usepackage{algpseudocode}
\usepackage{algorithm}
\usepackage{mathrsfs}
\usepackage{dsfont}
\usepackage{graphicx}
\usepackage{wasysym}
\usepackage[hypcap]{caption}
\usepackage{subfigure}
\usepackage{float}
\usepackage{mathtools}
\usepackage{mdwlist}
% \usepackage[ruled,vlined]{algorithm2e}
% \usepackage[linesnumbered]{algorithm2e}

% \SetAlCapSkip{1em}
% \SetKwInput{KwInput}{Input}
% \SetKwInput{KwOutput}{Output}

\usepackage[hmargin=3cm]{geometry}
% \usepackage{algorithm}
% \usepackage{algorithmic}
\usepackage{boxedminipage}


\renewcommand{\algorithmicforall}{\textbf{Foreach}}
\newcommand{\init}{\textbf{INIT }}
\newcommand{\pluseq}{\mathrel{+}=}
\newcommand{\asteq}{\mathrel{*}=}
\newcommand{\myto}{\textbf{TO }}
\newcommand*\colvec[3][]{
    \begin{pmatrix}\ifx\relax#1\relax\else#1\\\fi#2\\#3\end{pmatrix}
}
\newcommand{\myparagraph}[1]{\paragraph{#1}\mbox{}\\}
\DeclarePairedDelimiter\ceil{\lceil}{\rceil}
\DeclarePairedDelimiter\floor{\lfloor}{\rfloor}

\headrulewidth 0.5pt \addtolength{\headheight}{5pt}

\lhead[\fancyplain{}{\rm\thepage}]{\fancyplain{}{\rightmark}}
\rhead[\fancyplain{}{\leftmark}]{\fancyplain{}{\rm\thepage}}
\cfoot{}

\graphicspath{{../Figures/}}

\begin{document}

\pagestyle{fancyplain} \thispagestyle{empty}

\title{An Interactive Shader for Natural Diffraction Gratings}
\author{Michael Single}
\betreuer{Prof. Dr. Matthias Zwicker}
\ort{Bern}
\datum{2014}

\pagenumbering{roman} \setcounter{page}{1}
\maketitle

\newpage
\thispagestyle{empty}
\vspace{8cm}
\noindent
{\centerline {\bf \large Abstract}}
\vspace{1cm}


\noindent
%abstract
In nature color production is the result of physical interaction of light with a surface's nanostructure. In his pioneering work, Stam developed limited reflection models based on wave optics, capturing the effect of diffraction on very regular surface structures. We propose an adaption of his BRDF model such that it can handle complex natural gratings. On top of this, we describe a technique for interactively rendering diffraction effects, as a result of physical interaction of light with biological nanostructures such as snake skins. As input data, our method uses discrete height fields of natural gratings acquired by using atomic force microscopy (AFM). Based on Taylor Series approximation we leverages precomputation to achieve interactive rendering performance (about 5-15 fps). We demonstrate results of our approach using surface nanostructures of different snake species applied on a measured snake geometry. Lastly, we evaluate the qualtiy of our method by a comparision of the maxima for peak viewing angles using the data produced by our method against the maxima resulting by the grating equation.

\pagenumbering{roman} \setcounter{page}{1}
\tableofcontents

\newpage{\pagestyle{empty} \cleardoublepage}

% Hauptdokument
\pagenumbering{arabic} \setcounter{page}{1}
\pagestyle{fancy}

\chapter{Introduction}
\section{Motivation}
As a human being, we visually perceive and experience our whole world in terms of colors, resulting from various kinds of physical light interaction phenomenons. Particularly in biology, there are basically two main causes for color production. Firstly, due to pigmentation, which occures since certain molecules in a biological structure selectively absorb or reflect specific wavelengths from an incident light source. And secondly because of structural colors which are the result of physical interaction of light with a nanostructure, exclusively relying on the structring of the material and not any other property. One particular example for such biological color production are the colors we can see when having a closer look at the illuminated skin of snake species, as shown in figure $\ref{fig:snakespecies}$.

\begin{figure}[H]
  \centering
  \subfigure[Xenopeltis snake]{
    \includegraphics[scale=0.39]{background/xenopeltissnake.png}
    \label{fig:xenospeicies}
  }
~
  \subfigure[Elaphe Guttata snake]{
    \includegraphics[scale=0.39]{background/elaphesnake.png}
    \label{fig:elpahespecies}
  }
  \caption[example biological color production]{Examples of pigmentation color (green circles) and structural color (red circles) on snake species$\footnotemark$.}
  \label{fig:snakespecies}
\end{figure}
\footnotetext{image source of figure $\ref{fig:xenospeicies}$ \texttt{http://www.snakes-alive.co.uk/gallery\textunderscore5.html} and figure $\ref{fig:elpahespecies}$ \texttt{http://www.the-livingrainforest.co.uk/living/view\textunderscore price.php?id=464}} 

Comparing both snake species with each other, we observe that the Xenopeltis species expresses structural colors in form of iridescent patterns along its scales way stronger than the Elaphe species does. The reason for this lies on the nanostructure of their skins. A natural diffraction grating is a semitransparent layer of biological nanostructures which exhibits a certain degree of regularity to produce structural colors by diffracting an incident light source. There are a vast amount of additional reasons for producing structural colors in nature, such as thin film interference, intra-cellular photonic crystyls or diffraction gratings. More detailed examples are listed in figure $\ref{fig:structuralcolorexamples}$. 

\begin{figure}[H]
  \centering
  \subfigure[Thin Film Inteference soapbubble]{
    \includegraphics[scale=0.6]{background/soapbubble.png}
    \label{fig:soapbubble}
  }
~
  \subfigure[Multilayer Interference on abdomen of beetle]{
    \includegraphics[scale=0.6]{background/beetle.png}
    \label{fig:beetle}
  }
~
  \subfigure[Photonic crystals in Wings of buttefly]{
    \includegraphics[scale=0.6]{background/butterflypc.png}
    \label{fig:butterflyphotoniccristals}
  }

  \subfigure[Scattering of light on butterfly wing]{
    \includegraphics[scale=0.6]{background/butterflyscattering.png}
    \label{fig:butterflyscattering}
  }
~
  \subfigure[Artifical Diffraction Grating on CD]{
    \includegraphics[scale=0.6]{background/cd.png}
    \label{fig:cddiffractiongrating}
  } 
~
  \subfigure[Natural Diffraction Grating on snake skin]{
    \includegraphics[scale=0.6]{background/snakeskin.png}
    \label{fig:snakediffractiongrating}
  }   
  
  \caption[Structural color examples]{Examples$\footnotemark$ for structural colors on the wings and the abdomen of insects, liquids, synthetic structures, and on scales on the skin of reptiles.}
  \label{fig:structuralcolorexamples}
\end{figure}
\footnotetext{image source of figure:
\begin{itemize}
  \item \ref{fig:soapbubble}: \texttt{http://www.ualberta.ca/\textasciitilde pogosyan/teaching/PHYS\textunderscore 130/FALL\textunderscore 2010/lectures/lect33/lecture33.html}
  \item \ref{fig:beetle}: \texttt{http://www.itp.uni-hannover.de/\textasciitilde zawischa/ITP/multibeam.html}
  \item \ref{fig:butterflyphotoniccristals}: \texttt{http://upload.wikimedia.org/wikipedia/commons/a/a4/Parides\textunderscore sesostris\textunderscore MHNT\textunderscore dos.jpg}
  \item \ref{fig:butterflyscattering}: From paper \cite{struccolor}, figure 6.
  \item \ref{fig:cddiffractiongrating}: \texttt{http://cnx.org/content/m42496/latest/?collection=col11428/latest}
  \item \ref{fig:snakediffractiongrating}: \texttt{http://www.snakes-alive.co.uk/gallery\textunderscore 5.html}
  \end{itemize}
}

Already in the 17th century Robert Hooke was able to relate the cause of structural colors to the the microstructure of a material. During his examinations of peacock feathers he found that the colors could be made disappear by wetting the feathers and further observerd that the feathers are made of tiny ridges. Building on top of the latest knowledge about interference, Newton related structural colors with wave interference. Nowadays, in the field of computer graphics, many researchers have been attempted rendering structural colors but unforunately, most currently available approaches are not able to perfrom interactive rendering or are oversimplified and thus cannot model accurately the effect of diffraction.

\section{Goals}
The purpose of this thesis is to simulate realistically by rendering structural colors caused by the effect of diffraction on different biological structures in realtime. We focus on structural colors generated by diffraction gratings, in particular our approach applies to surfaces with quasiperiodic structures at the nanometer scale that can be represented as heighfields. such structures are found on the sehds of snkaes, wings of butterflies or the bodies of various insects. we restrict ourself and focus on different snake skins sheds which are acquired nanoscaled heightfields using atomic force microscopy. 

In oder to achieve our rendering purpose we will rely J. Stam's formulation of a BRDF which basically describes the effect of diffraction on a given surface assuming one knows the hightfield of this surface and will further extend this. Appart from Stam's approach, which models the heightfield as a probabilistic superposition of bumps and proceeds to derive an analytical expression for the BRDF, our BRDF representation takes the heightfield from explicit measurement. 
I.E. in our case, those heightfields are small patches of the microstructured surfaces (in nano-scale) taken by AFM of snake skin patches provided by our collaborators in Geneva..
So this approach is closer to real truth, since we use measured surfaces instead of statistical surface profile.

Therefore, this work can be considered as an extension of J. Stam's derivations for the case one is provided by a explicit height field on a quasiperiodic structure.

Real time performance is achieved with a representation of the formula as a power series over a variable related to the viewing and lighting directions. Values closely related to the coefficients in that power series are precomputed.

The contribution is that this approach is more broadly applicable than the previous work. Although the previously published formula theoretically has this much flexibility already, there is a novel contribution in demonstrating how such generality can be leveraged in practical implementation


\section{Previous work}
stam, hooke, see our paper, see stams paper, see own research.


Robert Hooke = observed connection between microscopic structures and colorisation
wave nature of light led to conclusion that the cause for the coloration lies in wave interference.

previous

In computer graphics literature, Stam was the first to develop reflection models based on wave optics called diffraction shaders, that can produce colorful diffraction effects. His approach is based on a far field approximation of the Kirchhof integral. He shows that for surfaces representeted as nanoscale heightfieds it is possible to derive their BRDF as the Fourier transformation of a function of the heightfield. Nevetheless, this formulation is not immediately useful for efficent rendering of measured complex nanostructures since this would require the on-the-fly evaluation of and and integration over Fourier transforms of the heightfield that depend on the light and viewing geometry. In his derivations, Stam models heightfields as probabilistic superpositions of bumps forming periodic like structures. This provides him an analytical identity for this class of heightfields. However, boplogocal nanostructures are way more complex and do not lend themself to this simplified statistical model.

follow ups


\section{Overview}
The reminder of this thesis is organized as the follows: due to the fact that this thesis has a rather advanced mathematical complexity the first part of chapter 2 introduces some important definitions which are required in order to be able to follow the derivations in the last third of chapter 2. Before starting the derivations, a brief summary of J. Stam's Paper about diffraction shaders is provided since this whole thesis is based on his BRDF representation. Our derivations itself are listed step-wiese, whereas there is a final representation provided by the end of chapter 2. Chapter 3 addresses the practical part of this thesis, the implementation of our diffraction model, explaining all precomputation steps and how rendering is preformed in our developed framework for this thesis. Chapter 4  gives some further insight about diffraction by explaining the topic about diffraction grating in depth. Furthermore, within this chapter we evaluates the qualitative validity of our BRDF models applied on different surface gratings by computing their reflectance and comparing this to the grating equation under similar conditions. Chapter 5 presents our rendered results, first the so called BRDF maps for all our gratings and shading approaches under various shading parameters and then the actual renderings on a snake mesh. Chapter 6 contains the conclusion of this thesis which starts by a review briefly discussing what has been achieved in this thesis and the drawbacks. There are also some words about my personal experience during this thesis.

\newpage{\pagestyle{empty} \cleardoublepage}
% 
\chapter{Theoretical Background}
\section{Basics in Modeling Light in Computer Graphics}

\subsection{Radiometry}
One purpose of Computer Graphics is to simulate the interaction of light on a surface and how a real-world observer, such as a human eye, will perceive this. These visual sensations of an eye are modeled relying on a virtual camera which captures the emitted light from the surface. The physical basis to measure such reflected light depicts radiometry which is about measuring the electromagnetic radiation transfered from a soruce to a receiver. 

Fundamentally, light is a form of energy propagation, consisting of a large collection of photons, whereat each photon can be considered as a quantum of light that has a position, direction of propagation and a wavelength $\lambda$. A photon travels at a certain speed $v = \frac{c}{n}$, that depends only the speed of light $c$ and the refractive index $n$ through which it progrates. Its frequency is defined by $f = \frac{v}{\lambda}$ and its carried amount of energy $q$, mearsured in the SI unit Joule, is given by $q = hf= \frac{hv}{\lambda n}$ where $h$ is the Plank's constant. The total energy of a large collection of photons is hence $Q = \sum_i q_i$.

\subsection{Spectral Energy}

It is important to understand that the human eye is not equally sensitive to all wavelength of the spectrum of light and therefore responds differently to specific wavelengths. Remember that our goal is to model the human visual perception. This is why we consider the energy distribution of a light spectrum rather than considering the total energry of a photon collection since then we could weight the distribution according the human visual system. So the question we want to answer is: How is the energy distributed across wavelengths of light?

The idea is to make an energy histrogram from a given photon collection. For this we have to order all photons by their associated wavelength, discretize wavelength spectrum, count all photons which then will fall in same wavelength-interval, and then, finally, normalize each interval by the total energy $Q$. This will give us a histogram which tells us the spectral energy $Q_{\lambda}$ for a given discrete $\lambda$ interval and thus models the so called spectral energy distribution $\footnote{Intensive quantities can be thought of as density functions that tell the density of an extensive quantity at an infinitesimal point.}$.

\subsection{Spectral Power}
Rendering an image in Computer Graphics corresponds to capturing the color sensation of an illuminated, target scene at a certain point in time. As previousely seen, each color is associated by a wavelength and is directly related to a certain amount of enegry. In order to determine the color of a to-be-rendered pixel of an image, we have to get a sense of how much light (in terms of energy) passes through the area which the pixel corresponds to. One possibility is to consider the flow of energy $\Phi = \frac{\Delta Q}{\Delta t}$ transferred through this area over a small period of time. This allows us to measure the energy flow through a pixel during a certain amount of time. 

In general, power is the estimated rate of energy production for light sources and corresponds to the flux. It is measured in the unit Watts, denoted by Q. Since power is a rate over time, it is well defined even when energy production is varying over time. As with Spectral Energy for rendering, we are really interested in the spectral power $\Phi_\lambda = \frac{\Phi}{\lambda}$, measured in Watts per nanometer.

\subsection{Spectral Irradiance}
Before we can tell how much light is reflected from a given point on a surface towards the viewing direction of an observer, we first have to know how much light arrives at this point. Since in general a point has no length, area or even volume associated, let us instead consider an infinitimal area $\Delta A$ around a such a point. Then, we can ask ourself how much light falls in such a small area. When further observing this process over a short period in time, this quantity is the spectral irradiance $E$ as illustrated in figure $\ref{fig:irradiance}$. Summarized, this quantity tells us how much spectral power is incident on a surface per unit area and mathematically is equal:

\begin{equation}
 E = \frac{\Phi_{\lambda}}{\Delta A}
\end{equation} 

\begin{figure}[H]
  \centering
  \includegraphics[scale=0.5]{background/irradiance.png}
  \caption{Irradiance is the summed up radiance over all directions}
  \label{fig:irradiance}
\end{figure}

\subsection{Spectral Radiance}
When rendering an image we have to determine the color of each pixel of the image. Although irradiance tells us how much light is arriving at a point as illustrated in figure $\ref{fig:irradiance}$, it tells us little about the direction that light comes from. This relates to how the human eye perceives the brightness of an illuminated objects when looking at it in a certain direction. 

\begin{figure}[H]
  \centering
  \subfigure[Radiances is the density of photons per area per solid angle]{
    \includegraphics[scale=0.6]{background/radiancehemisphere.png}
    \label{fig:radiance}
  }
~
  \subfigure[Solid angle is the area of a surface patch on a sphere with radius R which is spanned by a set of directions]{  
    \includegraphics[scale=0.42]{background/solidangle.png}
    \label{fig:solidangle}
  }
  
\label{fig:radianceBasics}
\end{figure}

This concepted is described by the radiometric quantity radiance. Basically, this is a measure of light energy passing through or is emitted off from a small area around a point on a surface towards a given direction during a short period in time. More formally this is the spectral power emerging from an arbitrary point (an infinitimal area around this point) and falls within a given solid angle (see figure$\footnote{Similar figute like used in computer graphics class 2012 in chapter colors}$ $\ref{fig:solidangle}$)in specific direction (usually towards the observer) as shown in figure $\ref{fig:radiance}$. Formally, this leads us to the following mathematical formalism: 

\begin{equation}
 L_{\lambda}(\omega) = \frac{d^2 \Phi_{\lambda}}{dA d\Omega} \approx \frac{\Phi_{\lambda}}{\Omega A}
\end{equation}

where $L$ is the observed spectral radiance in the unit energy per unit area per solid angle, which is $Wm^-2 sr^-1$ in direction $\omega$ which has an angle $\theta$ between the surface normal and $\omega$, $\Theta$ is the total flux or power emitted, $\theta$ is the angle between the surface normal and the specified direction, $A$ is the area of the surface and $\Omega$ is the solid angle in the unit steradian subtended by the observation or measurement.

It is useful to distinguish between radiance incident at a point on a surface and excitant from that point. Terms for these concepts sometimes used in the graphics literature are surface radiance $L_r$ for the radiance \textit{reflected} from a surface and field radiance $L_i$ for the radiance \textit{incident} at a surface.  

\subsection{BRDF}
In order to render the colorization of an observed object, a natural question in computer graphics is what portion of the reflected, incident light a viewer will receive, when he looks at an illuminated object. Therefore for any given surfaces which is illuminated from a certain direction $\omega_i$, we can ask ourself how much light is reflected off of any point on this surface towards a viewing direction $\omega_r$. This is where the Bidirectional Reflectance Distribution Function (short: BRDF) comes into play, which is a radiometric quantity telling us how much light is reflected at an opaque surface. Mathematically speaking, the BRDF is the ratio of the reflected radiance pointing to the direction $\omega_r$ to the incident irradiance comming from the inverse direction of $\omega_i$ as illustrated in figure $\ref{fig:brdfillustration}$. Hence the BRDF is a four dimensional function defined by four angles $\theta_i$, $\phi_i$, $\theta_r$ and $\phi_r$.

\begin{figure}[ht]
  \centering
  \includegraphics[scale=0.5]{background/mybrdfmodel.png}
  \caption[BRDF Model]{Illistration of the BRDF model, where $\omega_i$ is pointing to the light source and the existing direction is denoted by $\omega_r$. Both direction unit direction vectors defined w.r.t to a surface normal $\mathbf{n}$ for every point on the surface.}
  \label{fig:brdfillustration}  
\end{figure}

Which formally is for any given wavelength $\lambda$ equivalent to:

\begin{align}
  BRDF_{\lambda}(\omega_i, \omega_r)
  & = \frac{dL_r(\omega_r)}{dE_i(\omega_i)} \nonumber \\
  & = \frac{dL_r(\omega_r)}{L_i(\omega_i)cos(\theta_i)d\omega_i}
  \label{eq:defbrdf}
\end{align}

Where $L_{r}$ is the reflected spectral radiance, $E_i$ is the spectral irradiance and $\theta_{\text{i}}$ is the angle between $\omega_{\text{i}}$ and the surface normal $\mathbf n$. 

\subsection{Wavespectrum and Colors}
In order to see how crucial the role of human vision plays, let us consider the following definition of color by \textit{Wyszechkiu and Siles}$\footnote{mentioned in Computer Graphics Fundamentals Book from the year 2000}$ stating that \textit{Color is the aspect of visual perception by which an observer may distinguish differences between two structure-free fields of view of the same size and shape such as may be caused by differences in the spectral composition of the radiant energy concerned in the observation}. Therefore, similarly like the humans' perceived sensation of smell and taste, color vision is just another individual sense of perception giving us the ability to distinguish different frequency distribution of light experienced as color.

\begin{figure}[H]
  \centering
  \includegraphics[scale=0.35]{background/lightspec.png}
  \caption[visiblelightspectrum]{Frequency (top) and wavelenght (bottom) of colors of the visible light spectrum$\footnotemark$.}
  \label{fig:colorspectrum}
\end{figure}
\footnotetext{Similar figute like used in computer graphics class 2012 in chapter colors}

In general an eye consists of photoreceptor cells which are responsible for providing ability of color-perception. A schematic of an eye is illustratied in figure $\ref{fig:humaneye}$. Basically, there are two specialized types of photoreceptor cells, cone cells which are responsible for color vision and rod cells, which allow an eye to perceive different brigthness levels.

\begin{figure}[H]
  \centering
  \includegraphics[scale=0.35]{background/humaneye.png}
  \caption[humanayeschematic]{Schematic$\footnotemark$ of photoreceptor cells, cones and rods, in human eye }
  \label{fig:humaneye}
\end{figure}
\footnotetext{image of illustration has been taken from \href{http://en.wikipedia.org/wiki/Bidirectional_reflectance_distribution_function}{wikipedia}}

A human eye is made of three different types of cone cells, having their peak sensivity in sensing color at different wavelength ranges. More precisely, there are cone cells most sensitive to short wavelengths which are between $420 nm$ and $440 nm$, those which are most sensitive in the middle range between $530 nm$ and $550 nm$ and those which have their peak in the long range, from $560 nm$ to $580 nm$. In principle, any color sensation in human color perception as shown in figure $\ref{fig:colorspectrum}$ can therefore be described by just three parameters, corresponding to levels of stimulus of the three types of cone cells.  

\subsection{Colorspace}
In order to render accurately images of how a human observer sees its world, a mathematical model of the human color perception is required. Remember that color sensation is due to a visual stimulus processed by cone cells in an eye. A human eye contains three different types of cone cells. Therefore, one possible approach is to describe each kind of these cone cells as a function of wavelength, returning a certain intensivity. In the early 1920, from a series of experiments the so called CIE XYZ color space was derived, describing response of cone cells of an average human individual, the so called standard observer. Basically, a statistically sufficiently large number of probands were exposed to different target light colors expressed by their wavelength. The task of each proband was to reproduce these target colors by mixing three given primary colors, red-, green- and blue-light. The strength  of each primary color could be manually adjusted by setting their relative intensivity. Those adjustment weights have been measured, aggregated and averaged among all probands for each primary color. This model describes each color as a triple of three real valued numbers$\footnote{note that there are  negative color weights possible in the CIE XYZ colors space. This is why some human perceived color sensations could not be reconstructed using just an additive color model (adding three positively weighted primary values). Therefore, a probabant was also allowed to move one of the primary colors to the target color and instead was supposed to reproduce this new color mix using the two remaining primaries (subtractive model). The value of the selected, moved primary was then interpreted as beeing negative weighted in an additive color model.}$, the so called tristimulus values.

Pragmatically speaking, color spaces describes the range of colors a camera can see, a printer can print or a monitor can display. Thus, formally we can define it as a mapping a range of physically produced colors from mixed light to an objective description of color
sensations registered in the eye of an observer in terms of tristimulus values. 

\begin{figure}[H]
  \centering
  \includegraphics[scale=0.7]{background/somatchingfunctions.png}
  \caption{Plots of our color matching functions we used for rendering}
  \label{fig:matchingfunction}
\end{figure}

Interpolating all measured tristimuli values gives us three basis functions, the CIE color matching functions $\overline{x}(\lambda)$, $\overline{y}(\lambda)$, $\overline{z}(\lambda)$. In figure $\ref{fig:matchingfunction}$ are the numerical description of the chromatic response of the observer. They can be thought of as the spectral sensitivity curves of three linear light detectors yielding the CIE Tristimulus values X, Y and Z. 

The tristimulus values for a color with a spectral power distribution $I(\lambda)$, are given in terms of the standard observer by:

\begin{align}
    X= \int_{\Lambda} I(\lambda)\,\overline{x}(\lambda)\,d\lambda \nonumber \\
    Y= \int_{\Lambda} I(\lambda)\,\overline{y}(\lambda)\,d\lambda \nonumber \\
    Z= \int_{\Lambda} I(\lambda)\,\overline{z}(\lambda)\,d\lambda
\label{eq:tristimulusvalues}
\end{align}

Where $\lambda$, is the wavelength of the equivalent monochromatic light spectrum $\Lambda = [380nm, 780nm]$. Note taht it is not possible to build a display that corresponds to the CIE XYZ colorspace. For this reasons it is necessary to design other color spaces, which are physical realizable, offers efficient encoding, are perceptual uniform and have an intuitive color specification. There are simple conversions between XYZ color space, to other color space described as linear transformations.

\subsection{Spectral Rendering}
When rendering an image, most of the time we are using colors described in a certain RGB color space. However, a RGB colorspace results from a colorspace transformation of the tristimulus values, which themself are inherent to the human visual system. Therefore, many physically light phenomenon are poorly modeled when always relying on RGB colors for rendering. Using only RGB colors for rendering is alike we would assume that a given light source emits light of only one particular wavelength. But in reality this is barely the case. Spectral rendering is referring to use a certain wavelength spectrum, e.g. the human visible light spectrum, instead simply using the whole range of RGB values in order to render an illuminated scene. This captures the physical reality of specific light sources way more accurate. Keep in mind that, even when we make use of a spectral rendering approach, we have to convert the final spectra to RGB values, when we want to display an image on an actual display. 

\section{Wave Theory for Light and Diffraction}
\subsection{Basics in Wave Theory}
In order prepare the reader for physical relevant concepts used during later derivations and reasonings within this thesis, I am going to provide a quick introducation to the fundamental basics of wave theory and related concepts. In physics a wave describes a disturbance that travels from one location to another through a certain medium. The disturbance temporarly displaces the particles in the medium from their rest position which results in an energy transport along the medium during wave propagation. Usually, when talking about waves we are actually refering to a complex valued function which is a solution to the so called wave equation which is modeling how the wave disturbance proceeds in space during time. 

There are two types of waves, mechanical waves which deform their mediums during propagation like sound waves and electromagnetic waves consiting of periodic oscilations of an electromagnetic field such as light for example. Like simplified illustrated in figure $\ref{fig:wavebasics}$, there are several properties someone can use and apply in order to compare and distinguish different waves:

\begin{figure}[H]
  \centering
  \includegraphics[scale=0.65]{background/waveschematicimpr.png}
  \caption[sinewave]{Simplified, one dimensionaly real valued wave function$\footnotemark$, giving an idea about some important wave properties. We denote the crest of a wave as the hightest point relative to the equilibrium line (zero height along time axis) and similarly the trough as the lowest point.}
  \label{fig:wavebasics}
\end{figure}
\footnotetext{Image source: http://neutrino.ethz.ch/Vorlesung/FS2013/index.php/vorlesungsskript}

\begin{description}
  \item[Wavelength:] Is usually denoted by $\lambda$ and is a measure for the spatial distance from one point to another until the shape of a wave repeats
  \item[Amplitude:] Is denoted by $A$ and there are two possible interprations: First, it is a measure of the height from the equilibrium point to the heighest point of a crest on the wave or the lowest point of a trough. This mean the amplitude can be positive or negative. However, usually, someone is  just interested the absolute value of an amplitude, the magnitude of a wave. For light waves it is a relative measure of intensity or brightnes to other lught waves of the same wavelength. And secondly, it can be interpreted as a measure how much energy a wave carries wherate the greater the absoulte aplitute value, the bigger the amount of energy being carried.
  \item[Frequency:] Is a measure of the number of waves which are passing through a particular point in the propagation medium during a certain time and is denoted by $f$.
  \item[Phase:] Is denoted by $\phi$. Describes either the offset of initial position of a wave or the relative displacement between or among waves having the same frequency. Two waves two waves with same frequency are denoted by being in phase if they have the same phase. This means they line up everywhere. As a remark, we denote by $\omega$ the angular frequency which is equal $2\pi f$. 
\end{description}

A geometrical property of waves is their wavefront. This is either a surface or line along the path of wave propagation on which the disturbance at every point has the same phase. Three are basically three types of wavefronts: spherical-, cylindrical- and plane wavefront. If point in a isotropic medium is sending out waves in three dimensions, then the coresponding wavefronts are spheres, centered on the source point. Hence spherical wavefront is the result of a spherical wave, also denoted as a wavelet. Note that for electromagnetic waves, the phase is a poisition of a point in time on a wavefront cycle (motion of wave over a whole wavelength) whereat a complete cycle is defined as being equal 360 degrees.



\subsection{Wave Interference}
Next, after having seen that a wave is simply a traveling disturbance along a medium, having some special properties, someone could ask what happens when there are several waves travaling on the same medium. Especially, we are interested how these waves will interact with each other. In physics the term interference denotes the interaction of waves when they encounter each other at a point along their propagation medium. At each point where two waves superpose, their total displacement at these points is the sum of the displacements of each indiviudal wave at those points. Then, the resulting wave is having a greater or lower amplitude than each seperate wave and this we can interprete the interference as the addition operator for waves. Two extreme scenarios are illustrated in figure $\ref{fig:interferenceconcept}$. There are basically three variants of interferences which can occur, depending on how crest and troughs of the waves are matched up:

\begin{figure}[H]
  \centering
  \includegraphics[scale=0.65]{background/interferenceconcept.png}
  \caption[interference]{Interference scenarios$\footnotemark$ when two waves waves meet: On the left handside, there is constructive interference and on the right handside there is destructive interference illustrated.}
  \label{fig:interferenceconcept}
\end{figure}
\footnotetext{Image source: \texttt{http://en.wikipedia.org/wiki/Interference\textunderscore(wave\textunderscore propagation)} } 

\begin{itemize}
  \item Either a crest of a wave meets a crest of another wave or similarly a trough meets a trough of another wave. This scenario is denoted as constructive interference and occurs at any location along the medium where the two interfering waves have a displacement in the same direction. This is equivalent like saying that the phase difference between the waves is a multiple of $2\pi$. Then the resulting amplitude at that point is being much larger than the amplitude of an individual wave. For two waves with an equal amplitude interfering constructively, the resulting amplitude is twice as large as the amplitude of an individual wave.
  \item Either a crest of a wave meets a trough of another wave or vice versa. This scenario is denoted as destructive interference and occurs at any location along the medium where the two interfering waves have a displacement in the opposite direction. This is like saying that the phase difference between the waves is an odd multiple of $\pi$. Then the waves completely cancel each other out at any point they superimpose.
  \item If the phase difference between two waves is intermediate between the first two scenarios, then the magnitude of the displacement lies between the minimal and maximal values which we could get from constructive interference.
\end{itemize}
Keep in mind that when two or more waves interfere which each other, the resulting wave will have a different frequency. For a wave, having a different frequency also means having a different wavelength. Therefore, this directly implies that a light of a different color, than its source waves have, is emitted. 

\subsection{Wave Coherence}
When considering waves which are traveling on a shared medium along the same direction, we could examine how their phase difference is changing over time. Formulating the changement of their relative phase as a function of time will provides us a quantitative measure of the synchronism of two waves, the so called wave coherence. In order to better understand this concept, let us consider a perfectly mathematical sine wave and second wave which is a phase-shifted replica of the first one. A property of mathematical waves is that they keep their shape over an infinity amount of moved wavelengths. In our scenario, both waves are traveling along the same direction on the same medium, like exemplarily illustrated in figure $\ref{fig:coherencesinsignal}$.
\begin{figure}[H]
  \centering
  \includegraphics[scale=0.3]{background/coherencesinsignal.png}
  \caption{Two mathematical sine waves which are perfectly coherent which means that their phase difference is constant for every point in time.}
  \label{fig:coherencesinsignal}
\end{figure}

Taking the difference between the two sine waves from the previous figure yields always a constant number. Therefore, those two waves are said to be coherent and hence perfectly synchronous over time. Notice that this scenario is completly artificial since in nature there are no mathematical sine waves. Rather, the phase difference is then a fuction of time $p(t)$. The more coherent two waves are, the slower this function will change over time. 
In fact, two waves are said to be coherent if they are either of the same frequency, temporally in phase or have the same amplitude at every point in time.
Thus two waves are coherent if they are generated at the same time, having the same frequency, amplitude, and phase. Reversely, Waves are considered incoherent or also asynchronous if they have no stable phase difference. This means $p(t)$ is heavly varying over time. Coherence describes the effect of whether waves will tend to interfere with each other constructively or destructively at a certain point in time and space. Thus this is a property of waves that enables stationary interference. The more correlated two waves are, the higher their degree of coherence is. In physics coherence between waves is quantified by the corss-correlation function, which basically predicts the value of a second wave using the value of the first one. There are two basic coherence classifications:

\begin{itemize}
  \item Spatial coherence is dealing with the question of what is the range of distance between two points in space in the extend of a wave for which there is occuring a significant effect of interference when averaged over time. This is formally answered by considering the correlation between waves at different point in space. The range of distance is also denoted as the coherence area.
  \item Temporal coherence examines the ability of how well a wave will interfer with itself at different moments in time. Mathematically, this kind of coherence is computed by averaging the measured correlation between the value of the wave and the delayed version of itself at different pairs of time. The Coherence time denotes the time for which the propagating wave is coherent and we therefore can predict its phase using the correlation function. The distance a wave has traveled during the coherence time is denoted as the coherence length.
\end{itemize}


\subsection{Huygen's Principle}
Besides from a wave's phase and amplitude, also its propagation directly affects the interaction between different waves and how they could interfere with each other. This is why it makes sense to formulate a model which allow us to predict the position of a moving wavefront and how it moves in space. This is where Huygen's Principle comes into play. It states that any each point of a wavefront may be regarded as a point source that emits spherical wavelets in every direction. Within the same propagation medium, these wavelets traval at the same speed as their source wavefront. The position of the new wavefront results by superimposing all of these emitted wavelets. Geometrically, the surface that is tangential to the secondary waves can be used in order to determine the future position of the wavfront. Thesrefore, the new wavefront encloses all emitted wavelets. Figure $\ref{fig:huygensprinciple}$ visualizes Huygen's Prinicple for a wavfront reflected off from a plane surface.

\begin{figure}[H]
  \centering
  \includegraphics[scale=0.6]{background/huygensprinciple.png}
  \caption[Huygen's Principle]{A moving wavefront (blue) encounters an obstacle (a surface in brown colors) and produces a new wavefront (green) as a result of superposition among all secondary wavelets.}
  \label{fig:huygensprinciple}
\end{figure}

\subsection{Waves Diffraction}
Revisting Hugen's Principle we know that each point on a wavefront can be considered as a source of a spherical wavelet which propagates in every direction. But what exactly happens when a wave's propagation direction is occluded by an object? What will be the outcome when applying Huygen's Principle for that case? An example scenario for this case is shown in figure $\ref{fig:wavediffraction}$. 

\begin{figure}[H]
  \centering
  \includegraphics[scale=0.65]{background/diffractiontransmissive.png}
  \caption[Diffracted Wave]{Illustation$\footnotemark$ of a diffraction scenario in which a plane wavefront passes through a surface with a certain width and how the wave will be bent, also showing the intensity of the resulting wave.}
  \label{fig:wavediffraction}
\end{figure}
\footnotetext{Image source:\texttt{http://cronodon.com/images/Single\textunderscore slit\textunderscore diffraction\textunderscore 2b.jpg} } 

Whenever a propagating wavefront is partially occluded by an obstacle, the wave is not only moving in the direction along its propagation, but is also bent around the edges of the obstacle. In physics, this phenomenon is called diffraction. Waves are diffracted due to interference which occures among all wavelets when applying Huygen's Principle for the case when a wavfront hits an obstacle. Generally, the effect of diffraction is most expressed for waves whose wavelength is roughly similar in size to the dimension of the occluding object. Conversely, if the wavelength is hardly similar in size, then there is almost no wave diffraction perceivable at all. This relationship between the strength of wave diffraction and wavelength-obstacle-dimensions is conceptually illustrated in figure $\ref{fig:diffractionrelationshipdimension}$ when a wave is transmitted through a surface. A reflective example is provided in figure $\ref{fig:huygensprinciple}$.

\begin{figure}[H]
  \centering
  \subfigure[W $\ll$ $\lambda$]{
  \includegraphics[scale=0.7]{background/Aa2l.png}
    \label{fig:a1}
  }
~
  \subfigure[W $\approx$ $2 \lambda$]{
    \includegraphics[scale=0.7]{background/Aastl.png}
    \label{fig:a2}
  }
~
  \subfigure[W $\approx$ $6 \lambda$]{
    \includegraphics[scale=0.7]{background/Aa6l.png}
    \label{fig:a3}
  }
  \caption[diffractiondiemsnion]{Illustration$\footnotemark$ of how the effect of diffraction changes when a wave with wavelength $\lambda$ propagates through a slit of width equal $W$.}
  \label{fig:diffractionrelationshipdimension}
\end{figure}
\footnotetext{Image taken from:\texttt{http://neutrino.ethz.ch/Vorlesung/FS2013/index.php/vorlesungsskript}, chapter 9, figure 9.14 } 

In everday's life, we can see the direct outcome of the effect of wave diffraction in form of structural colors. There are examples from nature such as the irridecent colors on various snake skins as well as artificial examples such as the colorful patterns notable when having a close look at an illuminated compact disc. All in common having having a surface made of very regular nanostructure which is diffracting an incident light. Such a nanostructure which exhibits a certain degree of regularity is also denoted as diffraction grating. More about this in section $\ref{sec:diffractiongrating}$.

\section{Stam's BRDF formulation}
\label{sec:sumstam}
In his paper about Diffraction Shader, J. Stam derives a BRDF which is modeling the effect of diffraction for various analytical anisotropic reflexion models relying on the so called scalar wave theory of diffraction for which a wave is assumed to be a complex valued scalar. 
It's noteworthy, that Stam's BRDF formulation does not take into account the polarization of the light. Fortunately, light sources like sunlight and light bulbs are unpolarized. 

A further assumption in Stam's Paper is, the emanated waves from the source are stationary, which implies the wave is a superposition of independent monochromatic waves. This implies that each wave is associated to a definite wavelength lambda. However, sunlight once again fulfills this fact.

In our simulations we will always assume we have given a directional light source, i.e. sunlight. Hence, Stam's model can be used for our derivations.

For his derivations Stam uses the Kirchhoff integral (ADD REF TO WIKI), which is relating the reflected field to the incoming field. This equation is a formalization of Huygen’s well-known principle that states that if one knows the wavefront at a given moment, the wave at a later time can be deduced by considering each point on the first wave as the source of a new disturbance. Mathematically speaking, once the field  $\psi_1 =  e^{ik\mathbf{x} \cdot \mathbf{s}\mathbf{s}}$ on the surface is known, the field $\psi_2$ everywhere else away from the surface can be computed.
More precisely, we want to compute the wave $\psi_2$ equal to the reflection of an incoming planar monochromatic wave $\psi_1 = e^{ik \omega_i * x}$  traveling in the direction $\omega_i$ from a surface $S$ to the light source. Formally, this can be written as:

\begin{equation}
\psi_{2}(\omega_i, \omega_r) = \frac{i k e^{i K R}}{4 \pi R} (F(-\omega_i-\omega_r)-(-\omega_i+\omega_r)) \cdot I_{1}(\omega_i, \omega_r) 
\label{eq:kirchhoff}
\end{equation}

with

\begin{equation}
I_{1}(\omega_i, \omega_r) = \int_{S} \hat{\mathbf{n}} e^{ik(-\omega_i-\omega_{r}) \cdot \mathbf{s} d\mathbf{s}}
\label{eq:IBase}
\end{equation}

In applied optics, when dealing with scattered waves, one does use differential scattering cross-section rather than defining a BRDF which has the following identity: 

\begin{equation}
    \sigma^0 = 4 \pi \lim_{R \to \infty} R^2 \frac{\langle \left|\psi_2\right|^2\rangle}{\langle \left|\psi_1\right|^2\rangle}
\end{equation}

where R is the distance from the center of the patch to the receiving point $x_p$, $\hat{\mathbf{n}}$ is the normal of the surface at s and the vectors:

The relationship between the BRDF and the scattering cross section can be shown to be equal to 

\begin{equation}
 BRDF = \frac{1}{4\pi}\frac{1}{A}\frac{\sigma^0}{cos(\theta_i)cos(\theta_r)}
 \label{fig:crossscateringbrdfrelationship} 
\end{equation}

where $\theta_i$ and $\theta_r$ are the angles of incident and reflected directions on the surface with the surface normal $n$. See ~\ref{fig:geometricsetup}.

\begin{figure}[ht]
  \centering
  \includegraphics[scale=0.25]{brdfdiagram.png}
  \caption{$\omega_i$ points toward the light source, $\omega_r$ points toward the camera, $n$ is the surface normal}
  \label{fig:geometricsetup}  
\end{figure}

The components of vector resulting by the difference between these direction vectors:
In order to simplify the calculations involved in his vectorized integral equations, Stam considers the components of vector 
\begin{equation}
  (u,v,w) = -\omega_i - \omega_r 
\label{eq:uvw}
\end{equation}

explicitly and introduces the equation: 
\begin{equation}
  I(ku,kv) = \int_{S} \hat{\mathbf{n}} e^{ik(u,v,w) \cdot \mathbf{s} d\mathbf{s}} 
\label{eq:Istart}
\end{equation}

which is a first simplification of $\ref{eq:IBase}$. Note that the scalar $w$ is the third component of ~\ref{eq:uvw} and can be written as $w = -(cos(\theta_i)+cos(\theta_r))$ using spherical coordinates. The scalar $k=\frac{2\pi}{\lambda}$ represent the wavenumber.


During his derivations, Stam provides a analytical representation for the Kirchhoff integral assuming that each surface point $s(x,y)$ can be parameterized by $(x,y,h(x,y))$ where $h$ is the height at the position $(x,y)$ on the given $(x,y)$ surface plane. Using the tangent plane approximation for the parameterized surface and plugging it into $\ref{eq:Istart}$ he will end up with: 

\begin{equation}
    \mathbf{I}(ku, kv) = \int \int (-h_{x}(x,y), -h_{y}(x,y), 1) e^{ikwh(x,y)} e^{ik(ux + vy)} dx dy
\label{eq:I1}
\end{equation}

For further simplification Stam formulates auxillary function which depends on the provided height field: 
\begin{equation}
  p(x,y) = e^{iwkh(x,y)} 
\label{eq:px}
\end{equation}

which will allow him to further simplify his equation $\ref{eq:I1}$ to:

\begin{equation}
    \mathbf{I}(ku, kv) = \int \int \frac{1}{ikw}(-p_x, -p_y, ikwp) dx dy
\label{eq:I2}
\end{equation}

where he used that $(-h_{x}(x,y), -h_{y}(x,y), 1)e^{kwh(x,y)}$ is equal to $\frac{(-p_x, -p_y, ikwp)}{ikw}$ using the definition of the partial derivatives applied to the function $\ref{eq:px}$.

Let $P(x,y)$ denote the Fourier Transform (FT) of $p(x,y)$. Then, the differentiation with respect to x respectively to y in the Fourier domain is equivalent to a multiplication of the Fourier transform by $-iku$ or $-ikv$ respectively. This leads him to the following simplification for $\ref{eq:I1}$:

\begin{equation}
    \mathbf{I}(ku, kv) = \frac{1}{w}P(ku, kv) \cdot (u,v,w)
\label{eq:I3}
\end{equation}

Let us consider the term $g = (F(-\omega_i - \omega_r)-(-\omega_i + \omega_r))$, which is a scalar factor of $\ref{eq:kirchhoff}$. The dot product with $g$ and $(-\omega_i - \omega_r)$ is equal $2F(1 + \omega_i \cdot \omega_r)$. Putting this finding and the identity $\ref{eq:I3}$ into $\ref{eq:kirchhoff}$ he will end up with:

\begin{equation}
\psi_{2}(\omega_i, \omega_r) = \frac{i k e^{i K R}}{4 \pi R} \frac{2F(1 + \omega_i \cdot \omega_r)}{w} P(ku, kv)
\label{eq:kirchhoffFinding}
\end{equation}

By using the identity $\ref{fig:crossscateringbrdfrelationship}$, this will lead us to his main finding:
\begin{equation} 
  BRDF_{\lambda}(\omega_i, \omega_r) = \frac{k^2 F^2 G}{4\pi^2 A w^2} \langle \left|P(ku, kv)\right|^2\rangle
\label{eq:mainstam}
\end{equation}

where $G$ is the so called geometry term which is equal: 

\begin{equation}
  G =\frac{(1 + \omega_i \cdot \omega_r)^2}{cos(\theta_i)cos(\theta_r)}
\label{eq:geometricterm}
\end{equation}


\newpage{\pagestyle{empty} \cleardoublepage}
% % % 
\chapter{Derivations}

\section{Problem Statement and Challenges}
The goal of this thesis is to perform a physically accurate and interactive simulation of structural colors production like shown in figure $\ref{fig:problemstatementoutput}$, which we can see whenever a light source is diffracted on a natural grating. For this purpose we need the to be provided by the following input data as shown in figure $\ref{fig:problemstatement}$:
\begin{itemize}
  \item A mesh representing a snake surface$\footnote{Which is in our simulation an actual reconstruction of a real snake skin. These measurements are provided by the Laboratory of Artificial and Natural Evolition at Geneva.See their website:\texttt{www.lanevol.org}.}$ with associated texture coordinates as shown in figure $\ref{fig:strucgeom}$.
  \item A natural diffraction grating represented as a height field, its maximum height and its pixel-width-correspondence$\footnote{Since the nanostructure is stored as a grayscale image, we need a scale telling us what length and height one pixel cooresponds to in this provided image.}$.
  \item A vectorfield which describes how fingers on a provided surface of the nanostructrue are aligned as shown in figure $\ref{fig:patchvectorfield}$. 
\end{itemize}

\begin{figure}[H]
  \centering
  \subfigure[Structure Geometry]{
    \includegraphics[scale=0.40]{derivation/structuregeom.png}
    \label{fig:strucgeom}
  }
~
  \subfigure[Nanostructure Surface]{
    \includegraphics[scale=0.40]{derivation/nanostructuresurface.png}
    \label{fig:nanostruc}
  }
~
  \subfigure[Patch Orientation]{
    \includegraphics[scale=0.40]{derivation/vectorfieldalongcylinder.png}
    \label{fig:patchvectorfield}
  }
  \caption[Problem Statement]{Input for our simulation}
  \label{fig:problemstatement}
\end{figure}

We want to rely on the integral equation $\ref{eq:mainstam}$ derived by J. Stam in his paper $\cite{diffstam}$ about diffraction shaders. This equation formualtes a BRDF modeling the effect of diffraction under the assumption that a given grating can either be formulated as an analytical function or its structure is simple enough beeing modeled relying on statistical methods. These assumptions guarantee that $\ref{eq:mainstam}$ has an explicit solution. However, the complexisty of a biological nanostructures cannot sufficiently and accurately modeled simply using statistical methods. This is why interactive computation at high resolution becomes a hard task, since we cannot evaluate the given integral equation on the fly. Therefore, we have to adapt Stam's equation such that we are able to perform interactive rendering using explicitly provided height fields.

\begin{figure}[H]
  \centering
  \includegraphics[scale=0.5]{derivation/renderedstructuredcolors.png}
  \caption[Problem Statement: Output]{Output: Rendered Structural Colors}
  \label{fig:problemstatementoutput}
\end{figure}


\section{Approximate a FT by a DFT}
\subsection{Reproduce FT by DTFT}
In the previous section, we have found an identity for the reflected spectral radiance $L_{\lambda}(\omega_r)$ when using Stam's BRDF for a given input height field. However, the derived expression in equation $\ref{eq:nonrelativebrdffinding}$ requires to evaluate the Fourier Transform of our height field$\footnote{actually it requires the computation of the inverse Fourier Transform of a transformed version of the given heightfield, the function p(x,y) defined in equation \ref{eq:px}.}$ for every direction. In this section we explain how to approximate the FT by the DTFT and apply it to our previous derivations. Figure $\ref{fig:ftbydtft}$ graphically shows how to obtain the DTFT from the FT for a one dimensional signal$\footnote{For our case we are dealing with a two dimensional, spatial signal, the given height field. Nevertheless, without any constraints of generality, the explained approach applies to multi dimensional problems.}$ \\ \\

\begin{figure}[ht]
  \centering
  \includegraphics[scale=0.6]{derivation/ftbydtft.png}
  \caption[FT by DTFT]{Illustration of how to approximate the analytical Fourier Transform (FT) $\footnotemark$ of a given continuous signal by a Discrete Time Fourier Transform (DTFT). The DTFT applied on a bandlimited, discretized signal yields a continuous, periodic response in frequency space.}
  \label{fig:ftbydtft}  
\end{figure}
\footnotetext{Images of function plots taken from \texttt{http://en.wikipedia.org/wiki/Discrete\textunderscore Fourier\textunderscore transform} and are modified.} 

The first step is to uniformly discretize the given signal since computers are working finite, discrete arithmetic. We rely on the Nyquist–Shannon sampling theorem tells us how dense we have to sample a given signal $s(x)$ such that can be reconstructed its sampled version $\hat{s}[n]$$\footnote{n denotes the number of samples.}$. In particular, a sampled version according to the Nyquist–Shannon sampling theorem will have the same Fourier Transform as its origianl singal. The sampling theorem states that if $f_{max}$ denotes the highest frequency of $s(x)$, then, it has to be sampled by a rate of $f_s$ with $2f_{max} \leq f_s$ in order to be reconstructable. By convention $T = \frac{1}{f_s}$ represent the interval length between two samples. \\ \\

Next, we apply the Fourier Tranformation operator on the discretized singal $\hat{s}$ which gives us the following expression: 

\begin{align}
\mathcal{F}_{FT}\{\hat{s}\}(w)
& = \int_{\mathds{R}} \hat{s}[n] e^{-iwx} dx \nonumber\\
& = \int_{\mathds{R}} mask(x)s(x) e^{-iwx} dx \nonumber\\
& = T\sum_{x=-\infty}^{\infty} \hat{s}[x] e^{-iwx} \nonumber\\
& = T\mathcal{F}_{DTFT}\{s\}(w)
\label{eq:sampledsignalfttodtft}
\end{align} 
Equation $\ref{eq:sampledsignalfttodtft}$ tells us that if $\hat{s}$ is sufficiently sampled, then its DTFT corresponds to the FT of $s(x)$ . Notice that the resulting DTFT from the sampled signal has a height of $\frac{A}{T}$ where A is the height of the FT of $s$ and thus is a scaled version of the FT.

\subsection{Spatial Coherence and Windowing}
\label{sec:spatialcoherenceandwindowing}
Before we can derive a final expression in order to approximate a FT by a DFT, we first have to revisit the concept of coherence introduced in section $\ref{sec:wavecoherence}$ of chapter 2.Previousely we have seen that Stam's BRDf tells us what is the total contribution of all secondary sources which allows us to say what is the reflected spectral radiance at a certain point in space. This is related to stationary interference which itself depends on the coherence property of the emitted secondary wave sources. The ability for two points in space, $t_1$ and $t_2$, to interfere in the extend of a wave when being averages over time is the so called spatial coherence. The spatial distance between such two points over which there is significant intererence is limited by the quantity coherence area. For filtered sunlight on earth this is equal to 65$\mu m$ $\footnote{A proof for this number can be looked up in the book Optical Coherence and Quantum Optics$\cite{optcoherence}$ on page 153 and 154.}$.

\begin{figure}[H]
  \centering
  \includegraphics[scale=0.5]{derivation/windowinggaussian.png}
  \caption[Coherence Area using Gaussian Window]{A plane wave encoungers a surface. According to Huygens principle, secondary wavelets are emitted of from this surface. The resulting wave at a certain point in space (here indicated by a gray circle) depends on the inteference among all waves encountering at this position. The amount of significant interference is directly affected by the spatial coherence property of all the wavelets.}
  \label{fig:coherenceareagaussianwindow}  
\end{figure}

Figure $\ref{fig:coherenceareagaussianwindow}$ illustrates the concept of spatial coherence. A wavefront (blue line) encounters a surface. Due to Hugen's Principle, secondary wavelets are emitted off from the surface. The reflected radiance at a certain point in space, e.g. at a viewer's eye position (denoted by the gray circle), is a result of interference among all wavelets at that point. This interference is directly affected by the spatial coherence property of all the emitted wavelets. \\

In physics spatial coherence is predicted by the cross correlation between $t_1$ and $t_2$ and usually modeled by by a Gaussian Random Process. For any such Gaussian Processes we can use a spatial gaussian window $g(x)$ which is equal:

\begin{equation} 
  g(x) = \frac{1}{\sqrt{2\pi}\cdot\sigma}\cdot e^{-\frac{x^2}{2\sigma^2}} 
  \label{eq:gaussianwindowspacial}
\end{equation} 

We have chosen standard deviation $\sigma_s$ of the window such that it fulfills the equation $4 \sigma_s = 65\mu m$. This is equivalent like saying we want to predict about $99.99\%$$\footnote{Standard deviation values from confidence intervals table of normal distribution provided by Wolfram MatheWorld \texttt{http://mathworld.wolfram.com/StandardDeviation.html}.}$ of the resulting spatial coherence interference effects in our model by a cross correlation function. \\

By applying the Fourier Transformation to the spatial window we get the corresponding window in frequency space will look like:
\begin{equation} 
  G(f) = e^{-\frac{f^2}{2\sigma_f^2}}
  \label{eq:gaussianwindowfrequencyspace}
\end{equation} 

Notice that this frequency space window has a standard deviation $\sigma_f$ equal to $\frac{1}{2 \pi \sigma_s}$. Those two windows, the spatial- and the frequency space window, will be used in the next section in order to approximate the DTFT by the DFT by a windowing apporach.

\subsection{Reproduce DTFT by DFT}
\label{sec:gaussianwindow}

In this section we explain how and under what assumptions the DTFT of a discretized signal$\footnote{E.g. a sampled signal like already presented in figure $\ref{fig:ftbydtft}$}$ can be approximated by a DFT. The whole idea how to reproduce the DTFT by DFT is schematically illustrated in figure $\ref{fig:dtftbydft}$.

\begin{figure}[H]
  \centering
  \includegraphics[scale=0.4]{derivation/dtftbydft.png}
  \caption[DTFT by DFT]{Illustration of how to approximate the DTFT $\footnotemark$ by the DFT relying on the Convolution Theorem, using a gaussian window function.}
  \label{fig:dtftbydft}  
\end{figure}
\footnotetext{Images of function plots taken from \texttt{http://en.wikipedia.org/wiki/Discrete\textunderscore Fourier\textunderscore transform} and are modified. Note that the scales in the graphic are not appropriate.} 

Given a spatial, bandlimited and discretized one dimensional signal $\hat{s}$. Our goal is to approximate this spatial signal in a way such that when taking the DTFT of this approximated signal, it will yield almost the same like taking the DTFT of the original sampled $\hat{s}$. For this purpose we will use the previouse introduced concept of gaussian windows and the so called Convolution Theorem which is a fundamental property of all Fourier Transformations. \\

The Convolution Theorem states that the Fourier Transformation of a product of two functions, $f$ and $g$, is equal to convolving the Fourier Transformations of each individual function. Mathematically, this statement corresponds to equation $\ref{eq:convolutiontheorem}$:

\begin{equation} 
  \mathcal{F}\{f\cdot g\} = \mathcal{F}\{f\} * \mathcal{F}\{g\}
  \label{eq:convolutiontheorem}
\end{equation}

The principal issue is how to approximate our given signal $\hat{s}$. Therefore, let us consider another signal $\hat{s_N}$ which is the $N$ times replicated version of $\hat{s}$ (blue signal at center top in figure). \\

Remeber that in general, the signal repsonse at a certain point in space is the result of interference among all signals meeting at that position. In our scenario, the source of those signals are emitted secondary wavelets. The interference strength between these points is related to their spatial coherence. Windowing the signals by a gaussian window $g$ will capture a certain percentage of all interference effects. From the previous section $\ref{sec:spatialcoherenceandwindowing}$ we know that we can use gaussian window like in equation $\ref{eq:gaussianwindowspacial}$ in order to approximate such spatial signals interference effects. \\

Using this insight, we can approximate $\hat{s}$ by taking the product of $\hat{s_N}$ with a gaussian window $g$. This fact is illustrated in the first row of figure $\ref{fig:ftbydtft}$. So what will the DTFT of this approximation yield? We already know that the DTFT of $\hat{s}$ is a continuous, periodic signal, since $\hat{s}$ is bandlimited. Thus, taking the DTFT of this found approximation should give us approximatively the same continuous, periodic signal. \\

This is where the convolution theorem comes into play: Applying the DTFT to the product of $\hat{s_N}$ and $g$ is the same as convolving the DTFT of $\hat{s_N}$ by DTFT of $g$. From equation $\ref{eq:gaussianwindowfrequencyspace}$ we already know that the DTFT of $g$ is just another gaussian, denoted by $G$. On the other hand the DTFT of $\hat{s_N}$ yields a continuous, periodic signal. The higher the value of N, the sharper the signal gets (denoted by delta spiked) and the closer it converges toward to the DFT. This is why the DFT is the limit of a DTFT applied on periodic and discrete signals. Therefore, for a large number of $N$ we can replace the DTFT by the DFT operator when applied on $\hat{s_N}$. \\

Lastly, we see that the DTFT of $\hat{s}$ is approximitely the same like convolving a gaussian window by the DFT of $\hat{s_N}$. This also makes sense, since convolving a discrete, periodic signal (DFT of $\hat{s_N}$) by a continuous window function $G$ yiels a continuous, periodic function. \\

In practise, we cannot compute the DTFT $\ref{eq:dtft}$ numerically due to finite computer arithmetic and hence working with the DFT is our only option. Furthermore, there are numerically fast algorithmes in order to compute the DFT values of a function, the Fast Fourier Transformation (FFT). The DFT $\ref{eq:dft}$ of a discrete height field is equal to the DTFT of an infinitely periodic function consisting of replicas of the same height field. Not, let a spatial gaussian window $g$ having a standard deviation for which $4\sigma_s$ is equal $\mu m$. Then, from before, it followis:

\begin{equation}
\mathcal{F}_{dtft}\{\mathbf{s}\} \equiv \mathcal{F}_{dft} \{\mathbf{s}\} * G(\sigma_f)
\end{equation} 

Therefore we can deduce the following expression from this:

\begin{align}
\mathcal{F}_{dtft} \{\mathbf{t}\}(u,v)
& = \int_{-\infty}^{\infty} \int_{-\infty}^{\infty} {F}_{dft}\{\mathbf{t}\}(w_u,w_v) \phi(u-w_u, v-w_v) dw_u dw_v \nonumber \\
& = \int_{-\infty}^{\infty} \int_{-\infty}^{\infty} \sum_i \sum_j {F}_{dft} \{\mathbf{t}\}(w_u,w_v) \nonumber \\ 
& \quad \quad \delta(w_u-w_i, w_v-w_j)\phi(u-w_u, v-w_v) dw_u dw_v \nonumber \\
& = \sum_i \sum_j \int_{-\infty}^{\infty} \int_{-\infty}^{\infty}  {F}_{dft} \{\mathbf{t}\}(w_u,w_v) \nonumber \\
& \quad \quad \delta(w_u-w_i, w_v-w_j)\phi(u-w_u, v-w_v) dw_u dw_v \nonumber \\
& = \sum_i \sum_j {F}_{dft} \{\mathbf{t}\}(w_u,w_v) \phi(u-w_u, v-w_v)
\end{align}

where 

\begin{equation} \label{eq:gaussweight}
 \phi(x,y) = \pi e^{-\frac{x^2 + y^2}{2\sigma_{f}^2}}
\end{equation} 

\section{Adaption of Stam's BRDF discrete height fields}
\subsection{Rendering Equation}
As already discussed in the theoretical background chapter, colors are associated to radiance. Since we are starting with Stam's BRDF$\footnote{Remember that a BRDF is the portion of a incident light source reflected off a given surface towards a specified viewing direction.}$ formulation but want to perform a simulation rendering structural colors, we have to reformulate this BRDF equation such that we will end up with an identity of the reflected spectral radiance. This is where the rendering equation comes into play. Lets assume we have given an incoming light source with solid angle $\omega_i$ and $\theta_i$ is its angle of incidence, $\omega_r$ is the solid angle for the reflected light. Further let $\lambda$ denote the wavelength$\footnote{Notice that, to keep our terms simple, we have droped all $\lambda$ subscripts for spectral radiance quantites.}$ and $\Omega$ is the hemisphere of integration for the incoming light. Then, we are able to formulate a $BRDF_\lambda$ by using its definition $\ref{eq:defbrdf}$:  

\begin{alignat}{4}
& f_r(\omega_i, \omega_r) &&= \frac{dL_r(\omega_r)}{L_i(\omega_i)cos(\theta_i)d\omega_i} \nonumber \\
\Rightarrow{} & f_r(\omega_i, \omega_r) L_i(\omega_i)cos(\theta_i)d\omega_i &&= dL_r(\omega_r) \nonumber \\
\Rightarrow{} & \int_{\Omega}f_r(\omega_i, \omega_r) L_i(\omega_i)cos(\theta_i)d\omega_i &&= \int_{\Omega}dL_r(\omega_r) \nonumber\\
\Rightarrow{} & \int_{\Omega}f_r(\omega_i, \omega_r) L_i(\omega_i)cos(\theta_i)d\omega_i &&= L_r(\omega_r)
\label{eq:initialbrdf}
\end{alignat}

The last equation is the so called rendering equation $\label{sec:dirlighsourceassumption}$. We assume that our incident light is a directional, unpolarized light source like sunlight and therefore its radiance is given as 

\begin{equation}
 L_{\lambda}(\omega)=I(\lambda)\delta(\omega-\omega_i)
\label{eq:radiancedirlightsource}
\end{equation}

where $I(\lambda)$ is the intensity of the relative spectral power for the wavelength $\lambda$. By plugging the identity in equation $\ref{eq:radiancedirlightsource}$ into our current rendering equation $\ref{eq:initialbrdf}$, we will get:

\begin{align}
L_{\lambda}(w_r) 
& = \int_{\Omega} BRDF_{\lambda}(\omega_i, \omega_r) L_{\lambda}(\omega_i) cos(\theta_i) d\omega_i \nonumber \\
& = BRDF_{\lambda}(\omega_i, \omega_r) I(\lambda) cos(\theta_i)
\label{eq:deribrdfwithdirsource}
\end{align}

where $L_{\lambda}(\omega_i)$ is the incident radiance and $L_{\lambda}(\omega_r)$ is the radiance reflected by the given surface. Note that the integral in equation $\ref{eq:deribrdfwithdirsource}$ vanishes since $\delta(\omega-\omega_i)$ is only equal one if and only if $\omega = \omega_i$.

\subsection{Reflected Radiance of Stam's BRDF}
We are going to use Stam's main derivation $~\eqref{eq:mainstam}$ for the $BRDF(\omega_i, \omega_r)$ in $\ref{eq:deribrdfwithdirsource}$ by applying the fact that the wavenumber is equal $k=\frac{2\pi}{\lambda}$:

\begin{align}
BRDF(\omega_i, \omega_r) 
& = \frac{k^2 F^2 G}{4\pi^2 A w^2} \langle \left|P(ku, kv) \right|^2\rangle \nonumber\\
& = \frac{4 \pi^2 F^2 G}{4\pi^2 A \lambda^2 w^2} \langle \left|P(ku, kv)  \right|^2\rangle \nonumber\\
& = \frac{F^2 G}{A \lambda^2 w^2} \langle \left|P(\frac{2\pi u}{\lambda}, \frac{2\pi v}{\lambda})  \right|^2\rangle
\label{eq:minoradaptedstam}
\end{align}

Going back to equation $\ref{eq:deribrdfwithdirsource}$ and plugging equation $\ref{eq:minoradaptedstam}$ into it, using the definition of equation $\ref{eq:geometricterm}$ and the equation $\ref{eq:sphericalomega}$ for $\omega$ we will get the following:

\begin{align}
L_{\lambda}(\omega_r) 
& = \frac{F^2 (1 + \omega_i \cdot \omega_r)^2}{A \lambda^2 cos(\theta_i)cos(\theta_r)  \omega^2} \left \langle \left|P \left( \frac{2\pi u}{\lambda}, \frac{2\pi v}{\lambda}\right) \right|^2 \right \rangle cos(\theta_i) I(\lambda) \nonumber \\
& = I(\lambda) \frac{F^2 (1 + \omega_i \cdot \omega_r)^2}{\lambda^2 A \omega^2 cos(\theta_r)} \left \langle \left|P \left( \frac{2\pi u}{\lambda}, \frac{2\pi v}{\lambda}\right) \right|^2 \right \rangle
\label{eq:nonrelativebrdffinding}
\end{align}

Note that the Fresnel term $F$ is actually a function of $(w_i, w_r)$, but in order to keep the equations simple, we omitted its arguemts. 
So far we just plugged Stam's BRDF identity into the rendering equation and hence have not significantly deviated from his formulation. Keep in mind that $P$ deontes the Fourier transform of the provided height field which depends on the viewing and incidence light direction. Thus this Fourier Transform has to be recomputed for every direction which will slow down the whole computation quite a lot$\footnote{Even a fast variant of computation the Fourier Transform has a runtime complexitiy of O(N log N) where N is the number of sample.}$. One particular strategy to solve this issue is to approximate $P$ by the Discrete Fourier Transform (DFT)$\footnote{See appendix \ref{chap:appendixsignalprocessing} for further information about different kinds of fourier transformations.}$ and seperate its computation such that terms for many directions can be precomputed and then later retrieved by look ups. The approximation of $P$ happens in two steps: First we approximate the Fourier Transform by the Discrete Time Fourier Transform (DTFT) and then, afterwards, we approximate the DTFT by the DFT. For further about basics of signal processing and Fourier Transformations please consult the appendix $\ref{chap:appendixsignalprocessing}$. \\

Using the insight gained by equation $\ref{eq:sampledsignalfttodtft}$ allows us to further simplify equation $\ref{eq:nonrelativebrdffinding}$:

\begin{align}
L_{\lambda}(\omega_r) 
& = I(\lambda) \frac{F^2 (1 + \omega_i \cdot \omega_r)^2}{\lambda^2 A w^2 cos(\theta_r)} \left \langle \left|P \left( \frac{2\pi u}{\lambda}, \frac{2\pi v}{\lambda}\right) \right|^2 \right \rangle \nonumber \\
& = I(\lambda) \frac{F^2 (1 + \omega_i \cdot \omega_r)^2}{\lambda^2 A w^2 cos(\theta_r)} \left \langle \left|T^2 P_{dtft}\left( \frac{2\pi u}{\lambda}, \frac{2\pi v}{\lambda}\right) \right|^2 \right \rangle
\label{eq:nonrelativebrdffindingreproddtft}
\end{align}

Where $P_{dtft}$ is a substitude for $\mathcal{F}_{DTFT}\{s\}(w)$. Furthermore $T$ the sampling distance for the discretization of $p(x,y)$ assuming equal and uniform sampling in both dimensions $x$ and $y$.

\subsection{Relative Reflectance}
In this section we are going to explain how to scale our BRDF formulation such that all of its possible output values are mapped into the range $\left[0,1\right]$. Such a relative reflectance formulation will ease our life for later rendering purposes since usually color values are within the range $\left[0,1\right]$, too. Furthermore, this will allow us to properly blend the resulting illumination caused by diffraction with a texture map. \\

Let us examine what $L_\lambda(\omega_r)$ will be for a purely specular surface, for which $\omega_r = \omega_0 = \omega_i$ such that $\omega_0 = (0,0,1)$. For this specular reflection case, the correspinding radiance will be denoted as $L_\lambda^{spec}(\omega_0)$. When we know the expression for $L_\lambda^{spec}(\omega_0)$ we would be able to compute the relative reflected radiance for our problem $\ref{eq:nonrelativebrdffinding}$ by simply taking the fraction between $L_\lambda(\omega_r)$ and $L_\lambda^{spec}(\omega_0)$ which is denoted by: 

\begin{equation}
  \rho_\lambda(\omega_i,\omega_r) = \frac{L_\lambda(\omega_r)}{L_\lambda^{spec}(\omega_0)}
  \label{eq:rohrel}
\end{equation}

Notice that the third component $w$ from the vector in equation $\ref{eq:uvw}$ is squared eqaul $(cos(\theta_i)+cos(\theta_r))^2$$\footnote{Consult section $\ref{sec:componentw}$ in the appendix}$. But first, let us derive the following expression:

\begin{align}
L_\lambda^{spec}(\omega_0) 
& = I(\lambda) \frac{F(\omega_0, \omega_0)^2 (1+\colvec[0]{0}{1}\cdot\colvec[0]{0}{1})^2}{\lambda^2 A (cos(0)+cos(0))^2 cos(0)} \langle \left|T_0^2 P_{dtft}(0,0)  \right|^2\rangle \nonumber \\
& = I(\lambda) \frac{F(\omega_0, \omega_0)^2 (1+1)^2}{\lambda^2 A (1+1)^2 1}\left| T_0^2 N_{sample} \right|^2 \nonumber \\
& = I(\lambda) \frac{F(\omega_0, \omega_0)^2}{\lambda^2 A}\left| T_0^2 N_{sample} \right|^2 
\label{eq:lspec}
\end{align}

Where $N_{samples}$ is the number of samples of the DTFT $\ref{eq:dtft}$. Thus, we can plug our last derived expression $\ref{eq:lspec}$ into the definition for the relative reflectance radiance $\ref{eq:rohrel}$ in the direction $w_r$ and will get:

\begin{align}
\rho_\lambda(\omega_i,\omega_r)
& = \frac{L_\lambda(\omega_r)}{L_\lambda^{spec}(\omega_0)} \nonumber \\
& = \frac{I(\lambda) \frac{F(\omega_i, \omega_r)^2 (1 + \omega_i \cdot \omega_r)^2}{\lambda^2 A (cos(\theta_i)+cos(\theta_r))^2 cos(\theta_r)} \langle \left|T_0^2 P_{dtft}(\frac{2\pi u}{\lambda}, \frac{2\pi v}{\lambda}) \right|^2\rangle}{I(\lambda) \frac{F(\omega_0, \omega_0)^2}{\lambda^2 A}\left| T_0^2 N_{sample} \right|^2 } \nonumber \\
& = \frac{F^2(\omega_i,\omega_r)(1 + \omega_i \cdot \omega_r)^2}{F^2(\omega_0,\omega_0)(cos(\theta_i)+cos(\theta_r))^2 cos(\theta_r)} \langle \left|\frac{P_{dtft}(\frac{2\pi u}{\lambda}, \frac{2\pi v}{\lambda})}{N_{samples}}\right|^2\rangle
\label{eq:lspecrohrel}
\end{align}

For simplification and better readability, let us introduce the following expression, the so called gain-factor:

\begin{equation} 
    C(\omega_i,\omega_r) = \frac{F^2(\omega_i,\omega_r)(1 + \omega_i \cdot \omega_r)^2}{F^2(\omega_0,\omega_0)(cos(\theta_i)+cos(\theta_r))^2 cos(\theta_r) N_{samples}^2}
\label{eq:cfact}
\end{equation}

Using equation $\ref{eq:cfact}$, we will get the following expression for the relative reflectance radiance from equation $\ref{eq:lspecrohrel}$:

\begin{equation}
\rho_\lambda(\omega_i,\omega_r) =  C(\omega_i,\omega_r) \langle \left|P_{dtft}(\frac{2\pi u}{\lambda}, \frac{2\pi v}{\lambda})\right|^2\rangle
\label{eq:cpterm}
\end{equation}

Using the previous definition for the relative reflectance radiance equation $\ref{eq:rohrel}$:

\begin{equation}
 \rho_\lambda(\omega_i,\omega_r) = \frac{L_\lambda(\omega_r)}{L_\lambda^{spec}(\omega_0)} 
\end{equation}

Which we can rearrange to the expression: 

\begin{equation}
L_\lambda(\omega_r) = \rho_\lambda(\omega_i,\omega_r)L_\lambda^{spec}(\omega_0)
\label{eq:radianceomegarspec}
\end{equation}

Let us choose $L_\lambda^{spec}(w_0) = S(\lambda)$ such that is has the same profile as the relative spectral power distribution of CIE Standard Illuminant $D65$ discussed in $\ref{subsec:colortransformations}$. Furthermore, when integrating over $\lambda$ for a specular surface, we should get $CIE_{XYZ}$ values corresponding to the white point for $D65$. The corresponding tristimulus values using CIE colormatching functions $\ref{eq:tristimulusvalues}$ for the $CIE_{XYZ}$ values look like:

\begin{align}
X = \int_{\lambda}L_\lambda(\omega_r)\overline{x}(\lambda)d\lambda \nonumber \\
Y = \int_{\lambda}L_\lambda(\omega_r)\overline{y}(\lambda)d\lambda \nonumber \\
Z = \int_{\lambda}L_\lambda(\omega_r)\overline{z}(\lambda)d\lambda
\label{eq:tristimrad}
\end{align}

where $\overline{x}$, $\overline{y}$, $\overline{z}$ are the color matching functions. Combining our last finding from equation $\ref{eq:radianceomegarspec}$ for $L_\lambda(\omega_r)$ with the definition of the tristimulus values from equation $\ref{eq:tristimrad}$, allows us to 
derive a formula for computing the colors values using Stam's BRDF formula relying on the rendering equation $\ref{eq:initialbrdf}$. Without any loss of generality it suffices to derive an explicit expression for just one tristimulus term, for example Y, the luminance:

\begin{align}
Y 
& =\int_{\lambda}L_\lambda(\omega_r)\overline{y}(\lambda)d\lambda \nonumber \\
& =\int_{\lambda}\rho_\lambda(\omega_i,\omega_r)L_\lambda^{spec}(\omega_0) \overline{y}(\lambda)d\lambda \nonumber \\
& =\int_{\lambda}\rho_\lambda(\omega_i,\omega_r) S(\lambda) \overline{y}(\lambda)d\lambda \nonumber \\
& =\int_{\lambda} C(\omega_i,\omega_r) \langle \left|P_{dtft}(\frac{2\pi u}{\lambda}, \frac{2\pi v}{\lambda})\right|^2\rangle S(\lambda) \overline{y}(\lambda)d\lambda \nonumber \\
& = C(\omega_i,\omega_r) \int_{\lambda} \langle \left|P_{dtft}(\frac{2\pi u}{\lambda}, \frac{2\pi v}{\lambda})\right|^2\rangle S(\lambda) \overline{y}(\lambda)d\lambda \nonumber \\
& = C(\omega_i,\omega_r) \int_{\lambda} \langle \left|P_{dtft}(\frac{2\pi u}{\lambda}, \frac{2\pi v}{\lambda})\right|^2\rangle S_y(\lambda)d\lambda
\label{eq:structuralcolorsslow}
\end{align}

Where we used the definition $S_y(\lambda)\overline{y}(\lambda)$ in the last step.

\section{Optimization using Taylor Series}
\label{sec:taylorapproximation}
Our final goal is to render stractural colors resulting by the effect of wave diffraction. So far, we have derived an expression which can be used for rendering. Neverthesless, our current equation $\ref{eq:structuralcolorsslow}$ used for computing structural colors, cannot be used for interactive rendering, since $P_{dtft}$ had do be recomputed for every change in any direction$\footnote{viewing or incident light direction}$. \\

In this section, we will address this issue and deliver an approximation for the Fourier Transformation of Stam's auxiliary function $p(x,y)$. This will allow us to seperate $P_{dtft}$ in a way such that some computional expsensive terms can be precomputed. This derivation will rely on the definition of Taylor Series expansion like shown in equation $\ref{eq:deftaylor}$. Further, we will provide an error bound for our approximation approach for a given number of terms. Last, we will extend our current BRDF formula by the findings derived within this section. \\

Given $p(x,y)=e^{ikwh(x,y)}$ form Stam's Paper $\ref{sec:sumstam}$ where $h(x,y)$ is a given height field. Let $t$ be complex number, and let us consider the power series expansion of the exponential function:
 
\begin{equation}
  e^{t}=1+t+\frac{t^{2}}{2!}+\frac{t^{3}}{3!}+...=\sum_{n=0}^{\infty}\frac{t^{n}}{n!}
\end{equation}

Further, let us define 
\begin{align}
t 
= t(x,y) 
= ikwh(x,y)
\end{align}
 
where $i$ is the imaginary unit. For simplification, let us denote $h(x,y)$ as $h$. Then it follows by our previous stated identities: 

\begin{align}
 e^{t}
 &=1+(ikwh)+\frac{1}{2!}(ikwh)^{2}+\frac{1}{3!}(ikwh)^{3}+... \nonumber \\
 &=\sum_{n=0}^{\infty}\frac{(ikwh)^{n}}{n!}.
 \label{eq:expseriesexpr}
\end{align}

Please note that above's taylor series is convergent for any complex valued number $t$. Thus, plugging quation $\ref{eq:expseriesexpr}$ into the definiton of $p(x,y)$ will give us:

\begin{equation}
  p(x,y)=\sum_{n=0}^{\infty}\frac{(ikwh(x,y))^{n}}{n!}
  \label{eq:pxyseries}
\end{equation}

Next, let us now compute the Fourier Transformation $\mathcal{F}$ of equation $\ref{eq:pxyseries}$ form above:

\begin{align}
  \mathcal{F}\left\{ p\right\}
  & \equiv \mathcal{F}\left\{ \sum_{n=0}^{\infty}\frac{(ikwh)^{n}}{n!}.\right\} \nonumber \\
  & \equiv \sum_{n=0}^{\infty}\mathcal{F}\left\{ \frac{(ikwh)^{n}}{n!}\right\} \nonumber \\
  & \equiv \sum_{n=0}^{\infty}\frac{(ikw)^{n}}{n!}\mathcal{F}\left\{ h{}^{n}\right\}
  \label{eq:ftlinopid}
\end{align}

Where we have used the fact that the Fourier Transformation is a linear operator. Therefor, from equation $\ref{eq:ftlinopid}$ it follows: $P(\alpha,\beta)=\sum_{n=0}^{\infty}\frac{(ikw)^{n}}{n!}\mathcal{F}\left\{ h{}^{n}\right\} (\alpha,\beta)$ for which $\mathcal{F}\left\{ h{}^{n}\right\} (u,v)$.

Next we are going to look for an $N\mathbb{\in N}$ such that 
\begin{equation}
 \sum_{n=0}^{N}\frac{(ikwh)^{n}}{n!}\mathcal{F}\left\{ h{}^{n}\right\} (\alpha,\beta) \approx P(\alpha,\beta) 
\end{equation}

is a good approximation. But first, the following two facts would have to be proven$\footnote{Please have a look in section $\ref{chap:taylorseriesapproxappendix}$ in the appendix}$:

\begin{enumerate}
\item Show that there exist such an $N\mathbb{\in N}$ s.t the approximation
holds true.
\item Find a value for $B$ s.t. this approximation is below a certain error
bound, for example machine precision $\epsilon$. 
\end{enumerate}


Using now our approximation for $P_{dtft} = \mathcal{F}^{-1}\{p\}(u,v)$ for the tristimulus value Y, we will get:

\begin{align}
Y 
& = C(w_i,w_r) \int_{\lambda} \langle \left|P_{dtft}(\frac{2\pi u}{\lambda}, \frac{2\pi v}{\lambda})\right|^2\rangle S_y(\lambda)d\lambda \nonumber \\
& = C(w_i,w_r) \int_{\lambda} \left| \sum_{n=0}^N \frac{(wk)^n}{n!} \mathcal{F}^{-1}\{i^n h^n\}(\frac{2\pi u}{\lambda}, \frac{2\pi v}{\lambda})\right|^2 S_y(\lambda)d\lambda
\label{eq:xcolexpression}
\end{align}

\section{Spectral Rendering}
As the last step of our series of derivations, we plug all our findings together to one big equation in order to compute the color for each pixel on our mesh in the $CIE_{XYZ}$ colorspace. For any given heigh-field $h(x,y)$ representing a small patch of a nano structure of a surface and the direction vectors $w_i$ and $w_r$ from figure $\ref{fig:geometricsetup}$ the resulting color caused by the effect of diffraction can be computed like: Let 

\begin{equation}
P_{\lambda}(u,v) = {F}_{dft}\{i^n h^n\}(\frac{2\pi u}{\lambda},\frac{2\pi v}{\lambda})
\end{equation}

Then our final expression using our previous derivations will look like:

\begin{equation}
\begin{split}
\colvec[X]{Y}{Z}& = C(\omega_i,\omega_r) \int_{\lambda} \sum_{n=0}^N  \frac{(wk)^n}{n!} \sum_{(r,s) \in \mathcal{N}_1(u,v)} \left| P_{\lambda}(u-w_r,v-w_s) \right|^2 \\
& \quad \quad  \phi(u-w_r, v-w_s) \colvec[S_x(\lambda)]{S_y(\lambda)}{S_z(\lambda)}d\lambda
\end{split}
\label{eq:finalexpression}
\end{equation}

where $\phi(x,y) = \pi e^{-\frac{x^2 + y^2}{2\sigma_{f}^2}}$ is the Gaussian window $\ref{sec:gaussianwindow}$.

\section{Alternative Approach}
\subsection{PQ factors}
\label{sec:pq}
In this section we are presenting an alternative approach to the previous Gaussian window approach 
$\ref{sec:gaussianwindow}$ 
in order to solve the issue working with $DTFT$ instead the $DFT$. We assume, that a given surface $S$ is covered by a number of replicas of a provided representative surface patch $f$. In a simplified, one dimensional scenario, mathematically speaking, $f$ is assumed to be a periodic function, i.e. $\forall x \in \mathds{R} : f(x) = f(x+nT)$, where $T$ is its period and $n \in \mathds{N}_{0}$. Thus, the surfaces can be written formally as:

\begin{equation}
  S(x) = \sum_{n=0}^N f(x+nT)
\label{eq:replicatedpatchsurface}
\end{equation}

What we are looking for is an identity for the inverse Fourier transform of our surface $S$, required in order to simplify the $(X,Y,Z)$ colors from $\ref{eq:xcolexpression}$:

\begin{align}
\mathcal{F}^{-1}\{S\}(w)
& =\int f(x) e^{iwx}dx \nonumber \\
& =\int_{-\infty}^{\infty} \sum_{n=0}^{N} f(x+nT) e^{iwx}dx \nonumber \\
& =\sum_{n=0}^{N} \int_{-\infty}^{\infty} f(x+nT) e^{iwx}dx
\label{eq:pqsinit}
\end{align}

Next, apply the following substitution $x+nT = y$ which will lead us to:

\begin{gather}
x=y-nT \nonumber \\
dx=dy
\label{eq:substitude1dpq}
\end{gather} 

Plugging this substitution back into equation $\ref{eq:pqsinit}$ we will get: 

\begin{align}
\mathcal{F}^{-1}\{S\}(w)
& =\sum_{n=0}^{N} \int_{-\infty}^{\infty} f(x+nT) e^{iwx}dx \nonumber \\
& =\sum_{n=0}^{N} \int_{-\infty}^{\infty} f(y) e^{iw(y-nT)}dy \nonumber \\
& =\sum_{n=0}^{N} e^{-iwnT} \int_{-\infty}^{\infty} f(y) e^{iwy}dy \nonumber \\
& =\sum_{n=0}^{N} e^{-iwnT} \mathcal{F}^{-1}\{f\}(w) \nonumber \\
& =\mathcal{F}^{-1}\{f\}(w) \sum_{n=0}^{N} e^{-iwnT}
\label{eq:pqsub}  
\end{align}

We used the fact that the exponential term $e^{-iwnT}$ is a constant factor when integrating along $dy$ and the identity for the inverse Fourier transform of the function $f$. Next, let us examine the series $\sum_{n=0}^N e^{-iwnT}$ closer:

\begin{align}
\sum_{n=0}^N e^{-uwnT}
& =\sum_{n=0}^N (e^{-uwT})^n \nonumber \\
& =\frac{1-e^{iwT(N+1)}}{1-e^{-iwT}}
\label{eq:pqgeometricseries}
\end{align}

We recognize the geometric series identity for the left-hand-side of equation $\ref{eq:pqgeometricseries}$. Since our series is bounded, we can simplify the right-hand-side of equation $\ref{eq:pqgeometricseries}$.

Note that $e^{-ix}$ is a complex number. Every complex number can be written in its polar form, i.e. 

\begin{equation}
e^{-ix} = cos(x) + i sin(x) 
\label{eq:polarform}
\end{equation}

Using the following trigonometric identities
\begin{gather}
cos(-x) = cos(x) \nonumber \\
sin(-x) = -sin(x)
\end{gather}

combined with $\ref{eq:polarform}$ we can simplify the series $\ref{eq:pqgeometricseries}$ even further to:

\begin{align}
\frac{1-e^{iwT(N+1)}}{1-e^{-iwT}}
& =\frac{1-cos(wT(N+1)) + i sin(wT(N+1)) }{1-cos(wT) + i sin(wT)}
\label{eq:pq1minusexp}
\end{align}

Equation $\ref{eq:pq1minusexp}$ is still a complex number, denoted as $(p+iq)$. Generally, every complex number can be written as a fraction of two complex numbers. This implies that the complex number $(p+iq)$ can be written as $(p+iq) = \frac{(a+ib)}{(c+id)}$ for any $(a+ib), (c+id) \neq 0$. Let us use the following substitutions: 

\begin{align}
a& := 1 - cos(wT(N+1))&
b& =sin(wT(N+1)) \nonumber \\
c& =1-cos(wT)&
d& =sin(wT)
\label{eq:pqabcdsubstitudes}
\end{align}

Hence, using $\ref{eq:pqabcdsubstitudes}$, it follows 

\begin{equation}
  \frac{1-e^{iwT(N+1)}}{1-e^{-iwT}} = \frac{(a+ib)}{(c+id)}
\end{equation}

By rearranging the terms, it follows $(a+ib) = (c+id)(p+iq)$ and by multiplying its right hand-side out we get the following system of equations:

\begin{align}
(cp-dq)& =a \nonumber \\
(dp + cq)& =b
\label{eq:cdadcn}
\end{align}

After multiplying the first equation of $\ref{eq:cdadcn}$ by $c$ and the second by $d$ and then adding them together, we get using the law of distributivity new identities for $p$ and $q$:

\begin{align}
p& =\frac{(ac+bd)}{c^2 + d^2} \nonumber \\
q& =\frac{(bc+ad)}{c^2 + d^2}
\label{eq:pq1}
\end{align}

Using some trigonometric identities and putting our substitution from $\ref{eq:pqabcdsubstitudes}$ for $a$, $b$, $c$, $d$ back into the current representation $\ref{eq:pq1}$ of $p$ and $q$ we will get:

\begin{align}
p& =\frac{1}{2}+\frac{1}{2}\left(\frac{cos(wTN)-cos(wT(N+1))}{1-cos(wT)}\right) \nonumber \\
q& =\frac{sin(wT(N+1))-sin(wTN)-sin(wT)}{2(1-cos(wT))}
\end{align}

Since we have seen, that $\sum_{n=0}^N e^{-uwnT}$ is a complex number and can be written as $(p+iq)$, we now know an explicit expression for $p$ and $q$. Therefore, the one dimensional inverse Fourier transform of $S$ is equal:

\begin{align}
\mathcal{F}^{-1}\{S\}(w)
& =\mathcal{F}^{-1}\{f\}(w) \sum_{n=0}^{N} e^{-iwnT} \nonumber \\
& = (p+iq) \mathcal{F}^{-1}\{f\}(w)  
\label{eq:mainfinding1d}
\end{align}


Now lets consider our actual problem description. Given a patch of a nanoscaled sureface snake shed represented as a two dimensional heightfield $h(x,y)$. We once again assume that this provided patch is representing the whole surface $S$ of our geometry by some number of replicas of itself. Therefore, $S(x,y) = \sum_{n=0}^{N} h(x+nT_1, y+mT_2)$, assuming the given height field has the dimensions $T_1$ by $T_2$. In order to derive an identity for the two dimensional inverse Fourier transformation of $S$ we can similarly proceed like we did to derive equation $\ref{eq:mainfinding1d}$.

\begin{align}
\mathcal{F}^{-1}\{S\}(w_1, w_2)
& = \int_{-\infty}^{\infty}\int_{-\infty}^{\infty} \sum_{n_2=0}^{N_1} \sum_{n_2=0}^{N_2} h(x_1 + n_1 T_1, x_2 + n_2 T_2) e^{iw(x_1 + x_2)}dx_1 dx_2 \nonumber \\
& = \int_{-\infty}^{\infty}\int_{-\infty}^{\infty} \sum_{n_2=0}^{N_1} \sum_{n_2=0}^{N_2} h(y_1, y_2) e^{iw((y_1 - n_1 T_1) + (y_2 + n_2 T_2))}dx_1 dx_2 \nonumber \\
& =\sum_{n_2=0}^{N_1} \sum_{n_2=0}^{N_2} \int_{-\infty}^{\infty}\int_{-\infty}^{\infty} h(y_1, y_2) e^{iw(y_1 + y_2)} e^{-iw(n_1 T_1 + n_2 T_2)}dy_1 dy_2 \nonumber \\
& =\sum_{n_2=0}^{N_1} \sum_{n_2=0}^{N_2} e^{-iw(n_1 T_1 + n_2 T_2)} \int_{-\infty}^{\infty}\int_{-\infty}^{\infty} Box(y_1, y_2) e^{iw(y_1 + y_2)} dy_1 dy_2 \nonumber \\
& =\left(\sum_{n_2=0}^{N_1} \sum_{n_2=0}^{N_2} e^{-iw(n_1 T_1 + n_2 T_2)}\right) \mathcal{F}^{-1}\{h\}(w_1,w_2) \nonumber \\
& =\left(\sum_{n_2=0}^{N_1} e^{-iw n_1 T_1}\right) \left(\sum_{n_2=0}^{N_2} e^{-iw n_2 T_2}\right) \mathcal{F}^{-1}\{h\}(w_1,w_2) \nonumber \\
& =(p_1 + i q_1)(p_2 + i q_2) \mathcal{F}^{-1}\{h\}(w_1,w_2) \nonumber \\
& =((p_1 p_2 - q_1 q_2) + i(p_1 p_2 + q_1 q_2)) \mathcal{F}^{-1}\{h\}(w_1,w_2) \nonumber \\
& =(p + iq) \mathcal{F}_{DTFT}^{-1}\{h\}(w_1,w_2)
\label{eq:pqmainfinding}
\end{align}

Where we have defined 

\begin{align}
p := (p_1 p_2 - q_1 q_2) \nonumber \\ 
q := (p_1 p_2 + q_1 q_2)
\label{eq:pqsubst2d}
\end{align}

For the identity of equation $\ref{eq:pqmainfinding}$ we made use of Green's integration rule which allowed us to split the double integral to the product of two single integrations. Also, we used the definition of the 2-dimensional inverse Fourier transform of the height field function. We applied a similar substitution like we did in $\ref{eq:substitude1dpq}$, but this time twice, once for $x_1$ and once for $x_2$ separately. The last step in equation $\ref{eq:pqmainfinding}$, substituting with $p$ and $q$ in equation $\ref{eq:pqsubst2d}$ will be useful later in the implementation. The insight should be, that the product of two complex numbers is again a complex number. We will have to compute the absolute value of $\mathcal{F}^{-1}\{S\}(w_1,w_2)$ which will then be equal $(p^2 + q^2)^{\frac{1}{2}}\left|\mathcal{F}^{-1}\{h\}(w_1,w_2)\right|$


\subsection{Interpolation}
\label{sec:sincinterpolation}

In $\ref{sec:pq}$ we have derived an alternative approach when we are working with a periodic signal instead using the gaussian window approach from $\sec{sec:gaussianwindow}$. Its main finding $\ref{eq:pqmainfinding}$ that we can just integrate over one of its period instead iterating over the whole domain. Nevertheless, this main finding is using the inverse DTFT. Since we are using 

We are interested in recovering an original analog signal $x(t)$ from its samples $x[t] = $ 

Therfore, for a given sequence of real numbers $x[n]$, representing a digital signal, its correspond continuous function is: 

\begin{equation}
  x(t) = \sum_{n=-\infty}^{\infty} x[n] sinc\left(\frac{t-nT}{T}\right)
\end{equation}

which has the Fourier transformation $X(f)$ whose non-zero values are confined to the region $|f| \leq \frac{1}{2T} = B$.
When $x[n]$ represents time samples at interval $T$ of a continuous function, then the quantity $f_s = \frac{1}{T}$ is known as its sample rate and $\frac{f_s}{2}$ denotes the Nyquist frequency. The sampling Theorem states that when a function has a Bandlimit B less than the Nyquist frequency, then $x(t)$ is a perfect reconstruction of the original function. 

\begin{figure}[ht]
  \centering
  \includegraphics[scale=0.8]{background/sincreconstructed.png}
  \caption{Comparission between a given random one dimensional input signal $s(t)$ and its sinc interpolation $\hat{s}(t)$. Notice that for the interpolation there were $N=100$ samples from the original signal provided.}
  \label{fig:plotsincinterpolation}  
\end{figure}

\newpage{\pagestyle{empty} \cleardoublepage}
% 
\chapter{Implementation}
In computer graphics, we generally synthesize 2d images from a given 3d scene description$\footnote{A usual computer graphics scene consist of a viewer's eye, modeled by a virtuel camera, light sources and geometries placed in the world, having some material properties assigned to.}$. This process is denoted as rendering. A usual computer graphics scene consist of a viewer's eye, modeled by a virtuel camera, light sources and geometries placed in the world$\footnote{With the term world we are refering to a global coordinate system which is used in order to place all objects.}$, having some material properties$\footnote{Example material properties are: textures, surface colors, reflectance coefficients, refractive indices and so on.}$ assigned to. In our implementation, scene geometries are modeled by triangular meshes for which each triangle is represented by a triple of vertices. Each vertex has a position, a surface normal and a tangent vector associated with. \\

The process of rendering basically involves a mapping of 3d sceene objects to a 2d image plane and the computation of each image pixel's color according to the provided lighting, viewing and material information of the given scene. These pixel colors are computed in several statges in so called shader programs, directly running on the Graphic Processing Unit (GPU) hardware device. In order to interact with a GPU, for our implementations, we rely on the programing interface of OpenGL$\footnote{Official website:\texttt{http://www.opengl.org/}}$, a cross-language, multiplattform API. In OpenGL, there are two fundamental shading pipeline stages, the vertex- and the fragment shading stage, each applied sequentially. Vertex shaders apply all transformations to the mesh vertices and pass this data to the fragment shaders. Fragment shaders receive linearly interpolated vertex data of a particular triangle. They are responsible to compute the color of his triangle. \\

In this chapter we explain in detail a technique for rendering structural colors due to diffraction effects on natural graings, based on the model we have derived in the previouse chapter $\ref{chap:derivations}$, summarized in section $\ref{sec:spectralrendering}$. For this purpose we implemented a reference framework which is based on a class project of the lecture \emph{Computer Graphics} held by Mr. M. Zwicker which I attended in autum 2012$\footnote{The code of underlying reference framework is written in Java and uses JOGL and GLSL$\footnotemark$ in order to comunicate with the GPU and can be found at \texttt{https://ilias.unibe.ch/}}$. \\
$\footnotetext{JOGL is a Java binding for OpenGL (official website \texttt{http://jogamp.org/jogl/www/}) and GLSL is OpenGL's high-level shading language. Further information can be found on wikipedia: \texttt{http://de.wikipedia.org/wiki/OpenGL\textunderscore Shading\textunderscore Language}}$

For performing the rendering process, our implementation expects being provided by the following input data$\footnote{All data is provided by the Laboratory of Artificial and Natural Evolition in Geneva. See their website:\texttt{www.lanevol.org}}$:
\begin{itemize}
  \item the structure of snake skin of different species$\footnote{We are using height field data for Elaphe and Xenopeltis snakes individuals like shown in figure $\ref{fig:snakespecies}$}$ represented as discrete valued height fields acquired using AFM and stored as grayscale images.
  \item real measured snake geometry represented as a triangle mesh.
\end{itemize}

The first processing stage of our implementation is to compute the Fourier Terms of the provided height fields like described in section $\ref{sec:taylorapproximation}$. For this preprocessing purpose we use Matlab relying on its internal, numerically fast, libraries for computing Fourier Transformations$\footnote{Actually we use Matlab's inverse 2d Fast Fourier Transformation (FFT) implementation applied on different powers of quation $\ref{eq:px}$. Further information can be read up in section $\ref{sec:precompmatlabfourierimages}$}$. The next stage is to read these precomputed Fourier Terms into our Java renderer. This program also builds our manually defined rendering scene. The last processing stage of our implementation is rendering of the iridescent colorpatterns due to light diffracted on snake skins. We implemented our diffraction model from chapter $\ref{chap:derivations}$ as OpenGL shaders. Notice that all the necessary computations in order to simulate the effect of diffraction are performed within a fragment shader. This implies that we are modeling pixelwise the effect of diffraction and hence the overall rendering quality and runtime complexity depends on rendering window's resolution. \\

In the following sections of this chapter we are going to explain all render processing stages in detail. First, we discuss, how our precomputation process, using Matlab, actually works. Then, we introduce our Java Framework. It is followed by the main section of this chapter, the explanation how our OpenGL shaders are implemented. The last section discusses an optimization of our fragment shader such that it will have interactive runtime.

\section{Precomputations in Matlab}
\label{sec:precompmatlabfourierimages}
Our first task is to precompute the two dimensional discrete Fourier Fransformations for a given input height field, representing a natural grating. For that purpose we have written a small Matlab $\footnote{Matlab is a interpreted scripting language which offers a huge collection of mathematical and numerically fast and stable algorithms.}$ script conceptialized in algorithm $\ref{alg:matlabprecomp}$. Our Matlab script reads a given image, which is representing a nano-scaled height field, and computes its two dimensional DFT (2dDFT) by using Matlab's internal Fast Fourier Transformation (FFT) function, denoted by $ifft2$$\footnote{Remember, even we are talking about fourier transformations, in our actual computation, we have to compute the inverse fourier transformation. See paragraph $\ref{sec:electricalengeneeringftconvention}$ for further information. Furthermore our height fields are two dimensional and thus we have to compute a 2d inverse fourier transformation.}$. Note that we only require one color channel of the input image, since the input image is representing an height field, encoded by just one color. Keep in mind that taking the Fourier transformation of an arbitrary function will result in a complex valued output which implies that we will get a complex value for frequency pairs of our input image. Therefore, for each input image we get as many output images, representing the 2dDFT, as the minimal number of taylor terms required for a well-enough approximation. In order to store our output images, we have to use two color channels instead of just one like it was for the given input image. Some example visualizations for the Fourier Tranformation are shown in figure $\ref{fig:matlabBlazeFourierImages}$. We store these intermediate results as binary files to offer floating point precision for the run-time computations to ensure higher precision. \\

In our script every discrete frequency is normalized by its corresponding DFT extrema$\footnote{We are talking about the i2dFFT of our height fields to the power of n. This is an N by N matrix (assuming the discrete height field was an N by N image), for which each component is a complex number. Hence, there is a a complex extrema as well as a imaginary extrema.}$ in the range $\left[0,1\right]$ and the range extrema are stored seperately for each DFT term. The normalization is computed the following way: 

\begin{align}
  f:\left[x_{min},x_{max}\right]\to \left[0,1\right] \nonumber\\
  x \mapsto f(x) = \frac{x-x_{min}}{x_{max}-x_{min}}
\label{eq:dfttermnormalization}
\end{align}

Where $x_{min}$ and $x_{max}$ denote the extreme values of a DFT term. Later, during the shading process of our implementation, we have to apply the inverse mapping. This is non-linear interpolation which is required in order to rescaled all frequency values in the DFT terms. 

\begin{algorithm}[H]
\caption{Precomputation: Pseudo code to generate Fourier terms}
\textbf{INPUT} \ $heightfieldImg, \ maxH, \ dH, \ termCnt$ \\
\textbf{OUTPUT} \ $DFT \ terms \ stored \ in \ Files$
\begin{lstlisting}
% maxH:    A floating-point number specifying 
%          the value of maximum height of the 
%          height-field in MICRONS, where the 
%          minimum-height is zero. 
%         
% dH:      A floating-point number specifying 
%          the resolution (pixel-size) of the 
%          'discrete' height-field in MICRONS. 
%          It must be less than 0.1 MICRONS 
%          to ensure proper response for 
%          visible-range of light spectrum.
%
% termCnt: An integer specifying the number of 
%          Taylor series terms to use.

function ComputeFFTImages(heightfieldImg, maxH, dh, termCnt)
dH = dh*1E-6;
% load patch into heightfieldImg
patchImg = heightfieldImg.*maxH;
% rotate patchImg by 90 degrees
for t = 0 : termCnt
  patchFFT = power(1j*patchImg, t);
  fftTerm{t+1} = fftshift(ifft2(patchFFT));
  
  % rescale terms as
  imOut(:,:,1)  = real(fftTerm{t+1});
  imOut(:,:,2)  = imag(fftTerm{t+1});
  imOut(:,:,3)  = 0.5;
  
  % rotate imOut by -90 degrees
  % find real and imaginary extrema of 
  % write imOut, extrema, dH, into files.
end
\end{lstlisting}
\label{alg:matlabprecomp}
\end{algorithm}

They key idea of algorithm $\ref{alg:matlabprecomp}$ is to compute iteratively the Fourier Transformation for different powers of the provided height field. These DFT values are scaled by according to their extrema values. Another note about the command fftshift: It rearranges the output of the ifft2 by moving the zero frequency component to the centre of the image. This simplifies the computation of DFT terms lookup coordinates during rendering.

\label{sec:precompmatlabfft}
\begin{figure}[H]
  \centering
  \subfigure[Heightfield of a Blazed Grating]{
    \includegraphics[scale=0.25]{implementation/hf/blaze/blazeBig.png}
    \label{fig:matlabBlazePatch}
  }
~
  \subfigure[Plot of extreme values for different powers of blazed grating]{
    \includegraphics[scale=0.4]{implementation/hf/blaze/extrema.png}
    \label{fig:extremaBlaze}  
  }
  
  \subfigure[ImRe0]{
    \includegraphics[scale=0.6]{implementation/hf/blaze/AmpReIm0.png}
    \label{fig:blazeftimre0}
  }
~
  \subfigure[ImRe1]{
    \includegraphics[scale=0.6]{implementation/hf/blaze/AmpReIm1.png}
    \label{fig:blazeftimre1}
  }
~
  \subfigure[ImRe4]{
    \includegraphics[scale=0.6]{implementation/hf/blaze/AmpReIm4.png}
    \label{fig:blazeftimre4}
  }
~
  \subfigure[ImRe10]{
    \includegraphics[scale=0.6]{implementation/hf/blaze/AmpReIm10.png}
    \label{fig:blazeftimre10}
  }
~
  \subfigure[ImRe20]{
    \includegraphics[scale=0.6]{implementation/hf/blaze/AmpReIm20.png}
    \label{fig:blazeftimre20}
  }
 
  
  \subfigure[Re0]{
    \includegraphics[scale=0.6]{implementation/hf/blaze/re0.png}
    \label{fig:blazeftre0}
  }
~
  \subfigure[Re1]{
    \includegraphics[scale=0.6]{implementation/hf/blaze/re1.png}
    \label{fig:blazeftre1}
  }
~
  \subfigure[Re4]{
    \includegraphics[scale=0.6]{implementation/hf/blaze/re4.png}
    \label{fig:blazeftre4}
  }
~
  \subfigure[Re10]{
    \includegraphics[scale=0.6]{implementation/hf/blaze/re10.png}
    \label{fig:blazeftre10}
  }
~
  \subfigure[Re20]{
    \includegraphics[scale=0.6]{implementation/hf/blaze/re20.png}
    \label{fig:blazeftre20}
  }
  
  \subfigure[Im0]{
    \includegraphics[scale=0.45]{implementation/hf/blaze/im0.png}
    \label{fig:blazeftim0}
  }
~
  \subfigure[Im1]{
    \includegraphics[scale=0.6]{implementation/hf/blaze/im1.png}
    \label{fig:blazeftim1}
  }
~
  \subfigure[Im4]{
    \includegraphics[scale=0.6]{implementation/hf/blaze/im4.png}
    \label{fig:blazeftim4}
  }
~
  \subfigure[Im10]{
    \includegraphics[scale=0.6]{implementation/hf/blaze/im10.png}
    \label{fig:blazeftim10}
  }
~
  \subfigure[Im20]{
    \includegraphics[scale=0.6]{implementation/hf/blaze/im20.png}
    \label{fig:blazeftim20}
  }
  
  \caption[DFT Terms for a Blazed grating]{A visualization of DFT terms for a height field of a Blazed Grating.}
  \label{fig:matlabBlazeFourierImages}
\end{figure}

In figure $\ref{fig:matlabBlazeFourierImages}$ we see examples of a visualization of Fourier Transformations generated by our Matlab script for a blazed grating$\footnote{A blazed grating is a height field consisting of ramps, periodically aligned on a given surface.}$ as an input height field image, shown in figure $\ref{fig:matlabBlazePatch}$. Figure $\ref{fig:extremaBlaze}$ shows plots of the extreme values of DFT terms for different powers of the blazed grating. We recognize that, the higher the power of the grating becomes, the closer the extreme values of the corresponding DFT terms get. The figure line from figure $\ref{fig:blazeftimre0}$ until figure $\ref{fig:blazeftimre20}$ show us example visualizations of DFT terms for different powers of our grating's height field. Remember that DFT terms are complex valued matrices of dimension as their height field has. In this visualization, all real part values are stored in the red- and the imaginary parts in the green color channel of an DFT image. The figure line from figure $\ref{fig:blazeftre0}$ till figure $\ref{fig:blazeftre20}$ show us the real part images from above's line corresponding figures. Similarly for the figur line from figure $\ref{fig:blazeftim0}$ until figure $\ref{fig:blazeftim20}$ showing the correspinding imaginary part DFT term images.

\section{Java Renderer}
This section explains the architecture of the rendering program which I implemented$\footnote{This program is based on the code of a java real-time renderer, developed as a student project in the computer graphics class, held by M. Zwicker in autumn 2012.}$ and used for this project. The architecture of the program is divided into two parts: a rendering engine, the so called jrtr (java real time renderer) and an application program. Figure $\ref{fig:rendererArchitecture}$ outlines the architecture of the renderer. 

\begin{figure}[H]
  \centering
  \includegraphics[scale=0.7]{implementation/framework.png}
  \caption[Renderer Architecture]{Schematical architecture of our Java renderer.}
  \label{fig:rendererArchitecture}
\end{figure}

The application program relies on the MVC (Model-View-Controller) architecture pattern. The View just represents a canvas in which the rendered images are shown. The Controller implements the event listening functionalities for interactive rendering within the canvas. The Model of our application program consists of a Fabricator, a file reader and a constants manager. The main purpose of a Fabricator is to set up a rendering scene by accessing a constant manager containing many predefined scene constants. A scene consists of a camera, a light source, a frustum, shapes and their associated material constants. Such materials include a shape texture, precomputed DFT terms$\footnote{See section \ref{sec:precompmatlabfourierimages} for further information.}$ for a given height field$\footnote{and other height field constants such as the maximal height of its bumps or its pixel real-world width correspondance.}$ like visualized in figure $\ref{fig:matlabBlazeFourierImages}$. A shapes is a geometrical object defined by a triangular mesh as shown in figure $\ref{fig:wireframemesh}$. 

\begin{figure}[H]
  \centering
  \includegraphics[scale=0.7]{implementation/wireframemesh.png}
  \caption[Triangular Mesh]{Representation$\footnotemark$ of a triangular mesh represents an object as a set of triangles and a set of vertices.}
  \label{fig:wireframemesh}
\end{figure}
\footnotetext{Modified image which originally has been taken from \texttt{http://en.wikipedia.org/wiki/Polygon\textunderscore mesh}}

Such a mesh is represented as a data structure consisting of a list of vertices, each stored as a triplet of $x$, $y$, $z$ positions and triangles, each defined by a triple of vertex-indices. Besides its position, a vertex can have further data assigned to, like a surface color, normals and texture coordinates. The whole scene is encapsulated in a scene graph data-structures, defined and managed within the rendering engine. A scene graph contains all scene geometries and their transformations in a tree like structured hierarchy. \\

All required configuration, in oder to communicate with the GPU through OpenGL, is performed in the jrtr rendering engine. Furthermore, jrtr's render context object, the wholeresource-management for various types of low-level buffers, which are used within the rendering pipeline by our GLSL shaders, takes place in the rendering place. More precisely, this means allocating memory for the buffers, assigning them with scene data and flushing them, when not used anymore. The whole shading process is performed in the GPU, stage-wise: The first stage is the vertex shader (see section $\ref{sec:vertexshader}$) followed by the fragment shader (see section $\ref{sec:fragmentshader}$). The jrtr framework also offers the possibility to assign user-defined shaders written in GLSL.

\section{GLSL Diffraction Shader}
\subsection{Vertex Shader}
\label{sec:vertexshader}
In our implementation we want to simulate the structural colors a viewer sees when light diffracted is on grating, e.g. on the skin of a snake. For this purpose, we reproduce a 2d image of a given 3d scene as seen from the perspective of a viewer for given lightning conditions. The color computation of an image is performed in the GPU shaders of the rendering pipeline. In OpenGL, there are two basic shading stages performed to rendern an image whereas the vertex shader is the first shading stage in the rendering pipeline. \\

As an input, a vertex shader receives one vertex of a mesh and other vertex data such as a vertex normals. It only can access this data and has no information about the neighborhood of a vertex or the topology of its corresponding shape. Since vertex positions of a shape are defined in a local coordinate system$\footnote{Defining the positions of a shape in a local coordinate system simplifies its modeling process and allows us to apply transformations to a shape.}$ and we want to render an image of the perspective of viewer, we have to transform the locally defined positions to a perspectively projected viewer space. Therefore, the main purpose$\footnote{Furthermore, texture coordinates used for texture-lookup within the fragment shader and per vertex lightning can be computed.}$ of a vertex shader is to tranform the position of vertices. Notice that a vertex shader can manipulate the position of a vertices, but cannot generate additional mesh vertices. Therfore, the output of any vertex shader is a transformed vertex position. Keep in mind that all vertex shader outputs will be used within the fragment shader. For an example, please have a look at our fragment shader $\ref{sec:fragmentshader}$. \\
% TODO add an illustration of a vertex shader here, SHOW shader pipeline (stages: vertex shader, raterizer, fragment shader)illustration here
In the following let us consider the whole transformation, applied in the vertex shader, in depth. Let $p_{local}$ denote the position of a shape vertex, defined in a local coordinate system. Then the transformation from $p_{local}$
into the perspective projected position as seen by a observer $p_{projective}$ looks like the following:

\begin{equation}
  p_{projective} = P \cdot C^{-1} \cdot M \cdot p_{local}
  \label{eq:vertextransformation}
\end{equation}

where $P$, $C^{-1}$ and $M$ are transformation matrices$\footnote{These tranformation matrices are linear transformations expressed in homogenous coordinates.}$, defined the following way:

\begin{description}
\item[Model matrix $M$:] Each vertex position of a shape is initially defined in a local coordinate system. To make is feasible to place and transform shapes in a scene, a reference coordinate system, the so called world space, has to be introduced. Hence, for every shape a matrix $M$ is associated, defining the transformation from its local coordinate system into the world space. 
\item[Camera matrix $C$:] A camera models how the eye of a viewer sees an object, defined in world space like shown in figure $\ref{fig:cameracoordinatesystem}$. For calculating the transformation matrix $C$, a viewer's eye position and viewing direction, each defined in world space, are required. Therefore, $C$ denotes a transformation from coordinates defined in camera space into the world space. Thus, in order to transform a position from world space to camera space, we have to use the inverse of $C$, denoted by $C^{-1}$. 
\item[Frustum $P$:] The Matrix P defines a perspective projection onto image plance, i.e. for any given position in camera space, $P$ determines the corresponding 2d image coordinate. Persepctive projections project along rays that converge in center of projection.
\end{description}

Since we are interested in modeling how a viewer sees structural colors on a given scene shape as shown in figure $\ref{fig:cameracoordinatesystem}$, modeling a viewer's eye by formulating the corresponding camera matrix $C$, is the most important component of the whole transformation series applied in the vertex shader. Hence, we next will have a closer look in how a camera matrix $C$ actually can be computed. 

\begin{figure}[H]
  \centering
  \includegraphics[scale=0.7]{implementation/cameracoodsyst.png}
  \caption[Camera Coordinate System]{Illustration$\footnotemark$ of the Camera coordinate system where its origin defines the center of projection of camera.}
  \label{fig:cameracoordinatesystem}
\end{figure}
\footnotetext{This image has been taken from the lecture slides of computer graphics class 2012 which can be found on ilias.}

The camera matrix $C$ is constructed from its center of projection $e$, the position $d$ where the cameras looks at and a direction vector $up$, defining what is the direction in camera space pointing upwards. These components, $e$, $d$ and $up$, are defined in world coordinates. Figure $\ref{fig:cameramatrix}$ illustrates geometrical setup required in order to construct $C$.

\begin{figure}[H]
  \centering
  \includegraphics[scale=0.35]{implementation/cameramatrix.png}
  \caption[Camera Matrix]{Illustration$\footnotemark$ of involved components in order to construct the camera matrix $C$. The eye-vector $e$ denotes the positon of the camera in space, $d$ is the position the camera looks at, and $up$ denotes the cameras height. The camera space is spanned by the vectors helper vectors $x_c$, $y_c$ and $z_c$. Notice that objects we look at are in front of us, and thus have negative $z$ values}
  \label{fig:cameramatrix}
\end{figure}
\footnotetext{This image has been taken from the lecture slides of computer graphics class 2012 which can be found on ilias.}

The mathematical representation of these vectors, $x_c$, $y_c$ and $z_c$, spanning the camera space, introduced in figure $\ref{fig:cameramatrix}$, looks like the the following: 

\begin{align}
  &z_c = \frac{e-d}{||e-d||} \nonumber \\
  &x_c = \frac{up \times z_c}{||up \times z_c||} \nonumber \\
  &y_c = z_c \times x_c
  \label{eq:cameraspacespanningvectors}
\end{align}

As we can see, $x_c$, $y_c$ and $z_c$ are independent unit vectors. Therefore, they span a 3d space, the so called camera matrix. In order to express a coordinate in camera space, we have to project it onto these unit vectos. Using a homogenous coordinates representation, this a projection onto these unit vectors can be formulated by the transformation matrix $C$:

\begin{equation}
  C = \begin{bmatrix} x_c & y_c & z_c & e \\ 0 & 0 & 0 & 1 \end{bmatrix}
  \label{eq:cameramatrixeq}
\end{equation}

In our vertex shader, besides transforming the vertex positions like described in equation $\ref{eq:vertextransformation}$, for every vertex, we also compute the direction vectors $\omega_i$ and $\omega_r$ described like in figure $\ref{fig:geometricsetup}$. Those direction vectors are transformed onto the tangent space, a local coordinate system spanned by a vertex's normal, tangent and binormal vector. For further information and more insight about the the tangent space, please bave a look at the appendix in the section $\ref{sec:tangentspace}$. The algorithmic idea of our vertex shader, stating all its computational steps, is conceptualized in algorithm $\ref{alg:vertexshader}$.

\begin{algorithm}[H]
\caption{Vertex diffraction shader pseudo code}
\begin{table}[H]
  \begin{tabular}{@{}lll@{}}
    \textbf{Input:} & \emph{Mesh} with vertex \emph{normals} and \emph{tangents}  \\
    & Space tranformations $\{M, C^{-1}, P\}$  \\
    & Light direction \emph{lightDirection}  \\
    \textbf{Output:} & Incident light and viewer direction $\omega_i,\ \omega_r$ \\
    & Transformed position $p_{per}$ \\
  \end{tabular} 
\end{table}
\textbf{Procedures:} $normalize()$, $span()$, $projectVectorOnTo()$  \\
\setlength{\fboxrule}{0pt} 
\begin{boxedminipage}{1.0\textwidth}
  \begin{algorithmic}[1]
      \ForAll{$VertexPosition \thinspace position \in Mesh$}
        \State $ vec3 \ N = normalize(M * vec4(normal,0.0).xyz)$
        \State $ vec3 \ T = normalize(M * vec4(tangent,0.0).xyz)$
        \State $ vec3 \ B = normalize(cross(N, T))$
        \State $ TangentSpace = span(N, T, B)$
        \State $ viewerDir = ((cop_{w}-position)).xyz$
        \State $ lightDir = normalize(lightDirection)$
        \State $ \omega_i = projectVectorOnTo(lightDir, TangentSpace)$
        \State $ \omega_r = projectVectorOnTo(viewerDir, TangentSpace)$
        \State $normalize(\omega_i); \ normalize(\omega_r)$
        \State $p_{per} = P \cdot C^{-1} \cdot M \cdot p_{obj}$
      \EndFor
  \end{algorithmic}
  \end{boxedminipage}
  \vskip1.5pt
\label{alg:vertexshader}
\end{algorithm}

As input, our vertex shader algorithme $\ref{alg:vertexshader}$ takes a mesh with of a given scene shape. Each of this vertexc should have a normal and a tangent assigned to. Furthermore, the direction of the scene light is required. For our implementation we always used directional light sources. An example of an directional light source is given in figure $\ref{fig:dirlightsource}$. 

\begin{figure}[H]
  \centering
  \includegraphics[scale=0.45]{implementation/dirlightsource.png}
  \caption[Rays of a Directional Light]{Illustration$\footnotemark$ of our light source setup. For a directional light source, all light rays are in parallel.}
  \label{fig:dirlightsource}
\end{figure}
\footnotetext{This image has been taken from the lecture slides of computer graphics class 2012 which can be found on ilias.}

Last, in order to transform the positions of our mesh like described in equation $\ref{eq:vertextransformation}$, we also have to pass these transformation matrices$\footnote{When speaking about transformation matrices, we are refering to the model, camera and frustum matrix.}$. For simplification purposes, we introduced the following helper procedures used in the vertex shading algorithm:

\begin{description}
  \item[normalize():] Computes the normalized version of a given input vector.
  \item[span():] Assembles a matrix from a given set of vectors. This matrix spanes a vector space.
  \item[projectVectorOnTo():] Takes two arguments, a vector and a matrix. The first argument is projected onto each column of a given matrix. And returns a vector living the space spaned by the given argument matrix.
\end{description}

The output of our vertex shader is on the one side the transformed vertex postion and on the other side the incident light $\omega_i$ and viewing direction $\omega_r$ both transformed into the tangent space. The output of the vertex shader is used as the input of the fragment shader, discussed in the next section.

\subsection{Fragment Shader}
\label{sec:fragmentshader}
In the previous section we gave an introduction to the first shading stage of the OpenGL rendering pipeline by explaining the basics of a vertex shaders. Furthermore, we conceptually discussed the idea behind our vertex shading algorithm formulated in algorithm $\ref{alg:vertexshader}$. Summarized, the main purpose of our vertex shader is to compute the light- and viewing-direction vectors $\omega_i$ and $\omega_r$ defined like in figure $\ref{fig:geometricsetup}$. \\

After the vertex-shading stage, the next stage in the OpenGL rendering pipeline is the \emph{rasterization} of mesh triangles. As an input, a rasterizer takes a triple of mesh-triangle spaning vertices, each previousely processed by a vertex shader. For each pixel lying inside the current mesh triangle, a rasterizer computes its corresponding position in the triangle. According to its computed position, a pixel also gets interpolated values of the vertices attributes of its mesh triangle assigned. The set of interpolated vertex attributes together with the computes position of a pixel is denoted as a fragment. Figure $\ref{fig:trianglerasterization}$ conceptualizes the idea of processed set of fragment computed by a rasterizer.

\begin{figure}[H]
  \centering
  \includegraphics[scale=0.25]{implementation/rasterizedtriangle.png}
  \caption[Triangle Rasterization]{Illustration$\footnotemark$ of fragments covered by mesh triangle computed by the OpengGL rasterizer.}
  \label{fig:trianglerasterization}
\end{figure}
\footnotetext{This image was taken from \texttt{http://en.wikibooks.org/wiki/GLSL\textunderscore Programming/Rasterization}}

A fragment shader as shown in figure $FIGUREFRSGEMNTSHADER$, is the OpenGL pipeline stage subsequent after a mesh triangle is rasterized in the rasterization stage. As input value, a fragment shader takes at least a fragment computed by the rasterizer. It is also pissible to assign custome, non-interpolated values to a fragment shader. For each fragment in the fragment shading stage, a color value is computed. Furthermore, a fragment shader computes a depth value for each of its fragments, determining their visbility. Since all existing scene vertices were perspectively projected onto the 2d image plane, the rasterizer could have produced several fragments having assigned the same pixel position. Therefore, among all fragments with the same pixel position, the outputed pixel color is equal to the color of that fragment, which depth is closest to the viewer.

%% TODO FIGUREFRSGEMNTSHADER

\label{sec:inputlists}
In this section we explain how to render structural colors resulting due to light diffracted on a natural grating, based on the model described in section $\ref{sec:spectralrendering}$. The color values of the produced structural colors, resulting from our model, are computed by our fragment shader which expects being provided by the follwoing input:

\begin{itemize}
  \item Precompute DFT terms of the provided height field as explain in section $\ref{sec:precompmatlabfourierimages}$.
  \item The processed output, produced during the vertex shading stage (see section $\ref{sec:vertexshader}$), which is, the light- and viewing-direction vectors $\omega_i$ and $\omega_r$.
  \item Fragments produced during the rasterization stage using the output of our vertex shader.
\end{itemize}

Apart from these basic, varying input values, the following set of shading constants are initialized:

\begin{itemize}
  \item The number of iterations used for the taylor series approximation, determining the the approximation accuracy.
  \item The wavelength spectrum $\Lambda = [\lambda_{min}, \lambda_{max}]$ with a certain discretizing level $\lambda_{step}$.
  \item The color weights of the $CIE_{XYZ}$ color matching functions.
  \item THe height field image resolution and its pixel to width correspondance.
\end{itemize}

By Using all these inputs, our fragment shader performs a numerical integration over the given wavelength spectrum $\Lambda$ for our final derived expression, stated in equation $\ref{eq:finalexpression}$. For this integration we use the trapezoidal-rule with uniform discretization of the wavelength spectrum at $\lambda_{step}=5nm$ step sizes. This implies we are compressing sampled frequencies to the region near the origin due to the fact we are dividing the $(u,v)$ by the wavelength $\lambda$ and this implies that the $(u,v)$ space is sampeled non-linearly. \\

In section $\ref{sec:gaussianwindow}$ we have seen that we have to multiply our DFT terms by a Gaussian Window in order to approximate the DTFT which our model is based on. This windowing approach is performed for each discrete $\lambda$ value using a window large enough to span $4\sigma_f$ in both dimensions. Our DFT terms are computed from height fields that span at least 65$\mu m^2$ and are sampled at a resolution of at least 100$\mu m$ . 
This ensures that the spectral response encompasses all the wavelengths in the visible spectrum. \\

\label{sec:proceduresdescripton}
Next, we will discuss the actual fragment shading algorithm, listed in algorithm $\ref{alg:fragmentshaderall}$. Note, that our fragment shading algorithm uses some helper procedures. The two most fundamental one is the $\textbf{getlookupCoords}$ which computes the lookup coordinates in the DFT terms for a given (u,v) defined like in equation $\ref{eq:uvw}$ and a wavelength $\lambda$. The actual computation these coordinates is described in section $\ref{sec:texturelookupcoords}$. Notice that the routine $\textbf{getLocalLookUp}$ computes a local lookup coordinates used during the gaussian window approximation explained in section $\ref{sec:gaussianwindow}$. The routine $\textbf{distVecFromOriginTo}$ is used to compute the direction vector from the current position of a fragment in texture space, to one of its texture space living 1-neighborhood neighbors.

\begin{algorithm}[H]
  \caption{Fragment diffraction shader pseudo code}
  \begin{table}[H]
    \begin{tabular}{@{}lll@{}}
      \textbf{Input:} & Precomputed DFT Terms  \\
      & Mesh Triangles  \\
      & $\omega_i$ and $\omega_r$  \\
      \textbf{Output:} & Structural Color of a pixel \\
      \textbf{Procedures$\footnotemark$:} & $\emph{getColorWeights}$: get colormatching value for wavelength $\lambda$(see section \ref{sec:colorspace})\\
      & $\emph{getlookupCoords}$: get lookup coordinate(Eq.\ref{eq:ublookup}) by viewing-\& light direction \\
      & $\emph{distVecFromOriginTo}$: get direction vector from origin to a given position \\
      & $\emph{getLocalLookUp}$: get lookup coordinate(Eq.\ref{eq:gaussianwindowlook}) of DFT windwowing values \\
      & $\emph{rescaledFourierValueAt}$: rescales dft terms according to equation \ref{eq:dfttermnormalization}\\
      & $\emph{gaussWeightOf}$: computes gaussian window according to equation \ref{eq:gaussweight} \\
      & $\emph{dot}$: computes the dot-product of two given vectors \\
      & $\emph{gammaCorrect}$: apply camma correction on RGB vector(see section \ref{subsec:colortransformations}) \\  
    \end{tabular} 
  \end{table}
  
  \setlength{\fboxrule}{0pt} 
  \begin{boxedminipage}{1.0\textwidth}
  \begin{algorithmic}[1]
    \ForAll{$Pixel \thinspace p \in Fragment$}
      \State \init $BRDF_{XYZ}, BRDF_{RGB}$ \myto $vec4(0.0)$
      \State $(u,v,w) = -\omega_i - \omega_r$
      \For{$(\lambda = \lambda_{min};\thinspace \lambda \leq \lambda_{max};\thinspace \lambda = \lambda + \lambda_{step})$}
        \State $xyzWeights = getColorWeights(\lambda)$
        \State $lookupCoord = getlookupCoords(u, v, \lambda)$
        \State \init $P$ \myto $vec2(0.0)$
        \State $k = \frac{2\pi}{\lambda}$
        \For{$(n = 0$ \myto $T)$}
          \State $taylorScaleF = \frac{(kw)^n}{n!}$
          \State \init $F_{fft}$  \myto $vec2(0.0)$
          \State $anchorX = int(floor(center.x + lookupCoord.x * fftImWidth)$
          \State $anchorY = int(floor(center.y + lookupCoord.y * fftImHeight)$
          \For{$(i=(anchorX-winW)$ \myto $(anchorX + winW))$}
            \For{$(j=(anchorY - winW)$ \myto $(anchorY + winW))$}
              \State $dist = distVecFromOriginTo(i,j)$
              \State $pos = getLocalLookUp(i,j,n)$
              \State $fftVal = rescaledFourierValueAt(pos)$
              \State $fftVal \asteq gaussWeightOf(dist)$
              \State $F_{fft} \pluseq fftVal$
            \EndFor
          \EndFor
          \State $P \pluseq taylorScaleF*F_{fft}$
        \EndFor
        \State $xyzPixelColor \pluseq dot(vec3(\left|P\right|^2), xyzWeights)$
      \EndFor
      \State $BRDF_{XYZ} = xyzPixelColor*C(\omega_i, \omega_r)*shadowF$
      \State $BRDF_{RGB}.xyz = D_{65}*M_{XYZ-RGB}*BRDF_{XYZ}.xyz$
      \State $BRDF_{RGB}= gammaCorrect(BRDF_{RGB})$
    \EndFor
  \end{algorithmic}
  \end{boxedminipage}
  \vskip1.5pt
  \label{alg:fragmentshaderall}
\end{algorithm}
\footnotetext{Please have a look at section $\ref{sec:proceduresdescripton}$ to see further descriptions of these procedures.} 

Please note that for simplification purposes we omitted some inpute values in algorithm $\ref{alg:fragmentshaderall}$. A complete input value list can be found in section $\ref{sec:inputlists}$. Last, a a brief description of some code sections of our fragment shading algorithm:

\myparagraph{From line 4 to 26:} 
This loop performs uniform sampling along wavelength spectrum $\Lambda$ for the spectral integration seen in section $\ref{sec:spectralrendering}$. The procedure $ColorWeights(\lambda)$ computes the color weight for the current wavelength $\lambda$ by a linear interpolation between the color weight and the values $\ceil{\lambda}$ and $\floor{\lambda}$. The color weights are stored in a external table accessed by our fragment shader$\footnote{This color weight table contains data for lambda steps in size of 1nm}$. At line 6 the procedure call $lookupCoord(u, v, \lambda)$ returns the coordinates for the texture lookup which are computed like described in equation $\ref{eq:scalelook}$. At Line 25 the diffraction color contribution, computed during the integration over the wavelength spectrum for each wavelength $\lambda$, is accumulated.

\myparagraph{From line 9 to 24:} 
This loop performs the Taylor series approximation using a predefined number of iterations. Basically, the spectral response is approximated for our current value for $(u,v,\lambda)$. According to section $\ref{sec:gaussianwindow}$, we can approximate a DTFT by multiplying a gaussian winodw by DFT terms. When interpreting our DFT terms by a set of matrices, a particular DFT term value, corresponding to the position of a given fragment, can be looked up at the index ($anchorX$, $anchorY$). For our the Gaussian-Windowing approach we use a one-neighborhood around ($anchorX$, $anchorY$) in order to approximate the DFT value for a fragment.

\myparagraph{From line 14 to 22:} 
In this inner most loop, the convolution of the gaussian window with the DFT terms of the given height field is performed. The routine $gaussWeightOf(dist)$ computes the weights in equation $~\eqref{eq:gaussweight}$ from the distance between the current fragment's position and the current neighbor's position in texture space. Local lookup coordinates for the current fourier coefficient $fftVal$ value, stored in the DFT terms, are computed at line 17 and computed like described in equation $\ref{eq:gaussianwindowlook}$. The actual texture lookup is performed at line 18 using those local coordinates. Inside the procedure $rescaledFourierValueAt$ the values of $fftVal$ are rescaled by its extrema values$\footnote{This extrema values were precomputed in Matlab during the precomputation stage and are passed as an input to our fragment shader.}$, since $fftVal$ is normalized according to the description from section $\ref{sec:precompmatlabfft}$. The current $fftVal$ value in iteration is scaled by the current Gaussian Window weight and then summed to the final neighborhood FFT contribution at line 20.

\myparagraph{After line 26:} 
At line 27 the gain factor $C(\omega_i, \omega_r)$ from equation $\ref{eq:cfact}$ is multiplied by the computed pixel color like formulated in equation $\ref{eq:cpterm}$. The gain factor contains the geometric term of equation $\ref{eq:geometricterm}$ and the Fresnel term $F$. For computing the Fresnel term we use the Schlick approximation defined in equation $\ref{eq:schlickapprox}$, using an refractive index of 1.5 since this is close to the measured value from snake sheds. Our BRDF values are scaled by a shadowing function as described in the appendix of the paper$\cite{diffstam}$. The reason or this is that the nanostructure of a snake skins is grooved forming V-cavities. therefore some regions on the snake surface is shadowed. Last, we transform our colors from the $CIE_XYZ$ colorspace to the $CIE_RGB$ space using the CIE Standard Illuminant $D65$. Last we apply a gamma correction on our computed RGB color values. Consult the section $\ref{subsec:colortransformations}$ for further insight.

\section{Technical Details}
\subsection{Texture Lookup}
\label{sec:texturelookupcoords}
In our fragment shader we want to access DFT coefficients of our height field at a given location $(u,v)$ (defined like in equation $\ref{eq:uvw}$) to compute strucural colors according to equation $\ref{eq:finalexpression}$. \\

For any given nano-scaled surface patch $P$ with a resolution of $A$ by $A$ $\mu m$, stored as a $N$ by $N$ pixel image $I$, one pixel in any direction corresponds to $dH = \frac{A}{N} \mu m$. In Matlab we computed a series of $n$ output DFT terms $\{I_{out_1},...,I_{out_n}\}$ from this image $I$, where the n-th image represents the DFT coefficients for the n-th DFT term in our Taylor Series approximation. Notice, that every DFT image value $(u,v)$ is in the range $[-\frac{f_s}{2}, \frac{f_s}{2}]^2$ where $f_s$ is the sampling frequency equal to $\frac{1}{dH}$. \\

In order to get access to these image within our shader, we have to pass them as GLSL textures. In a GLSL shader the texture coordinates are normalized, which means that the size of the texture maps to the coordinates on the range $[0,1]$ in each dimension. By convention the the bottom left corner of an image has the coordinates $(0,0)$, whereas the top right corner has the value $(1,1)$ assigned. \\ 

However, $u$, $v$ coordinates are assumed to be in range $[-\frac{f_s}{2}, \frac{f_s}{2}]$, but GLSL texture coordinates are in range $[0,1]$. Therefore, for every computed lookup coordinate pair $(u,v)$, we have to apply a affine tranformation to be in the correct range. Figure $\ref{fig:lookupexample}$.  illustrates this issue of different lookup ranges. In general, an afine tranformation is a mapping of the form $f(x) = sx + b$ where $s$ is a scaling and $b$ is a translation.

\begin{figure}[H]
  \centering
  \includegraphics[scale=0.7]{implementation/lookupexampleAmReIm4.png}
  \caption[Lookup DFT Coefficients in Textures]{Illustration of $(u,v)$ lookup using GLSL textures.}
\label{fig:lookupexample}
\end{figure}

Figure $\ref{fig:lookupexample}$ visualized a particular DFT coefficients image for a blazed grating. In GLSL texture coordinates the range of such an image is $[0,1]^2$ whereat in a (u,v) coordinates system this would correspond to the range $[-\frac{f_s}{2}, \frac{f_s}{2}]^2$. Regarding the way the DFT terms were computed in matlab, their zero frequncy, i.e. $(u,v) = (0,0)$, is located at the center of the given DFT image. Since by convention the bottom left corner of a GLSL texture corresponds to the value $(0,0)$ but we want to work with $(u,v)$ coordinates, we have to introduce an helper coordinate system $(u_{lookup}, v_{lookup})$. This helper coordinates can be computed by a affine transformation applied on $(u,v)$ coordinates. Hence, we will explain how to compute the scaling $s$ and translation $b$ components of our affine transformation: 

\myparagraph{Translation component $b$ of affine tranformation:}
Since the zero frequency component of DFT images is shifted towards its centre position, we have to shift the coordinates $u$ and $v$ to the center of the current $N$ by $N$ pixel image by a bias $b$. The translation $b$ is a constant value and is computed like the following:

\begin{align}
b = \left\{ \begin{array}{rl}
\frac{N}{2} &\mbox{ if $N \equiv_2 0$} \\
\frac{N-1}{2} &\mbox{ otherwise}
\end{array} \right.
\label{eq:bias}
\end{align}

\myparagraph{Scaling component $s$ of affine transformation:}
For the scaling part of our affine transformation, we have to think a little further: let us consider a $T$ periodic signal in time, i.e. $x(t) = x(t+nT)$ for any integer $n$. After applying the DFT, we have its discrete spectrum $X[n]$ with frequency interval $w0 = 2pi / T$ and temporal interval $dH$. \\

Let $k$ denote the wavenumber which is equal to $\frac{2 \pi}{\lambda}$ for a given wavelength $\lambda$. Then the signal is both, periodic with time period $T$ and discrete with temporal interval $dh$. This implies that its spectrum should be discrete with frequency interval $w_0$ and periodic with frequency period $\Omega = \frac{2 \pi}{dH}$. This gives us an idea how to discretize the spectrum. For amy surface patch $P$ which is periodically distributed on its surface, its frequency interval along the x-axis is equal to: 

\begin{align}
  w_0 = \frac{2 \pi}{T} = \frac{2 \pi}{N \cdot dH}
\end{align}

Thus, only wavenumbers that are integer multiples of $w_0$ after a multiplication with $u$ must be considered, i.e. $ku$ is integer multiple of $w_0$. Hence the lookup for the $u$-direction will look like:

\begin{align}
    s(u)
    &=\frac{ku}{w_0} \nonumber \\
    &= \frac{ku N dH}{2 \pi} \nonumber \\
    &= \frac{u N dH}{\lambda}
\label{eq:scalelook}
\end{align}

The lookup for the $v$-direction will be equal $s(v)$ defined like in equation $\ref{eq:scalelook}$.

\myparagraph{Final affine transformation:}
Using the translation from equation $\ref{eq:bias}$ and the scaling from equation $\ref{eq:scalelook}$, the tranfromed texture lookup-coorinates $(u_{lookup}, v_{lookup})$ for a given wavelength $\lambda$ is equal to:

\begin{align}
  (u_{lookup}, v_{lookup}) 
  &= \left(s(u)+b,s(v)+b \right) \nonumber \\
  &= \left( \frac{u N dH}{\lambda} + b, \frac{v N dH}{\lambda} + b \right)
\label{eq:ublookup}
\end{align}  

Note that for the Windowing Approach, used in algorthm $\ref{alg:fragmentshaderall}$, we are visiting a one-pixel-neighborhood around each $(u,v)$ coordinate pair. For any position $(i,j)$ of its neighbor-pixels, these local coordinates $(u_{lookup}^{local_i}, v_{lookup}^{local_j})$ around the origin $(u_{lookup}, v_{lookup})$ from equation $\ref{eq:ublookup}$ are equal to:

\begin{equation}
  (u_{lookup}^{local_i}, v_{lookup}^{local_j}) = (i,j)-(u_{lookup}, v_{lookup})
\label{eq:gaussianwindowlook}
\end{equation}

\subsection{Texture Blending}
So far, we have seen how to render structural colors caused when light is diffracted on a grating. But usually, many objects, such as a snake mesh, also have a texture associated with. Therefore, will have a closer look how to combine colors of a given texture with computed structural colors. For this purpose we will use a so called texture blending approach. This means that, for each pixel, its final rendered color is a weighted average of different color components, such as the diffraction color, the texture color and the diffuse color. In our shader the diffraction color is weighted by a constant $w_{diffuse}$, denoting the weight of the diffuse color part. The texture color is once scales by the absolute value of the Fresnel Term $F$ and once by $1-w_{diffuse}$. Algorithm $\ref{alg:textureblending}$ shows in depth how our texture blending is implemented:

\begin{algorithm}[H]
\caption{Texture Blending}
\begin{table}[H]
  \begin{tabular}{@{}lll@{}}
    \textbf{Input:} & $c_{texture}:$ Texture color at given fragment postion \\
    & $c_{diffraction}:$ Structural color at given fragment postion \\
    \textbf{Output:} & $c_{out}:$ Mixed fragment texture- and structural-color \\
  \end{tabular} 
\end{table}
\setlength{\fboxrule}{0pt} 
\begin{boxedminipage}{1.0\textwidth}
  \begin{algorithmic}[1]
    \State $\alpha = abs(F)$
    \If{$(\alpha > 1)$} \State $\alpha = 1$ \EndIf
    \State $diffraction_{diff} = (1-w_{diffuse}) \cdot c_{diffraction}$
    \State $text_{spec} = (1-\alpha) \cdot c_{texture}$
    \State $text_{diff} = w_{diffuse} \cdot c_{texture}$
    \State $c_{out} = diffraction_{diff} + text_{spec} + text_{diff}$
  \end{algorithmic}
  \end{boxedminipage}
  \vskip1.5pt
\label{alg:textureblending}
\end{algorithm}

\subsection{Color Transformation}
\label{subsec:colortransformations}

In our fragment shader, we access a table which contains precomputed CIE's color matching functions values$\footnote{Such a function value table can be found at \texttt{cvrl.ioo.ucl.ac.uk} for example}$ from $\lambda_{min} = 380 nm$ to $\lambda_{max} = 780 nm$ in $5 nm$ steps. In algorithm $\ref{alg:fragmentshaderall}$ we describe how to compute the $CIE_{XYZ}$ color values as described in section $\ref{sec:colorspace}$. We can transform the color values into $CIE_{sRGB}$ gammut by performing the following linear transformation:

\begin{equation}
\begin{bmatrix}R\\G\\B\end{bmatrix} = M \cdot \begin{bmatrix}X\\Y\\Z\end{bmatrix}
\end{equation} 

where one possible transformation is: 

\begin{equation}
  M = \begin{bmatrix} 0.41847 & -0.15866 & -0.082835\\ -0.091169 & 0.25243 & 0.015708\\ 0.00092090 & -0.0025498 & 0.17860 \end{bmatrix}
\end{equation}

One aspect we have to keep in mind is what is the tristimulus value of the color defining our white point in our images. Defining what tristimulus value whit corresponds to, usually depends on the application. For our shaders we use use the CIE Standard Illuminant $D65$. $D65$ is intended to represent an average midday sun daylight. Applying the D65 illuminant, the whole colorspace transformation will look like:

\begin{equation}
\begin{bmatrix}R\\G\\B\end{bmatrix} = M \cdot \begin{bmatrix}X / D65.x \\ Y / D65.y \\Z / D65.z \end{bmatrix} 
\end{equation}

Each component of the Standard Illuminant acts as rescaling factor according to the whitepoint definiton of $D65$ for our computed color values. Last, we perfrom a gamma correction on each pixel's $(R,G,B)$ value. A gamma correction is a non linear transformation which controls the overall brightness of an image$\footnote{For further information about gamma correction, please have a look in the book \emph{Fundamentals of Computer Graphics}$\cite{fundcg}$.}$.

\section{Discussion}
\label{sec:impldiscus}
\subsection{Comparison: per Fragment-vs. per Vertex-Shading}
In this chapter we have seen how our render is implemented. We discussed our fragment shader which is responsible for computing the structural color values resulting by diffracted light. Our fragment shader's runtime performance depends on the resoultion of the final rendered image, since there is a correspondance between pixels and fragments. Notice that we also could have computed the strucural colors during the vertex shading process. Then, the whole rendering performanance of our shading approach would only depend on the vertex count of our meshes. Shading on a per vertex basis is usually bad since in order to get a nice and accurate rendering, the vertex count of a mesh has to be big. Furthermore, a vertex shader produces poor results for shapes like a cube for example, according to our structural color model. Therfore, shading on a per fragment shader scales depending on the rendered image resolution and is independent of the mesh vertices (their distribution and count).

\subsection{Optimization of Fragment Shading: NMM Approach}
The fragment shader algorithm described in algorithm $\ref{alg:fragmentshaderall}$ performs a gaussian window approach by sampling over the whole wavelength spectrum in uniform step sizes. This algorithm slow, since we for each pixel we iterate over the whole lambda spectrum. Furthermore, for any pixel, we iterate over its one-neighborhood. When also considering the number of taylor series approximation steps, we will have a run-time complexity of

\begin{equation}
O(\#spectrtumIter \cdot \#taylorIter \cdot neighborhoodRadius^2)
\end{equation}
 
Our goal is to optimize this runtime. Instead sampling over the whole wavelength spectrum, we could instead integrate over a minimal number of  wavelengths contributing the most to our shading result. These values are elicited like the following: Lets consider $(u,v,w)$ defined as in equation $\ref{eq:uvw}$. Let $d$ be the spacing between two slits of a grating. For any $L(\lambda) \neq 0$ it follows $\lambda_{n}^{u} = \frac{d u}{n}$ and $\lambda_{n}^{v} = \frac{d u}{n}$. Therefore we can derive the following boundaries of $n$:

\begin{table}[H]
  \centering 
  \begin{tabular}{l l}
    If $u,v > 0$ & $N_{min}^{u} = \frac{d u}{\lambda_{max}} \leq n_{u} \leq \frac{d u}{\lambda_{min}} = N_{min}^{u}$ \\  
    & $N_{min}^{v} = \frac{d v}{\lambda_{max}} \leq n_{v} \leq \frac{d v}{\lambda_{min}} = N_{min}^{v}$ \\
    & \\
    If $u,v < 0$ & $N_{min}^{u} = \frac{d u}{\lambda_{min}} \leq n_{u} \leq \frac{d u}{\lambda_{min}} = N_{max}^{u}$ \\
    & $N_{min}^{v} = \frac{d v}{\lambda_{min}} \leq n_{v} \leq \frac{d v}{\lambda_{min}} = N_{max}^{v}$ \\
    & \\
    If $(u,v)=(0,0)$ & $n_u = 0$ \\
    & $n_v = 0$ \\
  \end{tabular} 
  \label{eq:nminmaxapproximation}
\end{table}

By transforming those equations in the equation colletion $\ref{eq:nminmaxapproximation}$ to $(\lambda_{min}^{u}, \lambda_{min}^{u})$ and $(\lambda_{min}^{v}, \lambda_{min}^{v})$ respectively, we can reduce the total number of required iterations in our fragment shader. We denote this optimization by the $n_{min}$, $n_{max}$ (NMM) shading approach. 

\subsection{The PQ Shading Approach}
Another variant is the $PQ$ approach described in chapter 2 $\ref{sec:pq}$. Depending on the interpolation method, there are two possible variants we can think of as described in $\ref{sec:sincinterpolation}$. Either we try to interpolate linearly or use sinc interpolation.
The first variant does not require to iterate over a pixel's neighborhood, it is also faster than the gaussian window approach. One could think of a combination of those tho optimization approaches. Keep in mind, both of these approaches are further approximation. The quality of the rendered images will suffer using those two approaches. The second variant, using the sinc function interpolation is well understood in the field of signal processing and will give us reliable results. The drawback of this approach is that we again have to iterate over a neighborhood within the fragment shader which will slow down the whole shading. The following algorithm describes the modification of the fragment shader  $\ref{alg:fragmentshaderall}$ in oder to use sinc interpolation for the PQ approach $\ref{sec:pq}$.  

\begin{algorithm}[H]
  \caption{Sinc interpolation for PQ approach}
  \begin{algorithmic}
    \ForAll{$Pixel \thinspace p \in Image \thinspace I$}
      \State $w_p = \sum_{(i,j) \in \mathcal{N}_{1}(p)} sinc(\Delta_{p,(i,j)} \cdot \pi + \epsilon) \cdot I(i,j)$
      \State $c_p = w_p \cdot (p^2 + q^2)^{\frac{1}{2}}$
      \State $render(c_p)$
    \EndFor
  \end{algorithmic}
  \label{alg:sincinterpolation}
\end{algorithm}

In a fragment shader we compute for each pixel $p$ in the current fragment its reconstructed function value $f(p)$ stores in $w_p$. $w_p$ is the reconstructed signal value at $f(p)$ by the sinc function as described in $\ref{sec:sincinterpolation}$.
We calculate the distance $\Delta_{p,(i,j)}$ between the current pixel $p$ and each of its neighbor pixels $(i,j) \in \mathcal{N}_{1}(p)$ in its one-neighborhood. Multiplying this distance by $\pi$ gives us the an angle used for the sinc function interpilation. We add a small integer $\epsilon$ in order to avoid division by zeros side-effects.


\newpage{\pagestyle{empty} \cleardoublepage}
% 
what is this chapter about
how is evaluation perfromed
our shader
evaluation java table generator provided by daljit.
explain how this program works:
subvariants: sample whole lambda space, just a few lambdas, pq approach.
how from those generated tables to matlab
how to read those
discussion


\chapter{Evaluation Data Acquisition}
\section{Data Acquisition}
For measurement on the true surface topography of snake sheds, samples are stuck on glass plates using double face tape, the animal was pushed up below a hollowed plate letting the skin emerging from the top of the plate. the surface of the scale under measurement should be large compared to the size of the drop to avoid wetting the plasmic membrane that would corrupt the reasing of the contact angle. 

Measurements were carried out using intermittent contact mode in a Burker Dimension 3100 atomic force microscope (AFM) under ambient conditions using a Nanoscope V controller. 

An AFM is a microscope that uses a tiny probe mounted on a cantilever to scan the surface of an object. The probe is extremely close tobut does not touchthe surface. As the probe traverses the surface, attractive and repulsive forces arising between it and the atoms on the surface induce forces on the probe that bend the cantilever. The amount of bending is measured and recorded, providing a map of the atoms on the surface. Atomic force microscopes can achieve magnification of a factor of 5 × 106, with a resolution of 2 angstroms, sufficient to resolve individual carbon atoms. Also called scanning force microscope.
is a very high-resolution type of scanning probe microscopy, with demonstrated resolution on the order of fractions of a nanometer, more than 1000 times better than the optical diffraction limit.

The tops used were etched silicon TESP tips with a nomminal frequency and force constant of 320 kHz and 42 N/m respectively. 

SHOW AFM IMAGE

\section{Diffraction Gratings}

The so called idealised grating is made up of a set of slits of spacing $d$. In order to cause diffraction that spacing must be wider than the wavelength of interest. Each slit in the grating acts as a quasi point source from which light propagates in all directions. The diffracted light is composed of the sum of interfering wave components emenating from each slit in the grating. 

SHOW FIGURE ABC

At any given point in space through which diffracted light may pass, the path-length to each slit in the grating will vary. Since the path-length varies, so will the phases of the waves at that point from each of the slits. This those waves will either add or subtract from one another due to the phenomenon of interference. Note that a point has maximal intensity if and only if at angleswhich would satisfy the relationship $d \frac{sin(\phi)}{\lambda} = |m|$. 

SHOW FIGURE ABC1

where m is a integer and m equal zero corresponds to specular light. 
The detailed distribution of diffraction depends on the detailed structure of the grating elements and the number of grating elements but the grating equation will always give the maximal intensity in the directions.

This derivation of the grating equation is based on an idealised grating. However, the relationship between the angles of the diffracted beams, the grating spacing and the wavelength of the light apply to any regular structure of the same spacing, because the phase relationship between light scattered from adjacent elements of the grating remains the same. The detailed distribution of the diffracted light depends on the detailed structure of the grating elements as well as on the number of elements in the grating, but it will always give maxima in the directions given by the grating equation.


MERGE ME

A diffraction grating which consists of a very large number of parallel, evenly spaced slits in an opaque sheet

A typical grating would have 10,000 slits in 1 cm and thus the slit separation is much smaller than that used in the double-slit experiment. When a beam of monochromatic light is allowed to pass through the grating placed in a spectrometer, images of the sources can be seen through the telescope at different angles.

ADD FIGURE

Since a diffraction grating is a multiple-slit plate, the maxima occur at exactly the same position as a double slit interference. However unlike a double slit, the bright fringes are sharper and brighter.

Suppose monochromatic light is directed at the grating parallel to its axis as shown. Let the distance between successive slits be d.

ADD FIGURE

The diffraction pattern on the screen is the result of the combined effects of diffraction and interference. Each slit causes diffraction, and the diffracted beams in turn interfere with one another to produce the pattern. The path difference between waves from any two adjacent slits can be found by dropping a perpendicular line between the parallel waves. By geometry, this path difference is d sin θ . If the path difference equals one wavelength or some integral multiple of a wavelength, waves from all slits will be in phase and a bright line will be observed at that point. Therefore, the condition for maxima in the interference pattern at the angle θ is 

d sin θ = n λ
where n = 0,1,2,3…..

Because d is very small for diffraction grating, a beam of monochromatic light passing through a diffraction grating is splitted into very narrow maxima(bright fringes) at large angles θ.

ADD FIGURE

The maximum number of orders that can be found by letting maximum θ = 90o and finding n using equation
n ≤ (d sin 90o)/λ

When a narrow beam of white light is directed at a diffraction grating along its axis, instead of a monochromatic bright fringe, a set of coloured spectra are observed on both sides of the central white band as shown in the Figure.

ADD FIGURE

For a diffraction grating, the condition for the nth order maximum is given by d sin θ = nλ.

Since θ increases with wavelength λ, red light which has the longest wavelength is diffracted through the largest angle. Violet light has the shortest wavelength and is diffracted the least. Thus, white light is split into its component colours from violet to red light. The spectrum is repeated in the different orders of diffraction. Only the zeroth order spectrum is pure white.

The figure below shows the difference between a monochromatic and a white light spectra.

ADD FIGURE

wo colours of different orders may overlap if their angles of diffraction θ are equal. 

MERGE ME

ADD FIGURE

The performance of a simple diffraction grating can best be shown with reference to Figure 2. Notice that the optical beam enters the periodic pattern (spatial fringe pattern) with a particular angle of incidence. The beam then separates into one or more orders according to the grating equation: EQ

Three characteristics of the simple diffraction grating stand out. As figure 2 shows, the zero order is not diffracted and therefore continues undisturbed (but has some loss of power). Also, for a given wavelength, the amount of beam turning is a function of the groove period and of the angle of incidence. Finally, no real diffracted beam exists when the wavelength is greater than twice the groove period. However, when the grove period is large compared to the wavelength many orders can exist. Indeed, Echelle gratings often operate over hundreds of orders.

Free Spectral Range:
The free spectral range of a diffraction grating is defined as the largest bandwidth in a given order which does not overlap the same bandwidth in an adjacent order. Referring to Figure 3, if ℓ1 and ℓ2 are the extremes of the spectrum band then overlapping will occur at the long wavelength end of the spectrum when ℓ2 in order m is diffracted at the same angle as in order m+1. Conversely, overlapping will occur at the short wavelength end when ℓ1 in order m coincides with ℓ2 order m-1. To avoid overlapping, the required conditions is:
ℓ2-ℓ1≦ℓ1/m or ℓ2-ℓ1≦ℓ2/(m-1)
however, since ℓ1 < ℓ2, we may say that the free spectral range is equal to the shortest wavelength in the allowed bandwidth divided by the order number.

ADD FIGURE


\section{Evaluation}
UNIFY ANGLE NAMES: 

In order to check the physical reliability of our method we applied it on a syntetic blazed grating, Elaphe and Xenopeltis snake sheds and evaluated its response using the grating equation. This equation models the relationship between the grating spacing and the angles of the incident and diffracted beams of light. 

When light at a wavelength $\lambda$ falls on a sample presenting a periodicity $d$ along the incident plane under an incident angle $\theta$ compared to the normal of the surface the angle $\phi$ corresponding to the direction of the emerging beam showing constructive interferences (maximum in intensity) is given by the grating equation:

\begin{equation}
  sin(\theta) = sin(\phi) + \frac{m \lambda}{d}
\end{equation}

In our evaluation we are interested in the zero order diffraction, i.e. m equals zero which corresponds to direct transmission or specular reflection in the case of a reflection grating. 
Within our evaluation we further assume that the incident light direction $w_i$ is given. In contrast the direction of the reflected wave $w_r$ is not given.
In Mathematics, a three dimensional direction vector is fully defined by two two angles, i.e. it can be represented by spherical coordinates with radius $r = 1$. By convention, we denote those two vectors by $\theta$ and $\phi$. Hence, $\theta_i$, $\phi_i$ and $\phi_r$ are given constants whereas $\theta_i$ is a free parameter for our evaluation simulation. Therefore, we are going to compare the maxima for peak viewing angle corresponding to each wavelength using data produced by our method against the maxima resulting by the grating equation.

\subsection{Precomputation}
SHOW IMAGE OF GRID

Before being able to compare the output produced by our method by the grating equation, we have to discretise the wavelength space $\Lambda$ and the range $\Theta$ of our free parameter $\theta_i$. We also have to initially assign  $\theta_i$, $\phi_i$ and $\phi_r$. For our experiments we choose the following initial setup: $\theta_i = 75$ $\phi_i = 0$ $\phi_r = 180$ degree.
Further we discretise the lambda space $\Lambda = \{\lambda | \lambda = \lambda_{min} + k\lambda_{step}, k \in \{0,..,C-1\}\}$ where $\lambda_{step} = \frac{\lambda_{max}-\lambda_{min}}{C-1}$ and $C$ is the discretisation level of the lambda space, in our scenario $C = 42$. We similarly discretise the angle space by predefining an minimal and maximal angle boundary and $ceil(angMax - angMin) / angInc$ is the number of angles. 
We are going implement a similar algorithm as the diffraction fragment shader algorithme on the grid $[\Lambda, \Theta]$ and will store its spectral response in a matrix $R = \{response(\lambda_i, \theta_{r}^{j}) | i index(\Lambda), j index(\Theta)\}$. We also have evaluated our other shaders, mentioned within the discussion of the derivation and implementation chapters.

\subsection{Data evaluation}

\begin{algorithm}
\caption{Evaluation: lambda thetar graph}
\begin{lstlisting}
% load all variables computed in java
lInc = (lMax - lMin)./(lambdaCnt-1);
lambda = lMin + lInc*(-1+[1:lambdaCnt]);
[maxC maxI] = max(response.');
viewAngForMax = angMin + angInc * (maxI-1);
plot(lambda, viewAngForMax,'-r');s

for thetaI=baseAngle-eps:0.5:baseAngle+eps,
	% grating equation
	thetaV = asin(lambda./dPeriod - sin(thetaI*pi()/180))*180/pi();
	if(thetaI==75)
		plot(lambda, thetaV,'+ b');
	else
		plot(lambda, thetaV,'. g');
	end
end

\end{lstlisting}
\end{algorithm}

\includegraphics[scale=0.5]{blaze2500_75.png}
\includegraphics[scale=0.5]{blaze2500_75_closeup1.png}
\includegraphics[scale=0.5]{blaze2500_75_closeup2.png}

red graph is based on data produced by our method, the blue and green graphs are plots from the grating equation for different angles. If the blue graph is close the the red one, then our method performs well. 

SHOW PLOTS AND TALK ABOUT THEM

\newpage{\pagestyle{empty} \cleardoublepage}

\chapter{Results}
show all views results.
differece of this shader compared to evaluation shader
show real snake images for comparison with real rendered images
show experiments received
show rendered images by daljits implemetation of stams approach.
show our renderer's results 
mention all input parameters and their values.
mention system specs and how long it took in order to precompute
show some idft2 images, used patch, besids rendered image
what initial size was used patch?
mention GEM results.
mention real results from geneva - use same paramter setup.

\section{Other stuff}
$\forall \colvec[x]{y}{z} \in \mathbb{R}^3 : \exists r \in [0,\infty) \exists \phi \in [0,2\pi] \exists \theta \in [0,\pi] $ s.t.
\begin{equation*}
\colvec[x]{y}{z} = \colvec[r sin(\theta)cos(\phi)]{r sin(\theta)sin(\phi)}{r cos(\theta)}
\end{equation*}

\section{Matlab code}
\section{Discussion}


\begin{figure}[ht]
  \centering
  \subfigure[Blaze grating]{
    \includegraphics[scale=0.12]{results/diffPatches/fftBlazeHeight_0.25Microns_allL_weak_scale.png}
    \label{fig:brdfmapBlaze}
  }
~
  \subfigure[elaphe grating]{
    \includegraphics[scale=0.12]{results/diffPatches/elaph65.png}
    \label{fig:brdfmapElaphe}
  }
~
  \subfigure[xeno grating]{
    \includegraphics[scale=0.12]{results/diffPatches/xeno65.png}
    \label{fig:brdfmapXeno}
  }
  \label{brdfmapsDiffPatches}
  \caption{BRDF maps for different patches}
\end{figure}


\begin{figure}[ht]
  \centering
  \subfigure[$\lambda_{step=1 nm}$]{
    \includegraphics[scale=0.12]{results/different_lambda_steps/blaze/dl=1.png}
    \label{fig:brdfmapsDiffLambdaStepsL1BlazeL}
  }
~
  \subfigure[$\lambda_{step=5 nm}$]{
    \includegraphics[scale=0.12]{results/different_lambda_steps/blaze/dl=5.png}
    \label{fig:brdfmapsDiffLambdaStepsL5Blaze}
  }
~
  \subfigure[$\lambda_{step=10 nm}$]{
    \includegraphics[scale=0.12]{results/different_lambda_steps/blaze/dl=10.png}
    \label{fig:brdfmapsDiffLambdaStepsL10Blaze}
  }
  
  \subfigure[$\lambda_{step=25 nm}$]{
    \includegraphics[scale=0.12]{results/different_lambda_steps/blaze/dl=25.png}
    \label{fig:brdfmapsDiffLambdaStepsL25Blaze}
  }
~
  \subfigure[$\lambda_{step=50 nm}$]{
    \includegraphics[scale=0.12]{results/different_lambda_steps/blaze/dl=50.png}
    \label{fig:brdfmapsDiffLambdaStepsL50Blaze}
  }
~ 
  \subfigure[$\lambda_{step=100 nm}$]{
    \includegraphics[scale=0.12]{results/different_lambda_steps/blaze/dl=100.png}
    \label{fig:brdfmapsDiffLambdaStepsL100Blaze}
  }
  
  \label{brdfmapsDiffLambdaStepsBlaze}
  \caption{Blaze grating: Different lambda step size}
\end{figure}



\newpage{\pagestyle{empty} \cleardoublepage}

\chapter{Conclusion}

can we do better?

brief overwiew pf results achieved, what was the most important in the work, appropriate to provide an introduction to possible future work in this field. reflect the emotions associated with the work, what was especially difficult or particularly interesting, one may elaborate on open questions within subjects related to the thesis without giving any answer. discuss follow-ups.

statment what you've researched and what your original contribution of the fild is
explain why our approach is a good idea
explain how the straight foreward approach would behave compared to our approach, computing the fourier transformations straight away.
explain what we achieved, summary
say something about draw-backs and about limitations of current apporach
say something about the ongoing paper future work
maybe say something about runtime complexity

\section{Further Work}
\subsection{References}
\begin{itemize}
\item \textbf{{[}1{]}} http://en.wikipedia.org/wiki/Ratio\_test
\item \textbf{{[}2{]}} http://math.jasonbhill.com/courses/fall-2010-math-2300-005/lectures/taylor-polynomial-error-bounds\end{itemize}
\section{Acknowledgment}
% ty to abcdefg

\newpage{\pagestyle{empty} \cleardoublepage}

\begin{appendix}
\chapter{Signal Processing Basics}
\label{chap:appendixsignalprocessing}
The aim of this appendix chapter is to provide the reader some basic knowledge in signal processing, Fourier transformations and related mathematical concepts. It is recommended to understand all the mathematical concepts mentioned in this appendix chapter in order to be able to follow all derivation steps performed during this thesis. 

\section{Signals in Signal Processing}
A \emph{signal} is a function that conveys information about the behavior or attributes of some phenomenon. In the physical world, any quantity that exhibits variation in time or in space (such as an image) is potentially a signal. Such a signal might carry information about the state of a physical system, or might convey a message between observers. \\

\section{Fourier Transformation}
The \emph{Fourier transform} is a mathematical tool which allows to transform a given function or rather a given signal from defined over a time- (or spatial-) domain into its corresponding frequency-domain. \\

The Fourier transform is an important image processing tool which is used to decompose an image into its sine and cosine components. The output of such a Fourier transformation represents the image in its frequency domain. On the other hand, the input image is usually in the spatial domain. In the frequency domain of the Fourier transformation of an image, each point represents a particular frequency contained in the spatial domain image. \\

Next let us consider some mathematical definitions of Fourier transformations. Let $f$ be a measurable function over $\mathds{R}^n$. Then, its continuous \emph{Fourier Transformation} (\textbf{FT}) $\mathcal{F}\{f\}$$\footnote{Note that for simplification purposes we omit some constant factors in our definitions for the Fourier transformation.}$ is defined as:
 
\begin{equation}
  \mathcal{F}_{FT}\{f\}(w) = \int_{\mathds{R}^n} f(x)e^{-iwt} dt
  \label{eq:cft}
\end{equation}

whereas its \emph{inverse transformation} is defined like the following which allows us to obtain back the original signal:

\begin{equation}
  \mathcal{F}_{FT}^{-1}\{f\}(w) = \int_{\mathds{R}} \mathcal{F}\{w\}e^{iwt} dt
  \label{eq:icft}
\end{equation}

Where $w$ usually denotes the \emph{angular frequency}, which is equal to 
\begin{equation}
  w = \frac{2 \pi}{T} = 2 \pi v_f 
\end{equation}

Furthermore denotes $T$ the \emph{period} of the spectrum and $v_f$ is its corresponding \emph{frequency}. Notice that the definitions we use for the Fourier transformation correspond to the definition physicist and mathematicians typically use. However, there are other possibilities to define the Fourier transformation such as the definition used in electrical engineering. For further information about this kind of Fourier transformation please have a look at section $\ref{sec:electricalengeneeringftconvention}$. \\

By using Fourier analysis$\footnote{Fourier analysis is the technique to approximate a function by a sums of simpler trigonometric functions.}$ we can derive \emph{Discrete Time Fourier Transform} (\textbf{DTFT}). The DTFT operates on a discrete function. Such an input function is often created by digitally sampling a continuous function. The DTFT generates a continuous and periodic signal in the frequency domain. This operator is defined as the following:

\begin{equation}
  \mathcal{F}_{DTFT}\{f\}(w) = \sum_{-\infty}^{\infty} f(x) e^{-iwk}
  \label{eq:dtft}
\end{equation}

Note that the DTFT is not practically suitable for digital signal processing since there a signal can be measured only in a finite number of points. Thus, by discretize its frequency domain we get the \emph{Discrete Fourier Transformation} (\textbf{DFT}) of the input signal:

\begin{equation}
  \mathcal{F}_{DFT}\{f\}(w) = \sum_{n=0}^{N-1} f(x) e^{-iw_{n}k}
  \label{eq:dft}
\end{equation}

Where the angular frequency $w_n$ is defined as the following:

\begin{equation}
  w_n = \frac{2\pi n}{N} 
\end{equation}
\noindent
and $N$ is the number of samples within an equidistant period sampling. \\

Any continuous function $f(t)$ can be expressed as a series of sines and cosines. This representation is called the \emph{Fourier Series} (\textbf{FS}) of $f(t)$.
\begin{equation}
  f(t) = \frac{1}{2}a_0 + \sum_{n=1}^{\infty} a_n cos(nt) + \sum_{n=1}^{\infty} b_n cos(nt)
  \label{eq:dfs}
\end{equation}

where

\begin{align}
    a_0 = \int_{-\pi}^{\pi} f(t) dt \nonumber \\
    a_n = \frac{1}{\pi}\int_{-\pi}^{\pi} f(t) cos(nt) dt \nonumber \\
    b_n = \frac{1}{\pi}\int_{-\pi}^{\pi} f(t) sin(nt) dt
\end{align}

Figure $\ref{fig:contdiscft}$ illustrated the relationships between the different Fourier transformation types.

\begin{figure}[H]
  \centering
  \includegraphics[scale=0.5]{background/dcft.png}
  \caption[Relationships between of Different Fourier Transformations]{Relationship$\footnotemark$ between the continuous Fourier transform and the discrete Fourier transform: Left column: A continuous function (top) and its Fourier transform $\ref{eq:cft}$ (bottom). Center-left column: Periodic summation of the original function (top). Fourier transform (bottom) is zero except at discrete points. The inverse transform is a sum of sinusoids called Fourier series $\ref{eq:dfs}$. Center-right column: Original function is discretized (multiplied by a Dirac comb) (top). Its Fourier transform (bottom) is a periodic summation (DTFT) of the original transform. Right column: The DFT $\ref{eq:dft}$ (bottom) computes discrete samples of the continuous DTFT $\ref{eq:dtft}$. The inverse DFT (top) is a periodic summation of the original samples.}
\label{fig:contdiscft}
\end{figure}
\footnotetext{This image has been taken from \texttt{http://en.wikipedia.org/wiki/Discrete\textunderscore Fourier\textunderscore transform}}

Table $\ref{tab:ftoperatorsdependencies}$ represents a summary of the different Fourier transformation types. It tells the reader which Fourier Operator take what kind of input signal and what properties its corresponding output signal will have.

\begin{table}[H]
    \begin{tabular}{l|l|l}
    \hline
    Spatial signal $f(t)$ is & Operator & Transformed frequency signal $\hat{f}(\omega)$ is\\
    \hline
    continuous and periodic in $t$ & $FS$ see Eq. $\ref{eq:dfs}$ & only discrete in $\omega$ \\
    only continuous in $t$ & $FT$ see Eq. $\ref{eq:cft}$ & only continuous in $\omega$\\
    only discrete in $t$ & $DTFT$ see Eq. $\ref{eq:dtft}$ & continuous and periodic in $\omega$\\
    discrete and periodic in $t$ & $DFT$ see Eq.$\ref{eq:dft}$ & discrete and periodic in $\omega$\\
    \hline
    \end{tabular}
\caption[Fourier Transform Mapping]{Fourier operator to apply for a given spatial input signal and the properties of its resulting output signal in frequency space}
\label{tab:ftoperatorsdependencies}
\end{table}

\section{The Convolution Operator}
In signal processing we use the \emph{convolution} operator mostly for performing any kind of filtering operations applied on given signals.
Mathematically, this operator is a form of combining two signals (i.e. weighting one signal by the other). The output of a convolution is always a continuous function. The convolution $f*g$ of two functions $f$, $g$$\colon \mathds{R}^n \to \mathds{C} $ is defined as:  
\begin{equation}
  \mathcal (f*g)(t) = \int_{\mathds{R}^n} f(t)g(t-x) dx
  \label{eq:convolution}
\end{equation}

Note that the Fourier transform of the convolution of two functions is equal to the product of their Fourier transforms. In other words a convolution in a spatial domain is equivalent to a multiplication in frequency domain. Therefore, the inverse Fourier transform of the product of two Fourier transforms is equal to the convolution of the two inverse Fourier transforms. \\

\section{Taylor Series}
In mathematics we use \emph{Taylor series} in order to approximate functions by a series of its derivatives. Conceptually, a Taylor series is mathematical concept which allows to represent a function by a certain infinite series.  \\

The Taylor series $\mathcal T$ of an infinitely differentiable real or complex valued function $f(x)$ evaluated on a point $a$ is equal to the following power series:

\begin{equation}
  \mathcal T(f;a)(x) = \sum_{n=0}^{\infty} \frac{f^{n}(a)}{n!}(x-a)^n
  \label{eq:deftaylor}
\end{equation}
\newpage{\pagestyle{empty} \cleardoublepage}

% how about removing this
\chapter{Summary of Stam's Derivations}
\label{chap:stamsderivations}

For his derivations Stam uses the Kirchhoff integral$\footnote{See \texttt{http://en.wikipedia.org/wiki/Kirchhoff\textunderscore integral\textunderscore theorem} for further information.}$, which is relating the reflected field to the incoming field. This equation is a formalization of Huygen’s well-known principle that states that if one knows the wavefront at a given moment, the wave at a later time can be deduced by considering each point on the first wave as the source of a new disturbance. Once the wave $\psi_1 =  e^{ik\mathbf{x} \cdot \mathbf{s}\mathbf{s}}$ on the surface is known, the emitted wave $\psi_2$ can be computed.

% instead want to compute use idea behind kirchhof or something sim
We want to compute an emitted wave $\psi_2$ which is the reflected wave of an incoming planar monochromatic wave 

\begin{equation}
\psi_1 = e^{ik \omega_i * x}
\end{equation}


where $\omega_i$ denotes its propagation direction. Using the Kirchhoff integral we get:

\begin{equation}
\psi_{2}(\omega_i, \omega_r) = \frac{i k e^{i K R}}{4 \pi R} (F(-\omega_i-\omega_r)-(-\omega_i+\omega_r)) \cdot I_{1}(\omega_i, \omega_r) 
\label{eq:kirchhoff}
\end{equation}

with

\begin{equation}
I_{1}(\omega_i, \omega_r) = \int_{S} \hat{\mathbf{n}} e^{ik(-\omega_i-\omega_{r}) \cdot \mathbf{s} d\mathbf{s}}
\label{eq:IBase}
\end{equation}

In applied optics, when dealing with scattered waves, one does use differential scattering cross-section rather than defining a BRDF which has the following identity: 

\begin{equation}
    \sigma^0 = 4 \pi \lim_{R \to \infty} R^2 \frac{\langle \left|\psi_2\right|^2\rangle}{\langle \left|\psi_1\right|^2\rangle}
\end{equation}

where R is the distance from the center of the patch to the receiving point $x_p$, $\hat{\mathbf{n}}$ is the normal of the surface at s and the vectors:

The relationship between the BRDF and the scattering cross section can be shown to be equal to 

\begin{equation}
 BRDF = \frac{1}{4\pi}\frac{1}{A}\frac{\sigma^0}{cos(\theta_i)cos(\theta_r)}
 \label{fig:crossscateringbrdfrelationship} 
\end{equation}

where $\theta_i$ and $\theta_r$ are the angles of incident and reflected directions on the surface with the surface normal $n$ according to figure ~\ref{fig:geometricsetup}.

The components of vector resulting by the difference between these direction vectors:
In order to simplify the calculations involved in his vectorized integral equations, Stam considers the components of vector 
\begin{equation}
  (u,v,w) = -\omega_i - \omega_r 
\label{eq:uvwappendix}
\end{equation}

explicitly and introduces the equation: 
\begin{equation}
  I(ku,kv) = \int_{S} \hat{\mathbf{n}} e^{ik(u,v,w) \cdot \mathbf{s} d\mathbf{s}} 
\label{eq:Istart}
\end{equation}

which is a first simplification of $\ref{eq:IBase}$. Note that the scalar $w$ is the third component of ~\ref{eq:uvw} and can be written as $w = -(cos(\theta_i)+cos(\theta_r))$ using spherical coordinates. The scalar $k=\frac{2\pi}{\lambda}$ represent the wavenumber.


During his derivations, Stam provides a analytical representation for the Kirchhoff integral assuming that each surface point $s(x,y)$ can be parameterized by $(x,y,h(x,y))$ where $h$ is the height at the position $(x,y)$ on the given $(x,y)$ surface plane. Using the tangent plane approximation for the parameterized surface and plugging it into $\ref{eq:Istart}$ he will end up with: 

\begin{equation}
    \mathbf{I}(ku, kv) = \int \int (-h_{x}(x,y), -h_{y}(x,y), 1) e^{ikwh(x,y)} e^{ik(ux + vy)} dx dy
\label{eq:I1}
\end{equation}

For further simplification Stam formulates auxiliary function which depends on the provided height field: 
\begin{equation}
  p(x,y) = e^{iwkh(x,y)} 
\label{eq:pxappendix}
\end{equation}

which will allow him to further simplify his equation $\ref{eq:I1}$ to:

\begin{equation}
    \mathbf{I}(ku, kv) = \int \int \frac{1}{ikw}(-p_x, -p_y, ikwp) dx dy
\label{eq:I2}
\end{equation}

where he used that $(-h_{x}(x,y), -h_{y}(x,y), 1)e^{kwh(x,y)}$ is equal to $\frac{(-p_x, -p_y, ikwp)}{ikw}$ using the definition of the partial derivatives applied to the function $\ref{eq:px}$.

Let $P(x,y)$ denote the Fourier Transform (FT) of $p(x,y)$. Then, the differentiation with respect to x respectively to y in the Fourier domain is equivalent to a multiplication of the Fourier transform by $-iku$ or $-ikv$ respectively. This leads him to the following simplification for $\ref{eq:I1}$:

\begin{equation}
    \mathbf{I}(ku, kv) = \frac{1}{w}P(ku, kv) \cdot (u,v,w)
\label{eq:I3}
\end{equation}

Let us consider the term $g = (F(-\omega_i - \omega_r)-(-\omega_i + \omega_r))$, which is a scalar factor of $\ref{eq:kirchhoff}$. The dot product with $g$ and $(-\omega_i - \omega_r)$ is equal $2F(1 + \omega_i \cdot \omega_r)$. Putting this finding and the identity $\ref{eq:I3}$ into $\ref{eq:kirchhoff}$ he will end up with:

\begin{equation}
\psi_{2}(\omega_i, \omega_r) = \frac{i k e^{i K R}}{4 \pi R} \frac{2F(1 + \omega_i \cdot \omega_r)}{w} P(ku, kv)
\label{eq:kirchhoffFinding}
\end{equation}

By using the identity $\ref{fig:crossscateringbrdfrelationship}$, this will lead us to his main finding:
\begin{equation} 
  BRDF_{\lambda}(\omega_i, \omega_r) = \frac{k^2 F^2 G}{4\pi^2 A w^2} \langle \left|P(ku, kv)\right|^2\rangle
\label{eq:mainstamappendix}
\end{equation}

where $G$ is the so called geometry term which is equal: 

\begin{equation}
  G =\frac{(1 + \omega_i \cdot \omega_r)^2}{cos(\theta_i)cos(\theta_r)}
\label{eq:geometrictermappendix}
\end{equation}
\newpage{\pagestyle{empty} \cleardoublepage}

\chapter{Derivation Steps in Detail}
\section{Taylor Series Approximation}
\label{chap:taylorseriesapproxappendix}

In order to prove equation $\ref{eq:taylorseriesapproximationofp}$ from section $\ref{sec:taylorapproximation}$ we have to show the following: For any $N\mathbb{\in N}$ and
\begin{equation}
 \sum_{n=0}^{N}\frac{(ikwh)^{n}}{n!}\mathcal{F}\left\{ h{}^{n}\right\} (\alpha,\beta) \approx P(\alpha,\beta) 
\end{equation}
we have to prove:
 
\begin{enumerate}
\item Show that there exist such an $N\mathbb{\in N}$s.t the approximation
holds true.
\item Find a value for B s.t. this approximation is below a certain error
bound, for example machine precision $\epsilon$. 
\end{enumerate}

\subsection{Proof Sketch of 1.}

By the \textbf{ratio test} relying on Taylor's Theorem$\footnote{Please have a look at \texttt{http://en.wikipedia.org/wiki/Taylors\textunderscore theorem} in order to see a proper definiton of the ratio test.}$ It is possible to show that the series $\sum_{n=0}^{N}\frac{(ikwh)^{n}}{n!}\mathcal{F}\left\{ h{}^{n}\right\} (\alpha,\beta)$ converges absolutely:

\textbf{Proof}: Consider $\sum_{k=0}^{\infty}\frac{y^{n}}{n!}$ where
$a_{k}=\frac{y^{k}}{k!}$. By applying the definition of the ratio test for this series it follows: 

\begin{equation}
 \forall y:limsup_{k\rightarrow\infty}|\frac{a_{k+1}}{a_{k}}|=limsup_{k\rightarrow\infty}\frac{y}{k+1}=0 
\end{equation}

Thus this series converges absolutely, no matter what value we will pick for y.

\subsection{Part 2: Find such an N}
Let $f(x)=e^{x}$. We can formulate its Taylor-Series, stated above. Let $P_{n}(x)$denote the n-th Taylor polynomial, 

\begin{equation}
 P_{n}(x)=\sum_{k=0}^{n}\frac{f^{(k)}(a)}{k!}(x-a)^{k}
\end{equation}

where $a$ is our developing point (here a is equal zero). 

We can define the error of the n-th Taylor polynomial to be $E_{n}(x)=f(x)-P_{n}(x)$. the error of the n-th Taylor polynomial is difference between the value of the function and the Taylor polynomial. This directly implies $|E_{n}(x)|=|f(x)-P_{n}(x)|$. By using the Lagrangian Error Bound it follows: 

\begin{equation}
 |E_{n}(x)|\leq\frac{M}{(n+1)!}|x-a|^{n+1} 
\end{equation}

with $a=0$, where \textbf{M} is some value satisfying $|f^{(n+1)}(x)|\leq M$ on the interval $I=[a,x]$. Since we are interested in an upper bound of the error and since \textbf{a} is known, we can reformulate the interval as $I=[0,x_{max}]$, where 

\begin{equation}
 x_{max} = \|i\| k_{max} w_{max} h_{max}
\end{equation}

We are interested in computing an error bound for $e^{ikwh(x,y)}$. Assuming the following parameters and facts used within Stam's Paper: 

\begin{itemize}
\item Height of bump: 0.15micro meters
\item Width of a bump: 0.5micro meters
\item Length of a bump: 1micro meters
\item $k=\frac{2\pi}{\lambda}$ is the wavenumber, $\lambda\in[\lambda_{min,}\lambda_{max}]$ and
thus $k_{max}=\frac{2\pi}{\lambda_{min}}$. Since $(u,v,w) = -\omega_i - \omega_r$ and both are unit direction vectors, 
each component can have a value in range {[}-2, 2{]}.
\item for simplification, assume$[\lambda_{min,}\lambda_{max}]=[400nm,700nm].$

\end{itemize}

We get:  

\begin{align}
x_{max}
 &= \|i\|*k_{max}*w_{max}*h_{max} \nonumber \\
 &= k_{max}*w_{max}*h_{max} \nonumber \\
 &=2*(\frac{2\pi}{4*10^{-7}m})*1.5*10^{-7} \nonumber \\
 &=1.5\pi
\end{align}

and it follows for our interval $I=[0,1.5\pi]$. Next we are going to find the value for $M$. Since the exponential function is monotonically growing (on the interval I) and the derivative of the \textbf{exp} function is the exponential function itself, we can find such an $M$: 

\begin{align*}
 M
 &=e^{x_{max}} \nonumber \\
 &=exp(1.5\pi)
\end{align*}

and $|f^{(n+1)}(x)|\leq M$ holds. With 

\begin{align}
|E_{n}(x_{max})|
 &\leq\frac{M}{(n+1)!}|x_{max}-a|^{n+1} \nonumber \\
 &= \frac{exp(1.5\pi)*(1.5\pi)^{n+1}}{(n+1)!}
\end{align}

we now can find a value of $n$ for a given bound, i.e. we can find an value of $N\mathbb{\in N}$ s.t. $\frac{exp(1.5\pi)*(1.5\pi)^{N+1}}{(N+1)!}\leq\epsilon$. With Octave/Matlab we can see: 

\begin{itemize}
\item if N=20 then $\epsilon\approx2.9950*10^{-4}$
\item if N=25 then $\epsilon\approx8.8150*10^{-8}$
\item if N=30 then $\epsilon\approx1.0050*10^{-11}$
\end{itemize}

With this approach we have that $\sum_{n=0}^{25}\frac{(ikwh)^{n}}{n!}\mathcal{F}\left\{ h{}^{n}\right\} (\alpha,\beta)$ is an approximation of $P(u,v)$ with error $\epsilon\approx8.8150*10^{-8}$. This means we can precompute 25 Fourier Transformations in order to approximate P(u,v) having an error $\epsilon\approx8.8150*10^{-8}$. 

\section{PQ approach}
\subsection{One dimensional case}
\label{sec:pqonedimappendix}

Since our series is bounded, we can simplify the right-hand-side of equation $\ref{eq:pqgeometricseries}$. Note that $e^{-ix}$ is a complex number. Every complex number can be written in its polar form, i.e. 

\begin{equation}
e^{-ix} = cos(x) + i sin(x) 
\label{eq:polarform}
\end{equation}

Using the following trigonometric identities
\begin{gather}
cos(-x) = cos(x) \nonumber \\
sin(-x) = -sin(x)
\end{gather}

combined with $\ref{eq:polarform}$ we can simplify the series $\ref{eq:pqgeometricseries}$ even further to:

\begin{align}
\frac{1-e^{iwT(N+1)}}{1-e^{-iwT}}
& =\frac{1-cos(wT(N+1)) + i sin(wT(N+1)) }{1-cos(wT) + i sin(wT)}
\label{eq:pq1minusexp}
\end{align}

Equation $\ref{eq:pq1minusexp}$ is still a complex number, denoted as $(p+iq)$. Generally, every complex number can be written as a fraction of two complex numbers. This implies that the complex number $(p+iq)$ can be written as $(p+iq) = \frac{(a+ib)}{(c+id)}$ for any $(a+ib), (c+id) \neq 0$. Let us use the following substitutions: 

\begin{align}
a& := 1 - cos(wT(N+1))&
b& =sin(wT(N+1)) \nonumber \\
c& =1-cos(wT)&
d& =sin(wT)
\label{eq:pqabcdsubstitudes}
\end{align}

Hence, using $\ref{eq:pqabcdsubstitudes}$, it follows 

\begin{equation}
  \frac{1-e^{iwT(N+1)}}{1-e^{-iwT}} = \frac{(a+ib)}{(c+id)}
\end{equation}

By rearranging the terms, it follows $(a+ib) = (c+id)(p+iq)$ and by multiplying its right hand-side out we get the following system of equations:

\begin{align}
(cp-dq)& =a \nonumber \\
(dp + cq)& =b
\label{eq:cdadcn}
\end{align}

After multiplying the first equation of $\ref{eq:cdadcn}$ by $c$ and the second by $d$ and then adding them together, we get using the law of distributivity new identities for $p$ and $q$:

\begin{align}
p& =\frac{(ac+bd)}{c^2 + d^2} \nonumber \\
q& =\frac{(bc+ad)}{c^2 + d^2}
\label{eq:pq1}
\end{align}

Using some trigonometric identities and putting our substitution from $\ref{eq:pqabcdsubstitudes}$ for $a$, $b$, $c$, $d$ back into the current representation $\ref{eq:pq1}$ of $p$ and $q$ we will get:

\begin{align}
p& =\frac{1}{2}+\frac{1}{2}\left(\frac{cos(wTN)-cos(wT(N+1))}{1-cos(wT)}\right) \nonumber \\
q& =\frac{sin(wT(N+1))-sin(wTN)-sin(wT)}{2(1-cos(wT))}
\end{align}

Since we have seen, that $\sum_{n=0}^N e^{-uwnT}$ is a complex number and can be written as $(p+iq)$, we now know an explicit expression for $p$ and $q$. Therefore, the one dimensional inverse Fourier transform of $S$ is equal:

\begin{align}
\mathcal{F}^{-1}\{S\}(w)
& =\mathcal{F}^{-1}\{f\}(w) \sum_{n=0}^{N} e^{-iwnT} \nonumber \\
& = (p+iq) \mathcal{F}^{-1}\{f\}(w)  
\label{eq:mainfinding1dappendix}
\end{align}

\subsection{Two dimensional case}
\label{sec:pqtwodimappendix}

\begin{align}
\mathcal{F}^{-1}\{S\}(w_1, w_2)
& = \int_{-\infty}^{\infty}\int_{-\infty}^{\infty} \sum_{n_2=0}^{N_1} \sum_{n_2=0}^{N_2} h(x_1 + n_1 T_1, x_2 + n_2 T_2) e^{iw(x_1 + x_2)}dx_1 dx_2 \nonumber \\
& = \int_{-\infty}^{\infty}\int_{-\infty}^{\infty} \sum_{n_2=0}^{N_1} \sum_{n_2=0}^{N_2} h(y_1, y_2) e^{iw((y_1 - n_1 T_1) + (y_2 + n_2 T_2))}dx_1 dx_2 \nonumber \\
& =\sum_{n_2=0}^{N_1} \sum_{n_2=0}^{N_2} \int_{-\infty}^{\infty}\int_{-\infty}^{\infty} h(y_1, y_2) e^{iw(y_1 + y_2)} e^{-iw(n_1 T_1 + n_2 T_2)}dy_1 dy_2 \nonumber \\
& =\sum_{n_2=0}^{N_1} \sum_{n_2=0}^{N_2} e^{-iw(n_1 T_1 + n_2 T_2)} \int_{-\infty}^{\infty}\int_{-\infty}^{\infty} Box(y_1, y_2) e^{iw(y_1 + y_2)} dy_1 dy_2 \nonumber \\
& =\left(\sum_{n_2=0}^{N_1} \sum_{n_2=0}^{N_2} e^{-iw(n_1 T_1 + n_2 T_2)}\right) \mathcal{F}^{-1}\{h\}(w_1,w_2) \nonumber \\
& =\left(\sum_{n_2=0}^{N_1} e^{-iw n_1 T_1}\right) \left(\sum_{n_2=0}^{N_2} e^{-iw n_2 T_2}\right) \mathcal{F}^{-1}\{h\}(w_1,w_2) \nonumber \\
& =(p_1 + i q_1)(p_2 + i q_2) \mathcal{F}^{-1}\{h\}(w_1,w_2) \nonumber \\
& =((p_1 p_2 - q_1 q_2) + i(p_1 p_2 + q_1 q_2)) \mathcal{F}^{-1}\{h\}(w_1,w_2) \nonumber \\
& =(p + iq) \mathcal{F}_{DTFT}\{h\}(w_1,w_2)
\label{eq:pqmainfindingappendix}
\end{align}

Where we have defined 

\begin{align}
p := (p_1 p_2 - q_1 q_2) \nonumber \\ 
q := (p_1 p_2 + q_1 q_2)
\label{eq:pqsubst2dappendix}
\end{align}
\newpage{\pagestyle{empty} \cleardoublepage}

\chapter{Appendix}
\section{Schlick's approximation}
The specular reflection coefficient $R$ can be approximated by:

\begin{equation}
 R(\theta) = R_0 + (1 - R_0)(1 - \cos \theta)^5
\label{eq:schlickapprox}
\end{equation}

and

\begin{equation*}
  R_0 = \left(\frac{n_1-n_2}{n_1+n_2}\right)^2
\end{equation*}

where $\theta$ is the angle between the viewing direction and the half-angle direction, which is halfway between the incident 
light direction and the viewing direction, hence $\cos\theta=(H\cdot V)$. And $n_1,\,n_2$ are the indices of refraction of the two medias at the interface and $R_0$ is the reflection coefficient for light incoming parallel to the normal (i.e., the value of the Fresnel term when $\theta = 0$ or minimal reflection). In computer graphics, one of the interfaces is usually air, meaning that $n_1$ very well can be approximated as 1.








\newpage{\pagestyle{empty} \cleardoublepage}
\end{appendix}

\addcontentsline{toc}{chapter}{\numberline{}List of Tables}
\listoftables

\addcontentsline{toc}{chapter}{\numberline{}List of Figures}
\listoffigures

\addcontentsline{toc}{chapter}{\numberline{}List of Algorithms}
\listofalgorithms

\addcontentsline{toc}{chapter}{\numberline{}Bibliography}
\bibliographystyle{alphadin}
\nocite{*}
\bibliography{thesis}


% This is required since 2012!!
\includepdf{Erklaerung.pdf}

\end{document}
