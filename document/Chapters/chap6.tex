\section{Conclusion}

can we do better?

brief overwiew pf results achieved, what was the most important in the work, appropriate to provide an introduction to possible future work in this field. reflect the emotions associated with the work, what was especially difficult or particularly interesting, one may elaborate on open questions within subjects related to the thesis without giving any answer. discuss follow-ups.

statment what you've researched and what your original contribution of the fild is
explain why our approach is a good idea
explain how the straight foreward approach would behave compared to our approach, computing the fourier transformations straight away.
explain what we achieved, summary
say something about draw-backs and about limitations of current apporach
say something about the ongoing paper future work
maybe say something about runtime complexity

\subsection{Further Work}
\subsubsection{References}
\begin{itemize}
\item \textbf{{[}1{]}} http://en.wikipedia.org/wiki/Ratio\_test
\item \textbf{{[}2{]}} http://math.jasonbhill.com/courses/fall-2010-math-2300-005/lectures/taylor-polynomial-error-bounds\end{itemize}
\section{Acknowledgment}
ty to abcdefg
\section{TODOs}

when use aplitude approach?
finish-up previous work.

\subsection{keep me alive}
Eine Welle ist die räumliche ausbreitende Veränderung bzw. Störung oder auch Schwingung einer Ort-und Zeitabhängigen phsikalischen Grösse. Wenn man in der Mathematik von einer Welle spricht, meint man die Wellenfunktion $y(x,t) = A sin(wt - kx)$ welche eine Lösung der Wellengleichung ist. Diese Funktionen hängen im Allgemeinen von Ort r und Zeit t ab.
Die maximale mögliche Auslenkung der Welle wird mit A bezeichnet. Die Phase einer Welle gibt an, in welchem Abschnitt innerhalb einer Periode sich die Welle zu einem Referenzzeitpunkt und -ort befindet
In der Natur vorkommende Wellen sind in den seltensten Fällen reine monochromatische Wellen, sondern eine Überlagerung aus vielen Wellen unterschiedlicher Wellenlängen. Die Überlagerung erfolgt dabei durch das Superpositionsprinzip, was mathematisch bedeutet, dass alle Wellenfunktionen der einzelnen Wellen addiert werden. Die Anteile der Wellenlängen werden als Spektrum bezeichnet. Z.B. Sonnenlicht ist eine Überlagerung aus elektromagnetischen Wellen. Das Spektrum umfasst einen Wellenlängenbereich von Infrarot über sichtbares Licht bis Ultraviolett. Kontinuierliches Spektrum. 
Dabei können verschiedene Effekte auftretten wie Interferenz − Überlagert man Wellen, so kann es zu einer konstruktiven Verstärkung, aber auch zu einer teilweisen oder gar totalen Auslöschung der Welle.
Wellen können auf unterschiedliche Art und Weise eine Änderung in ihrer Form erfahren wie durch wie Reflexion, Transmission, Brechung. In dieser Arbeit interessieren wie uns insbesondere für die Beugung von Wellen, which cannot be modeled using the standard ray theory of light.

Wenn eine Wellenfront durch ein Hindernis teilweise eingeschränkt wird, bewegt sich die Welle nicht nur in der durch die Strahlengeometrie gegebenen Richtung weiter, sondern es tritt auch eine komplizierte Wellenbewegung ausserhalb der geometrischen Strahlengrenzen auf. Diese Erscheinung wird Beugung genannt. Die Beugung von Wellen tritt grundsätzlich bei jeder Begrenzung der Welle durch ein Hindernis unmittelbar an den Rändern auf. 