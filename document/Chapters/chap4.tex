
\section{Evaluation Data Acquisition}

Wellengleichung: Lösung der DGL deren Lösung eine Wellenfunktion y(x,t) ist. Allgemeine Herleutung elektromag. Wellen aus den Maxwellschen Gleichungen.

$\frac{d^2 y}{dx^2} = \alpha frac{d^2 y}{dt^2}$

Harmonische Wellen: $y(x,t) = A sin(kx - wt) = A sin(2\pi \frac{x}{\lambda} + \delta)$ erzeugt durch superposition aus anderen harmonischen Wellen. Sei $k = \frac{2 \pi}{\lambda}$ dann $y(x,t) = A sin(kx)$

Stehende Wellen: Wellen nur in einem best. räumlichen Gebiet ausbreitbar. Beide Enden Reflexionen treten auf.

Interferenz harm. Wellen ist Überlagerung von Wellen: $y_1 = A sin(kx - wt)$ und $y_2 = A sin(kx - wt + \delta)$ dann $y_1 + y_2 = abcd$.

Interferenz: Zei Wellen gleicher Freq. bei ihrer Überlagerung wahrnehmbar räumlich Intensivitätsmuster erzeugt.

Welleneffekt wie Brechung, Reflexion, Transmission nun auch die Beugung.

Wenn eine Wellenfront durch ein Hindernis teilweise eingeschränkt wird, bewegt sich die Welle nicht nur in der durch die Strahlengeometrie gegebenen Richtung weiter, sondern es tritt auch eine komplizierte Wellenbewegung ausserhalb der geometrischen Strahlengrenzen auf. Diese Erscheinung wird Beugung genannt. Die Beugung von Wellen tritt grundsätzlich bei jeder Begrenzung der Welle durch ein Hindernis unmittelbar an den Rändern auf. Für die Teile der Wellenfront, die von der Kante weiter als einige Wellenlängen entfernt sind, ist die Beugung vernachlässigbar, und die Welle breitet sich in der Richtung der einfallenden Strahlen ungestört aus. Wenn die Wellenfronten auf ein Hindernis mit einer Blende /Lochh) treffen, die nur wenige Wellenlöngen breit ist, wird der gesamte Bereich der Wellenfronten an der Blende gebeugt. 

Stärke der Beugung hängt davon ab, ob die Wellenlänge gross oder klein im Vergleich zur Blendenabmessung ist. Wenn die Wellenlänge im Vergleich zu den Abmessungen der Blende gross ist, dann sind die Beugungseffekte gross und die durch die Blende hindurchgegangenen Wellen breiten sich weit in den geometrischen Schattenbeireich aus. Umgekehrt nur geringe Beugung.


Die Interferenzt ist die Bildung eines sichtbaren Intensitötsmisters infolge der ¨berlagerung von zwei oder mehr koheränten Wellen, die in einmem Raumpunkt zusammentreffen. Unter Beugung versteht man die Abweichung der Wellenausbreitung von der geometrischen Strahlenrichtung an einer Öffnung oder einem Hindernis im Lichtweg.

Interferenz: Zwei einander überlagende Lichquellen ergeben ein Interferenzmuster, wenn ihre Phsendifferenz während der für die Beobachtun notwenigen Zeitspanne konstant ist. Die Welle interferieren konstruktiv, wenn ihre Phasendifferenz nill ist oder ein ganzzahliges Vielfaches von 360 bzw 2PI beträgt. Sie interferieren destruktiv, wenn ohre Phasendifferenz 180 oder ein ungradzahliges Vielfaches von PI beträgt.

Beugung: Beugung tritt auf, wenn ein Teil einer Wellenfront durch ein Hindernis oder eine Öffnung begrenzt wird. Die Lichtintensität an einem beliebigen Punkt im Raum lässt sich mithilfe des Huygens'schen Prinzip bestimmen, indem man das daraus resultierende Interferenzmuster berechnet.

Beugungsgitter nestejt ais eomer grpssem Anzahl von äquidistanten Linien oder Spalten. Mit ihm kann die Wellenlänge des Lichts bestimmt werden, das von einer Lichtquelle ausgestrahlt wird. Doe Omterferenzmaxima m-ter Ordnung bei einem Gitter treten bei Winkeln auf, für die gilt: g sin thetam = m lambda. m = 0,1,2,...
Darin ist g die Gitterkonstante, dh, der Abstand von benachbarten Linien oder Spalten.

Frauenhofer Diffraction: sind in grossem Abstand vom beugenden Objekt (Hindernis oder Öffnung) zu beobachten. Hierbei treffen die gebeugten Strahlen nahezu parallel auf den Beobachtungsschirm. Sie können auch durch eine Linse geführt werden, in deren Brennebene sich der Schirm befindet.

Fresnel'sche Diffraction: wird in geringem Abstand vom beugenden Objekt beobachtet.





what is this chapter about
how is evaluation perfromed
our shader
evaluation java table generator provided by daljit.
explain how this program works:
subvariants: sample whole lambda space, just a few lambdas, pq approach.
how from those generated tables to matlab
how to read those
discussion

The true surface topography is extracted by flattening a single snake scale on a metal disc covered with carbon adhesive tape. Then, measurements were carried out using intermittment contact mode in a Bruker Dimension 3100 atomic force microscope (AFM) under ambient conditions, using a nanoscope V controller. The tops used were etched silicon TESP tips with a nomminal frequency and force constant of 320 kHz and 42 N/m respectively. 

In order to check the physical reliability of our mthod we used our method as a virtual diffraction experimental bench on a syntetic blazed grating, Elaphe and Xenopeltis snakes. When light at a wavelength $\lambda$ falls on a sample presenting a periodicity $a$ along the incident plane under an incident angle $\theta$ compared to the normal of the surface the angle $\phi$ corresponding to the direction of the emerging beam showing constructive interferences (maximum in intensity) is given by

GRATING EQUATION

where m is the order of diffraction. 

SHOW PLOTS AND TALK ABOUT THEM


\subsection{Diffraction Gratings}
Gratings may be of the reflective or transmissive type, analogous to a mirror or lens respectively. A grating has a zero-order mode (where m=0), in which there is no diffraction and a ray of light behaves according to the laws of reflection and refraction the same as with a mirror or lens respectively.

An idealised grating is considered here which is made up of a set of slits of spacing d, that must be wider than the wavelength of interest to cause diffraction. Assuming a plane wave of wavelength λ with normal incidence (perpendicular to the grating), each slit in the grating acts as a quasi point-source from which light propagates in all directions (although this is typically limited to a hemisphere). After light interacts with the grating, the diffracted light is composed of the sum of interfering wave components emanating from each slit in the grating. At any given point in space through which diffracted light may pass, the path length to each slit in the grating will vary. Since the path length varies, generally, so will the phases of the waves at that point from each of the slits, and thus will add or subtract from one another to create peaks and valleys, through the phenomenon of additive and destructive interference. When the path difference between the light from adjacent slits is equal to half the wavelength, λ/2, the waves will all be out of phase, and thus will cancel each other to create points of minimum intensity. Similarly, when the path difference is λ, the phases will add together and maxima will occur. The maxima occur at angles θm, which satisfy the relationship dsinθm/λ=|m| where θm is the angle between the diffracted ray and the grating's normal vector, and d is the distance from the center of one slit to the center of the adjacent slit, and m is an integer representing the propagation-mode of interest.


Thus, when light is normally incident on the grating, the diffracted light will have maxima at angles $\theta_m$ given by:
\begin{equation*}
d sin(\theta_m) = m\lambda
\end{equation*}

It is straightforward to show that if a plane wave is incident at any arbitrary angle θi, the grating equation becomes:

\begin{equation*}
d(sin(\theta_i) + sin(\theta_m)) = m \lambda
\end{equation*}

When solved for the diffracted angle maxima, the equation is:

\begin{equation*}
sin(\theta_m) = \left(\frac{m\lambda}{d}-sin(\theta_i)\right)
\end{equation*}

The light that corresponds to direct transmission (or specular reflection in the case of a reflection grating) is called the zero order, and is denoted m = 0. The other maxima occur at angles which are represented by non-zero integers m. Note that m can be positive or negative, resulting in diffracted orders on both sides of the zero order beam.

This derivation of the grating equation is based on an idealised grating. However, the relationship between the angles of the diffracted beams, the grating spacing and the wavelength of the light apply to any regular structure of the same spacing, because the phase relationship between light scattered from adjacent elements of the grating remains the same. The detailed distribution of the diffracted light depends on the detailed structure of the grating elements as well as on the number of elements in the grating, but it will always give maxima in the directions given by the grating equation.


$\forall \colvec[x]{y}{z} \in \mathbb{R}^3 : \exists r \in [0,\infty) \exists \phi \in [0,2\pi] \exists \theta \in [0,\pi] $ s.t.
\begin{equation*}
\colvec[x]{y}{z} = \colvec[r sin(\theta)cos(\phi)]{r sin(\theta)sin(\phi)}{r cos(\theta)}
\end{equation*}

\subsection{Matlab code}
\subsection{Discussion}


