what is this chapter about
how is evaluation perfromed
our shader
evaluation java table generator provided by daljit.
explain how this program works:
subvariants: sample whole lambda space, just a few lambdas, pq approach.
how from those generated tables to matlab
how to read those
discussion


\section{Evaluation Data Acquisition}

for measurement on sheds, samples are stuck on glass plates using double face tape, the animal was pushed up below a hollowed plate letting the skin emerging from the top of the plate. the surface of the scale under measurement should be large compared to the size of the drop to avoid wetting the plasmic membrane that would corrupt the reasing of the contact angle. 

the surface energy was derived from the contact angle measurements following the method proposed by Fowkes

microscopy
scaning method: scanning electron microscopy (sem) and atomic force microscopy (AFM).

The true surface topography is extracted by flattening a single snake scale on a metal disc covered with carbon adhesive tape. Then, measurements were carried out using intermittment contact mode in a Bruker Dimension 3100 atomic force microscope (AFM) under ambient conditions, using a nanoscope V controller. The tops used were etched silicon TESP tips with a nomminal frequency and force constant of 320 kHz and 42 N/m respectively. 


\subsection{Diffraction Gratings}

The so called idealised grating is made up of a set of slits of spacing $d$. In order to cause diffraction that spacing must be wider than the wavelength of interest. Each slit in the grating acts as a quasi point source from which light propagates in all directions. The diffracted light is composed of the sum of interfering wave components emenating from each slit in the grating. 

SHOW FIGURE ABC

At any given point in space through which diffracted light may pass, the path-length to each slit in the grating will vary. Since the path-length varies, so will the phases of the waves at that point from each of the slits. This those waves will either add or subtract from one another due to the phenomenon of interference. Note that a point has maximal intensity if and only if at angleswhich would satisfy the relationship $d \frac{sin(\phi)}{\lambda} = |m|$. 

SHOW FIGURE ABC1

where m is a integer and m equal zero corresponds to specular light. 
The detailed distribution of diffraction depends on the detailed structure of the grating elements and the number of grating elements but the grating equation will always give the maximal intensity in the directions.

This derivation of the grating equation is based on an idealised grating. However, the relationship between the angles of the diffracted beams, the grating spacing and the wavelength of the light apply to any regular structure of the same spacing, because the phase relationship between light scattered from adjacent elements of the grating remains the same. The detailed distribution of the diffracted light depends on the detailed structure of the grating elements as well as on the number of elements in the grating, but it will always give maxima in the directions given by the grating equation.


MERGE ME

A diffraction grating which consists of a very large number of parallel, evenly spaced slits in an opaque sheet

A typical grating would have 10,000 slits in 1 cm and thus the slit separation is much smaller than that used in the double-slit experiment. When a beam of monochromatic light is allowed to pass through the grating placed in a spectrometer, images of the sources can be seen through the telescope at different angles.

ADD FIGURE

Since a diffraction grating is a multiple-slit plate, the maxima occur at exactly the same position as a double slit interference. However unlike a double slit, the bright fringes are sharper and brighter.

Suppose monochromatic light is directed at the grating parallel to its axis as shown. Let the distance between successive slits be d.

ADD FIGURE

The diffraction pattern on the screen is the result of the combined effects of diffraction and interference. Each slit causes diffraction, and the diffracted beams in turn interfere with one another to produce the pattern. The path difference between waves from any two adjacent slits can be found by dropping a perpendicular line between the parallel waves. By geometry, this path difference is d sin θ . If the path difference equals one wavelength or some integral multiple of a wavelength, waves from all slits will be in phase and a bright line will be observed at that point. Therefore, the condition for maxima in the interference pattern at the angle θ is 

d sin θ = n λ
where n = 0,1,2,3…..

Because d is very small for diffraction grating, a beam of monochromatic light passing through a diffraction grating is splitted into very narrow maxima(bright fringes) at large angles θ.

ADD FIGURE

The maximum number of orders that can be found by letting maximum θ = 90o and finding n using equation
n ≤ (d sin 90o)/λ

When a narrow beam of white light is directed at a diffraction grating along its axis, instead of a monochromatic bright fringe, a set of coloured spectra are observed on both sides of the central white band as shown in the Figure.

ADD FIGURE

For a diffraction grating, the condition for the nth order maximum is given by d sin θ = nλ.

Since θ increases with wavelength λ, red light which has the longest wavelength is diffracted through the largest angle. Violet light has the shortest wavelength and is diffracted the least. Thus, white light is split into its component colours from violet to red light. The spectrum is repeated in the different orders of diffraction. Only the zeroth order spectrum is pure white.

The figure below shows the difference between a monochromatic and a white light spectra.

ADD FIGURE

wo colours of different orders may overlap if their angles of diffraction θ are equal. 




\subsection{Evaluation}
In order to check the physical reliability of our mthod we used our method as a virtual diffraction experimental bench on a syntetic blazed grating, Elaphe and Xenopeltis snakes. When light at a wavelength $\lambda$ falls on a sample presenting a periodicity $a$ along the incident plane under an incident angle $\theta$ compared to the normal of the surface the angle $\phi$ corresponding to the direction of the emerging beam showing constructive interferences (maximum in intensity) is given by

GRATING EQUATION

where m is the order of diffraction. 

SHOW PLOTS AND TALK ABOUT THEM


which angle are we iterating over and which is steady?
iterate over $ceil(angMax - angMin) / angInc$

when light at a wavelength $\lambda$ falls on a sample presenting a periodicity a along the incident plane, under an incident angle $\theta$ compared ti the normal of the surface, the angle $\phi$ corresponding to the direction of the meerging beam whoswing constructive interferences (maximum in intensity) is given by $sin(\phi) = sin(\theta) + m\frac{\lambda}{a}$
where $m$ is the order of diffraction. 


optical spectral range goes from $\lambda=240 nm$ to $\lambda=1.6 \mu m$
it is possible to measure $\phi$ angles from the grazing conditions (-90 degree) to a position where the detector is in conteact with the source





















