\section{Introduction}


rendering diffraction colors caused by coherent (as opposed to statistical) surface micro-geometry.
tables containing the result of the spectral integration predicted by the model of Stam. 
The theory is applied to render diffraction caused by acquired geometric models of the microstructure of snake skin, so as to reproduce the irridescent effects displayed by these animals.


\subsection{Motivation}
In Nature, coloring mostly comes from the inherent colors of materials but sometimes colorization has a pure physical origin such as the effect diffraction or interference of light. Both phenomenon are causing the so called structural coloration, which is the production of color through the interaction of visible light with micrioscopically structured surfaces. 
Color production is die to wave interference with quasiperiodic structures whose periodicity leads to interaction with visible light. Therefore we perceive color when the different wavelengths composing white light are selectively interfered with by matter (absorbed, reflected, refracted, scattered, or diffracted) on their way to our eyes, or when a non-white distribution of light has been emitted.
In animals, such as feathers of birds and the scales of butterflies, interference is created by a range of photonic mechanisms, including diffraction grating, selective mirrors, photonic crystals.
The connection beween microscopic structures and coloration has been observed by Robert Hooke in the early seventeenth centrury. The discovery of the wave nature of light led to the conclusion that the cause for the coloration lies in wave interference.

In the field of computer graphics, many researchers have been attempted rendering of structural colors but most of the techniques so far, however, are either too slow for interactive rendering or rely on simplofying assumption, like modeling light as rays, to achieve real-time performance. 

\subsection{Goals}
The purpose of this thesis is to render realtime structural colors caused by the effect of diffraction on differentr biological structures. We focus on structural colors generated by diffraction grating, in particular our approach applies to surfaces with quasiperiodic structures at the nanometer scale that can be represented as heighfields. such structures are found on the sehds of snkaes, wings of butterflies or the bodies of various insects. we focus on sknake skins and we acquirednanoscale heightfields of different snake sheds using atomic force microscopy. 

In oder to achieve this rendering purpose we will rely J. Stam's formulation of a BRDF which basically describes the effect of diffraction on a given surface assuming one knows the hightfield on this surface and will further extend it. 
In our case, those heightfields are small patches of the microstructured surface (in nano-scale) of AFM taken snake skin patches provided by GENEVA.

In his Paper, J. Stam assuming distribution on his heightfields whereas we require a an explicit provided hightfield of the surface or at least a small patch. Therefore, this work can be considered as an extension of J. Stam's derivations for the case one is provided by a explicit height field on a quasiperiodic structure.
Since one goal of this work is to render in realtime, we have to perform also precomputations which will require us to slightly modify Stam's main derivation.


\subsection{Key Concepts}

Diffraction occurs when the surface detail is comparable to the wave- length of light
Fourier analysis has enabled us to tackle the difficult task of computing the reflected waves off of these surfaces

Diffraction is a purely wave-like phenomenon which cannot be modeled using the standard ray the- ory of light.
Diffraction should not be confused with the related phenomenon of interference
Interesting diffraction phenomena, however, occur mostly when the surface detail is highly anisotropic, viz. non-isotropic
Interference produces colorful effects due to the phase differences caused by a wave traversing thin media of different indices of refraction
Interference,unlike diffraction, can be modeled using the ray theory of light alone

well known from physical optics that ray theory is only an approximation of the more fundamental wave theory
+ ray theory is sufficient to visually capture the reflected field from many commonly occurring surfaces
+ common belief that models based on wave theory are computationally too expensive to be of any use in computer graphics

derive analytical reflection models based on wave theory that capture the effects of diffraction


\subsection{Previous work}
stam, hooke, see our paper, see stams paper, see own research.


Robert Hooke = observed connection between microscopic structures and colorisation
wave nature of light led to conclusion that the cause for the coloration lies in wave interference.

\subsection{Overview}
outline chapters of thesis and show what is about.













