\section{Introduction}
\subsection{Motivation}
effect of diffraction, stam, genf, rendering snake skin 

Introducation
bla 
In phicss/bioligy

The purpose of this thesis is to render realtime the effect of diffraction on different snakes skins in a photorealistic manor. In oder to achieve this purpose we will rely J. Stam's formulation of a BRDF which basically describes the effect of diffraction on a given surface assuming one knows the hightfield on this surface.
In our case, those heightfields are small patches of the nanostracture of the snake skin provided by GENEVA taken by MIKROSKOP.
In his Paper, J. Stam assuming distribution on his heightfields whereas we require a an explicit provided hightfield of the surface or at least a small patch. Therefore, this work can be considered as an extension of J. Stam's derivations for the case one is provided by a explicit height field on a quasiperiodic structure.
Since one goal of this work is to render in realtime, we have to perform also precomputations which will require us to slightly modify Stam's main derivation.

\subsection{Related Work}
see papaer listing
\subsection{Thesis Outline}
describe what is which chapter

