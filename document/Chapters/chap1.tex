\section{Introduction}



$WIKI: http://en.wikipedia.org/wiki/Electromagnetic_radiation$

Electromagnetic radiation (EM radiation or EMR) is one of the fundamental phenomena of electromagnetism, behaving as waves propagating through space, and also as photon particles traveling through space, carrying radiant energy.

EMR is characterized by the frequency or wavelength of its wave. The electromagnetic spectrum, in order of increasing frequency and decreasing wavelength, consists of radio waves, microwaves, infrared radiation, visible light, ultraviolet radiation, X-rays and gamma rays. 

The eyes of various organisms sense a somewhat variable but relatively small range of frequencies of EMR called the visible spectrum or light. Higher frequencies correspond to proportionately more energy carried by each photon; for instance, a single gamma ray photon carries far more energy than a single photon of visible light.

EMR carries energy—sometimes called radiant energy—through space continuously away from the source

In general, EM radiation (the designation $'radiation'$ excludes static electric and magnetic and near fields) is classified by wavelength into radio, microwave, infrared, the visible spectrum we perceive as visible light, ultraviolet, X-rays, and gamma rays. Arbitrary electromagnetic waves can always be expressed by Fourier analysis in terms of sinusoidal monochromatic waves, which in turn can each be classified into these regions of the EMR spectrum.

The behavior of EM radiation depends on its frequency. Lower frequencies have longer wavelengths, and higher frequencies have shorter wavelengths, and are associated with photons of higher energy. There is no fundamental limit known to these wavelengths or energies, at either end of the spectrum, although photons with energies near the Planck energy or exceeding it (far too high to have ever been observed) will require new physical theories to describe.

$WIKI: http://en.wikipedia.org/wiki/Radiant_energy$
Radiant energy is the energy of electromagnetic waves.[1] The quantity of radiant energy may be calculated by integrating radiant flux (or power) with respect to time and, like all forms of energy, its SI unit is the joule. The term is used particularly when radiation is emitted by a source into the surrounding environment. Radiant energy may be visible or invisible to the human eye.

The term $"radiant energy"$ is most commonly used in the fields of radiometry, solar energy, heating and lighting,


$WIKI: http://en.wikipedia.org/wiki/Radiance$
Radiance and spectral radiance are measures of the quantity of radiation that passes through or is emitted from a surface and falls within a given solid angle in a specified direction. They are used in radiometry to characterize diffuse emission and reflection of electromagnetic radiation. In astrophysics, radiance is also used to quantify emission of neutrinos and other particles. The SI unit of radiance is watts per steradian per square metre (W·sr−1·m−2), while that of spectral radiance is W·sr−1·m−2·Hz−1 or W·sr−1·m−3 depending on whether the spectrum is a function of frequency or of wavelength.

Radiance characterizes total emission or reflection. Radiance is useful because it indicates how much of the power emitted by an emitting or reflecting surface will be received by an optical system looking at the surface from some angle of view. In this case, the solid angle of interest is the solid angle subtended by the optical system's entrance pupil. Since the eye is an optical system, radiance and its cousin luminance are good indicators of how bright an object will appear. For this reason, radiance and luminance are both sometimes called $"brightness"$. This usage is now discouraged – see Brightness for a discussion. The nonstandard usage of $"brightness"$ for $"radiance"$ persists in some fields, notably laser physics

Def $L = \frac{d^2 \Phi}{dA d\Omega cos(\theta)} \approx \frac{\Phi}{\Omega A cos(\theta)}$


\subsection{Motivation}
effect of diffraction, stam, genf, rendering snake skin 

Introducation
bla 
In phicss/bioligy

The purpose of this thesis is to render realtime the effect of diffraction on different snakes skins in a photorealistic manor. In oder to achieve this purpose we will rely J. Stam's formulation of a BRDF which basically describes the effect of diffraction on a given surface assuming one knows the hightfield on this surface.
In our case, those heightfields are small patches of the nanostracture of the snake skin provided by GENEVA taken by MIKROSKOP.
In his Paper, J. Stam assuming distribution on his heightfields whereas we require a an explicit provided hightfield of the surface or at least a small patch. Therefore, this work can be considered as an extension of J. Stam's derivations for the case one is provided by a explicit height field on a quasiperiodic structure.
Since one goal of this work is to render in realtime, we have to perform also precomputations which will require us to slightly modify Stam's main derivation.

\subsection{Related Work}
see papaer listing
\subsection{Thesis Outline}
describe what is which chapter

