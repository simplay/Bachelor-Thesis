\section{Introduction}

\subsection{Motivation}
In Nature, coloring mostly comes from the inherent colors of materials but sometimes colorization has a pure physical origin such as the effect diffraction or interference of light. Both phenomenon are causing the so called structural coloration, which is the production of color through the interaction of visible light with micrioscopically structured surfaces. 
Color production is die to wave interference with quasiperiodic structures whose periodicity leads to interaction with visible light. Therefore we perceive color when the different wavelengths composing white light are selectively interfered with by matter (absorbed, reflected, refracted, scattered, or diffracted) on their way to our eyes, or when a non-white distribution of light has been emitted.
In animals, such as feathers of birds and the scales of butterflies, interference is created by a range of photonic mechanisms, including diffraction grating, selective mirrors, photonic crystals.
The connection beween microscopic structures and coloration has been observed by Robert Hooke in the early seventeenth centrury. The discovery of the wave nature of light led to the conclusion that the cause for the coloration lies in wave interference.

In the field of computer graphics, many researchers have been attempted rendering of structural colors by formulating a the bidirectional reflectance distribution function (BRDF) for this purpose. But most of the techniques so far, however, are either too slow for interactive rendering or rely on simplifying assumption, like modeling light as rays, to achieve real-time performance, which are not able capturing the essence of diffraction at all. 

\subsection{Goals}
The purpose of this thesis is to render realtime structural colors caused by the effect of diffraction on different biological structures. We focus on structural colors generated by diffraction grating, in particular our approach applies to surfaces with quasiperiodic structures at the nanometer scale that can be represented as heighfields. such structures are found on the sehds of snkaes, wings of butterflies or the bodies of various insects. we restrict ourself and focus on different snake skins sheds which are acquired nanoscaled heightfields using atomic force microscopy. 

In oder to achieve our rendering purpose we will rely J. Stam's formulation of a BRDF which basically describes the effect of diffraction on a given surface assuming one knows the hightfield of this surface and will further extend this. Appart from Stam's approach, which models the heightfield as a probabilistic superposition of bumps and proceeds to derive an analytical expression for the BRDF, our BRDF representation takes the heightfield from explicit measurement. 
I.E. in our case, those heightfields are small patches of the microstructured surface (in nano-scale) of AFM taken snake skin patches provided by labs in GENEVA.
So this approach is closer to real truth, since we use measured surfaces instead of statistical surface profile.

Therefore, this work can be considered as an extension of J. Stam's derivations for the case one is provided by a explicit height field on a quasiperiodic structure.

Real time performance is achieved with a representation of the formula as a power series over a variable related to the viewing and lighting directions. Values closely related to the coefficients in that power series are precomputed.

The contribution is that this approach is more broadly applicable than the previous work. Although the previously published formula theoretically has this much flexibility already, there is a novel contribution in demonstrating how such generality can be leveraged in practical implementation


\subsection{Previous work}
stam, hooke, see our paper, see stams paper, see own research.


Robert Hooke = observed connection between microscopic structures and colorisation
wave nature of light led to conclusion that the cause for the coloration lies in wave interference.

\subsection{Overview}
outline chapters of thesis and show what is about.


