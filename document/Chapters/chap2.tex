\section{Theoretical Background}
\subsection{Definitions}

\subsubsection{Signal}
A signal is a function that conveys information about the behavior or attributes of some phenomenon.
In the physical world, any quantity exhibiting variation in time or variation in space (such as an image) is potentially a signal that might provide information on the status of a physical system, or convey a message between observers

\subsubsection{Fourier Transformation}
The Fourier-Transform is a mathematical tool which allows to transform a given function or rather a given signal from defined over a time- (or spatial-) domain into its corresponding frequency-domain.
 
Let $f$ an measurable function over $\mathcal{R}^n$. Then, the coninious Fourier Transformation, denoted as FT, $\mathcal{F}\{t\}$ of $f$ is defined as, ignoring all constant factors in the formula:
 
\begin{equation}
  \mathcal{F}\{w\}_{FT} = \int_{\mathcal{R}^n} f(x)e^{-iwt} dt
\end{equation}

whereas its inverse transform is defined like the following which allows us to obtain back the original signal:

\begin{equation}
  \mathcal{F}\{w\}^{-1}_{FT} = \int_{\mathcal{R}^n} \mathcal{F}\{w\}e^{iwt} dt
\end{equation}

By using fourier analysis, which is the approach to approximate any function by sums of simpler trigonometric functions, we gain the so called discrete time fourier transform (in short DTFT). The DTFT operates on a discrete function. Usually, such an input function is often created by diitally sampling a continius function. The DTFT itself is operation on a discretized signal on a continious, periodic frequency domain and looks like the following:

\begin{equation}
  \mathcal{F}\{w\}_{DFT} = \sum_{-\infty}^{\infty} f(x) e^(-iwk)
\end{equation}

we can further discretize the frequency domain and will get then the discrete fourier transformation (in short DFT) of the input signal:
\begin{equation}
  \mathcal{F}\{w\}_{DFT} = \sum_{n=0}^{N-1} f(x) e^(-iw_{n}k)
\end{equation}
Where the angular frequency $w_n$ is defined like the following $w_n = \frac{2\pi n}{N}$ and N is the number of samples within an equidistant periode sampling.

\subsubsection{Convolution}

\begin{equation}
  \mathcal (f*g)(t) = \int_{\mathcal{R}^n} f(t)g(t-x) dx
\end{equation}

Note that the Fourier transform of the convolution of two functions is the product of their Fourier transforms. This is equivalent to the fact that Convolution in spatial domain is equivalent to multiplication in frequency domain. Therefore, the inverse Fourier transform of the product of two Fourier transforms is the convolution of the two inverse Fourier transforms

\subsubsection{BRDF}
The bidirectional reflectance distribution function, in short BRDF, denoted as $f_r(w_i, w_r)$ is a four dimensional function that defines how light is reflected at an opaque surface. The function takes a negative incoming light direction, $\omega_{\text{i}}$, and outgoing direction, $\omega_{\text{r}}$, both defined with respect to the surface normal $\mathbf{n}$ and returns the ratio of reflected radiance exiting along $\omega_{\text{r}}$ to the irradiance incident on the surface from direction $\omega_{\text{i}}$
  
\begin{align}
  f_r(w_i, w_r)
  & = \frac{dL_r(w_r)}{dE_i(w_i)} \\
  & = \frac{dL_r(w_r)}{L_i(w_i)cos(\theta_i)dw_i}
\end{align}

where L is radiance, or power per unit solid-angle-in-the-direction-of-a-ray per unit projected-area-perpendicular-to-the-ray, E is irradiance, or power per unit surface area, and $\theta_{\text{i}}$ is the angle between $\omega_{\text{i}}$ and the surface normal, $\mathbf n$. The index $\text{i}$ indicates incident light, whereas the index $\text{r}$ indicates reflected light.

The reason the function is defined as a quotient of two differentials and not directly as a quotient between the undifferentiated quantities, is because other irradiating light than $\operatorname dE_{\text{i}}(\omega_{\text{i}})$, which are of no interest for $f_{\text{r}}(\omega_{\text{i}},\, \omega_{\text{r}})$, might illuminate the surface which would unintentionally affect $L_{\text{r}}(\omega_{\text{r}})$, whereas $\operatorname dL_{\text{r}}(\omega_{\text{r}})$ is only affected by $\operatorname dE_{\text{i}}(\omega_{\text{i}})$.

\subsubsection{Taylor Series}
Taylor series is a representation of a function as an infinite sum of terms that are calculated from the values of the function's derivatives at a single point.

The Taylor series of a real or complex-valued function ƒ(x) that is infinitely differentiable at a real or complex number a is the power series:
\begin{equation}
  \mathcal T(f;a)(x) = \sum_{n=0}^{\infty} \frac{f^{n}(a)}{n!}(x-a)^n
\end{equation}


\subsubsection{The Effect Of Diffraction}
+ mention frauenhofer diffraction, fresnel diffraction?
\subsubsection{BRDF - Spectral Rendering}
In Computer Graphics, spectral rendering is where a scene's light transport is modeled considering the whole span of wavelengths instead of R,G,B values (still relating on geometric optic, which ignore wave phase). The motivation is that real colors of the physical world are spectrum; trichromatic colors are only inherent to Human Visual System.

\subsection{Derivations}
\subsubsection{Taylor Series Approximation}

Based on J. Stam's Paper about Diffraction shaders we will show that
there is an approximation of his equation (5), \textbf{p(x,y)}, for
a explicitly given heightfield \textbf{h(x,y)}. This approximation
is achieved by using Taylor-Series and using this identity we will
further be able to approximate the Fourier-Transformation of p(x,y),
denoted as \textbf{P(u,v)}. Finally we will give an error bound for
this approximation.

\subsubsection{Taylor Series of p}

Given $p(x,y)=e^{ikwh(x,y)}$form Stam's Paper where h(x,y) is here
a given heightfield. Also given the definition $e^{y}=1+y+\frac{y^{2}}{2!}+\frac{y^{3}}{3!}+...=\sum_{n=0}^{\infty}\frac{y^{n}}{n!}$where
y can be real or even complex valued - note this identity can either
be derieved by power series or by Taylor-Series(using the derivatives
of the exp-function and developing the Taylor-Series around the point
a=0). Let us now set $y=ikwh(x,y)$where $i$ is the imaginary number.
For simplification, let us denote h(x,y) as h. It follows by our previous
stated identities: $e^{y}=1+(ikwh)+\frac{1}{2!}(ikwh)^{2}+\frac{1}{3!}(ikwh)^{3}+...=\sum_{n=0}^{\infty}\frac{(ikwh)^{n}}{n!}$.
Hence it holds $p(x,y)=\sum_{n=0}^{\infty}\frac{(ikwh(x,y))^{n}}{n!}.$


\subsubsection{Fourier Transformation of function p}

Let us now compute the Fourier Transformation of p(x,y) form above:$\mathcal{F}\left\{ p\right\} (u,v)=\mathcal{F}\left\{ \sum_{n=0}^{\infty}\frac{(ikwh(x,y))^{n}}{n!}.\right\} =^{\mathcal{F}\, lin\, Operator}\sum_{n=0}^{\infty}\mathcal{F}\left\{ \frac{(ikwh(x,y))^{n}}{n!}\right\} =\sum_{n=0}^{\infty}\frac{(ikwh)^{n}}{n!}\mathcal{F}\left\{ h(x,y){}^{n}\right\} $.
Hence it follows: $P(\alpha,\beta)=\sum_{n=0}^{\infty}\frac{(ikwh)^{n}}{n!}\mathcal{F}\left\{ h{}^{n}\right\} (\alpha,\beta)$.

\textbf{NB}: $\mathcal{F}\left\{ h{}^{n}\right\} (u,v)$ denotes the
two dimensional Fourier Transformation of p(x,y) and can be nummerically
computed by the two dimensional \textbf{DFT} or rather by the two
dimensional \textbf{FFT} over h(x,y). 


\subsubsection{Approximation of function P}

Next we are going to look for an $N\mathbb{\in N}$s.t. $\sum_{n=0}^{N}\frac{(ikwh)^{n}}{n!}\mathcal{F}\left\{ h{}^{n}\right\} (\alpha,\beta)\approx P(\alpha,\beta)$.
is a good approximation. We have to prove two things:
\begin{enumerate}
\item Show that there exist such an $N\mathbb{\in N}$s.t the approximation
holds true.
\item Find a value for B s.t. this approximation is below a certain error
bound, for example machine precision $\epsilon$. 
\end{enumerate}

\subsubsection{Proof Sketch of 1.}

By the \textbf{ratio test} (see \textbf{{[}1{]}}) we can show that
the series $\sum_{n=0}^{N}\frac{(ikwh)^{n}}{n!}\mathcal{F}\left\{ h{}^{n}\right\} (\alpha,\beta)$
converges absolutely:

\textbf{Proof}: Consider $\sum_{k=0}^{\infty}\frac{y^{n}}{n!}$where
$a_{k}=\frac{y^{k}}{k!}$. By the definition of the ratio test for
series it follows: $\forall y:$$limsup_{k\rightarrow\infty}|\frac{a_{k+1}}{a_{k}}|=limsup_{k\rightarrow\infty}\frac{y}{k+1}=0$ 

Thus this series converges absolutely, no matter what value we will
pick for y.


\subsubsection{Part 2: Find such an N}

Let $f(x)=e^{x}$. We can formulate its Taylor-Series, stated above.
Let $P_{n}(x)$denote the n-th Taylor-Polinomial, $P_{n}(x)=\sum_{k=0}^{n}\frac{f^{(k)}(a)}{k!}(x-a)^{k},$where
a is our developing point (here, in this case a=0). We can define
the error of the n-th Taylor-Polinomial to be $E_{n}(x)=f(x)-P_{n}(x)$.
That error is the actual value minus the Taylor polinomial. It holds
true: $|E_{n}(x)|=|f(x)-P_{n}(x)|$. By using the Lagrangien Error
Bound - (see source \textbf{{[}2{]}}) it follows: $|E_{n}(x)|\leq\frac{M}{(n+1)!}|x-a|^{n+1}$
with a=0, where \textbf{M }is some value satisfying $|f^{(n+1)}(x)|\leq M$
on the interval $I=[a,x]$.Since we are interested in an upper bound
of the error and since \textbf{a} is known, we can reformulate the
interval as $I=[0,x_{max}]$, where $x_{max}=|i|*k_{max}*w_{max}*h_{max}$,
since we are interested in computing an error bound for $e^{ikwh(x,y)}$.
From Stam's Paper about diffraction shader we know some paramters
for the length, width and height for a given sample patch, i.e. heightfield
h(x,y) and when using those parameters are able to find a explicit
number for $x_{max}$.

Facts we are using from Stam's Paper:

\begin{itemize}
\item Height of bump: 0.15micro meters
\item Width of a bump: 0.5micro meters
\item Length of a bump: 1micro meters
\item $k=\frac{2\pi}{\lambda}$ is the wavenumber and $\lambda\in[\lambda_{min,}\lambda_{max}]$its
wavelength hence $k_{max}=\frac{2\pi}{\lambda_{min}}$ 
\item $w$~is a component of the vector $\vec{v}=\vec{k_{1}}-\vec{k_{2}}=(u,v,w)$,
where $\vec{k_{1}}$ and $\vec{k_{2}}$ are \textbf{normalized} direction
vectors and this each component can have a value in range {[}-2, 2{]}.
\item for simplification, assume$[\lambda_{min,}\lambda_{max}]=[400nm,700nm].$
\end{itemize}

Hence $x_{max}=|i|*k_{max}*w_{max}*h_{max}=$$k_{max}*w_{max}*h_{max}=2*(\frac{2\pi}{4*10^{-7}m})*1.5*10^{-7}=1.5\pi$and
it follows for our intervall $I=[0,1.5\pi]$. Next we are going to
find the value for M. Since the exponential function is monoton growing
(on the interval I) and and the derivative of the \textbf{exp} function
is the exp function itself, we can find such an M: $M=e^{x_{max}}=exp(1.5\pi)$and
$|f^{(n+1)}(x)|\leq M$ holds. With $|E_{n}(x_{max})|\leq\frac{M}{(n+1)!}|x_{max}-a|^{n+1}=\frac{exp(1.5\pi)*(1.5\pi)^{n+1}}{(n+1)!}$we
now can find a value of n for a given bound, i.e. we can find an value
of $N\mathbb{\in N}$ s.t. $\frac{exp(1.5\pi)*(1.5\pi)^{N+1}}{(N+1)!}\leq\epsilon$.
With Octave/Matlab we can see: 
\begin{itemize}
\item if N=20 then $\epsilon\approx2.9950*10^{-4}$
\item if N=25 then $\epsilon\approx8.8150*10^{-8}$
\item if N=30 then $\epsilon\approx1.0050*10^{-11}$
\end{itemize}

\subsubsection{Conclusion}

With this approach we have that$\sum_{n=0}^{25}\frac{(ikwh)^{n}}{n!}\mathcal{F}\left\{ h{}^{n}\right\} (\alpha,\beta)$is
an approximation of P(u,v) with error$\epsilon\approx8.8150*10^{-8}$.
This means we can precompute 25 Fourier Transformations (for example
via FFT2) and then sum them up in order to approximate P(u,v) and
$\epsilon\approx8.8150*10^{-8}$. This approach will allow us to speed
up our shader. Furthermore we see that when we just take 5 more iterations,
we will reduce the error bound to the dimension of $10^{-11}$.

\textbf{NB}: More explanation about how the shader itself works during
our discussion.

\subsubsection{Our derivations}

EXPLAIN: Why do we want a formulation for $L_{\lambda}(w_r)$ in some words. what does it represent?

Definition of $BRDF(w_i, w_r) := f_r(w_i, w_r) = \frac{dL_r(w_r)}{dE_i(w_i)}=\frac{dL_r(w_r)}{L_i(w_i)cos(\theta_i)dw_i}$  
Hence, we can dervie the following expression:
\begin{align*}
f_r(w_i, w_r) = \frac{dL_r(w_r)}{L_i(w_i)cos(\theta_i)dw_i} \\
=> f_r(w_i, w_r) L_i(w_i)cos(\theta_i)dw_i = dL_r(w_r) \\
=> \int_{\Omega}f_r(w_i, w_r) L_i(w_i)cos(\theta_i)dw_i = \int_{\Omega}dL_r(w_r) \\
=> L_r(w_r) = \int_{\Omega}f_r(w_i, w_r) L_i(w_i)cos(\theta_i)dw_i
\end{align*}

We assume, that our incident light is a directional light source like sun-light and therefore its radiance is given as $L_{\lambda}(w)=I(\lambda)\delta(w-w_i)$ where $I(\lambda)$ is the intensity of the relative spectral power for the wavelength $\lambda$. Thus we get for our the brdf formulation:

\begin{align}
L_{\lambda}(w_r) 
& = \int_{\Omega} BRDF_{\lambda}(w_i, w_r) L_{\lambda}(w_i) cos(\theta_i) dw_i \\
& = BRDF_{\lambda}(w_i, w_r) I(\lambda) cos(\theta_i)
\end{align}

where $w_i$ is the solid angle for the incoming light, $\theta{_i}$ is the angle of incidence,
$w_r$ is the solid angle for the reflected light, $\lambda$ wavelength, $\Omega$ is the hemisphre we of integration for the incomming light.
Radiance reflected by given surface in given direction:
$L_{\lambda}(w_i)$ is the incomming radiance, $L_{\lambda}(w_r)$ is the reflected radiance

For the $BRDF(w_i, w_r)$ we are going to use the formulation dervied by Stam described above which looks like this using the fact that wavenumber $k=\frac{2\pi}{\lambda}$:

\begin{align*}
BRDF(w_i, w_r) 
& = \frac{k^2 F^2 G}{4\pi^2 A w^2} \langle \left|P(ku, kv) \right|^2\rangle \\
& = \frac{k^2 F^2 (1-\hat{\mathbf{k_1}}\cdot\hat{\mathbf{k_2}})}{cos(\theta_1)cos(\theta_2) 4\pi^2 A w^2} \langle \left|P(ku, kv)  \right|^2\rangle \\
& = \frac{4 \pi^2 F^2 (1-\hat{\mathbf{k_1}}\cdot\hat{\mathbf{k_2}})}{cos(\theta_1)cos(\theta_2) 4\pi^2 A \lambda^2 w^2} \langle \left|P(ku, kv)  \right|^2\rangle \\
& = \frac{F(w_i, w_r)^2 (1-\hat{\mathbf{k_1}}\cdot\hat{\mathbf{k_2}})}{cos(\theta_1)cos(\theta_2) A \lambda^2 w^2} \langle \left|P(ku, kv)  \right|^2\rangle
\end{align*}

where $\hat{\mathbf{k_t}}$ represents a unit vector whose spherical coordinates are given by the solid angle $t$.
Since we are going to integrate over a sphere $\Omega$ we can write the component $w=(cos(\theta_i)+cos(\theta_r))$
SHOW WHY WE ARE ALLOWED TO WRITE IT LIKE THIS => SPHERICAL COORDINATES DIFFERENCE $(k1 - k2) = (u,v,w)$ and so on.
this our identity for $L_{r}(w_r)$ will lead us to the following identiy using our identity :

\begin{align*}
L_{\lambda}(w_r) 
& = \frac{F(w_i, w_r)^2 (1-\hat{\mathbf{k_1}}\cdot\hat{ \mathbf{k_2}})^2}{A \lambda^2 cos(\theta_i)cos(\theta_r)  (cos(\theta_i)+cos(\theta_r))^2} \langle \left|P_{cont}(\frac{2\pi u}{\lambda}, \frac{2\pi v}{\lambda})  \right|^2\rangle cos(\theta_i) I(\lambda) \\
& = I(\lambda) \frac{F(w_i, w_r)^2 (1-\hat{\mathbf{k_1}}\cdot\hat{\mathbf{k_2}})^2}{\lambda^2 A (cos(\theta_i)+cos(\theta_r))^2 cos(\theta_r)} \langle \left|P_{cont}(\frac{2\pi u}{\lambda}, \frac{2\pi v}{\lambda})  \right|^2\rangle \\
& = I(\lambda) \frac{F(w_i, w_r)^2 (1-\hat{\mathbf{k_1}}\cdot\hat{\mathbf{k_2}})^2}{\lambda^2 A (cos(\theta_i)+cos(\theta_r))^2 cos(\theta_r)} \langle \left|T_0^2 P_{dtft}(\frac{2\pi u}{\lambda}, \frac{2\pi v}{\lambda})  \right|^2\rangle
\end{align*}

$P_{cont}$ is the continious inverse Fourier transform for the taylor seies of our hight-field representing the nano structure, i.e. $P(k,l) = \mathcal{F}^{-1}\{p\}(k,l)$ and $P_{dtft}$ is the dicrete-time inverse Fourier Transform for the same problem domain and $T_0$ the sampling distance for the discretization pf $p(x,y)$ assuming equal and uniform sampling in both dimensions $x,y$.

\subsection{Relative BRDF}
reason why relative brdf: In order to scale the reflactiance such that we are able to texture. 
convex combination reflectance with texture. Scale illumination.

Let us examine what $L_\lambda(w_r)$ will be for $w_r = w_0 := (0,0,*)$ i.e. specular reflection case, denoted as $L_\lambda^{spec}(w_0)$. 
When we know the expression for $L_\lambda^{spec}(w_0)$ we would be able to compute the relative reflected radiance for our problem by simply dividing $L_\lambda(w_r)$ by $L_\lambda^{spec}(w_0)$, denoted as 

\begin{equation}
    \rho_\lambda(w_i,w_r) = \frac{L_\lambda(w_r)}{L_\lambda^{spec}(w_0)}
\end{equation}

But first, let us derive the following expression:

\begin{align*}
L_\lambda^{spec}(w_0) 
& = I(\lambda) \frac{F(w_0, w_0)^2 (1-\colvec[0]{0}{1}\cdot\colvec[0]{0}{-1})^2}{\lambda^2 A (cos(0)+cos(0))^2 cos(0)} \langle \left|T_0^2 P_{dtft}(0,0)  \right|^2\rangle \\
& = I(\lambda) \frac{F(w_0, w_0)^2 (1+1)^2}{\lambda^2 A (1+1)^2 1}\left| T_0^2 N_{sample} \right|^2 \\
& = I(\lambda) \frac{F(w_0, w_0)^2}{\lambda^2 A}\left| T_0^2 N_{sample} \right|^2 
\end{align*}

Where $N_{samples}$ is the number of samples of the dtft.

Thus, we can plug our last derived expression into the definition for the relative reflectance radiance in the direction $w_r$ and will get:

\begin{align*}
\rho_\lambda(w_i,w_r)
& = \frac{L_\lambda(w_r)}{L_\lambda^{spec}(w_0)} \\
& = \frac{I(\lambda) \frac{F(w_i, w_r)^2 (1-\hat{\mathbf{k_1}}\cdot\hat{\mathbf{k_2}})^2}{\lambda^2 A (cos(\theta_i)+cos(\theta_r))^2 cos(\theta_r)} \langle \left|T_0^2 P_{dtft}(\frac{2\pi u}{\lambda}, \frac{2\pi v}{\lambda})  \right|^2\rangle}{I(\lambda) \frac{F(w_0, w_0)^2}{\lambda^2 A}\left| T_0^2 N_{sample} \right|^2 } \\
& = \frac{F^2(w_i,w_r)(1-\hat{\mathbf{k_1}}\cdot\hat{\mathbf{k_2}})^2}{F^2(w_0,w_0)(cos(\theta_i)+cos(\theta_r))^2 cos(\theta_r)}  \langle \left|\frac{P_{dtft}(\frac{2\pi u}{\lambda}, \frac{2\pi v}{\lambda})}{N_{samples}}\right|^2\rangle
\end{align*}

for simplification and a better overview, let us introduce the following expression, the so called gain factor

\begin{equation}
    C(w_i,w_r) = \frac{F^2(w_i,w_r)(1-\hat{\mathbf{k_1}}\cdot\hat{\mathbf{k_2}})^2}{F^2(w_0,w_0)(cos(\theta_i)+cos(\theta_r))^2 cos(\theta_r) N_{samples}^2}
\end{equation}

Using this substitute, we will end up with the following expression for the relative reflectance radiance

\begin{equation}
\rho_\lambda(w_i,w_r) =  C(w_i,w_r) \langle \left|P_{dtft}(\frac{2\pi u}{\lambda}, \frac{2\pi v}{\lambda})\right|^2\rangle
\end{equation}

using the previous definition for the relative reflectance radiance $\rho_\lambda(w_i,w_r) = \frac{L_\lambda(w_r)}{L_\lambda^{spec}(w_0)}$ which we can rearrange to the expression 

\begin{equation}
L_\lambda(w_r) = \rho_\lambda(w_i,w_r)L_\lambda^{spec}(w_0)
\end{equation}

Let us choose $L_\lambda^{spec}(w_0) = S(\lambda)$ such that is has the same pforifle as the relative spectral power distribuation of CIE Standard Illuminant $D65$. Further, when integration over $\lambda$ for a specular surface we should get $CIE_XYZ$ values corresponding to the white point for $D65$ 

the corresponding tristimulus values using CIE colormatching functions for the $CIE_XYZ$ values look like:

SEE HOW THIS DEFINITION DIFFERS FROM THE WIKIDEF AND HOW WE COULD END UP WITH A SIMILAR DEFINITION.

\begin{equation}
X = \int_{\lambda}L_\lambda(w_r)\overline{x}(\lambda)d\lambda
\end{equation} 

\begin{equation}
Y = \int_{\lambda}L_\lambda(w_r)\overline{y}(\lambda)d\lambda
\end{equation}

\begin{equation}
Z = \int_{\lambda}L_\lambda(w_r)\overline{z}(\lambda)d\lambda
\end{equation}

where $\overline{x}$, $\overline{y}$, $\overline{z}$ are the color matching functions

Using our last finding for $L_\lambda(w_r)$ and the definition for the tristimulus values we can actually derive an expression for computing the colors for our brdf model. Since X, Y, Z are defined similarly, it satisfies to derive an explicit expression for just one tristimulus term, for example X. The other two will look the same, except the we have to replace all X with Y or Z respectively. Therefore, we get:

\begin{align*}
X 
& =\int_{\lambda}L_\lambda(w_r)\overline{x}(\lambda)d\lambda \\
& =\int_{\lambda}\rho_\lambda(w_i,w_r)L_\lambda^{spec}(w_0) \overline{x}(\lambda)d\lambda \\
& =\int_{\lambda}\rho_\lambda(w_i,w_r) S(\lambda) \overline{x}(\lambda)d\lambda \\
& =\int_{\lambda} C(w_i,w_r) \langle \left|P_{dtft}(\frac{2\pi u}{\lambda}, \frac{2\pi v}{\lambda})\right|^2\rangle S(\lambda) \overline{x}(\lambda)d\lambda \\
& = C(w_i,w_r) \int_{\lambda} \langle \left|P_{dtft}(\frac{2\pi u}{\lambda}, \frac{2\pi v}{\lambda})\right|^2\rangle S(\lambda) \overline{x}(\lambda)d\lambda \\
& = C(w_i,w_r) \int_{\lambda} \langle \left|P_{dtft}(\frac{2\pi u}{\lambda}, \frac{2\pi v}{\lambda})\right|^2\rangle S_x(\lambda)d\lambda
\end{align*}

Where we used the definition $S_x(\lambda)\overline{x}(\lambda)$ in the last step.

\subsubsection{Taylour approximation for BRDF}
Using $P_{dtft} = \mathcal{F}^{-1}\{p\}(u,v)$ definied in the section of the taylor approximationwe get for the tristumulus value X, we will get:

\begin{align*}
X 
& = C(w_i,w_r) \int_{\lambda} \langle \left|P_{dtft}(\frac{2\pi u}{\lambda}, \frac{2\pi v}{\lambda})\right|^2\rangle S_x(\lambda)d\lambda \\
& = C(w_i,w_r) \int_{\lambda} \left| \sum_{n=0}^N \frac{(wk)^n}{n!} \mathcal{F}^{-1}\{i^n h^n\}(\frac{2\pi u}{\lambda}, \frac{2\pi v}{\lambda})\right|^2 S_x(\lambda)d\lambda
\end{align*}


\subsubsection{Sampling: Gaussian Window}


Let $window_g$ denote the gaussian window with $4\sigma_s$ $\mu m$ where $\sigma_f = \frac{1}{2\pi\sigma_s}$
let us further substitute $\mathbf{t(x,y)}=i^n h(x,y)^n$

\begin{equation}
\mathcal{F}_{dtft}^{-1}\{\mathbf{t}\}(u,v) = \mathcal{F}_{fft}^{-1}\{\mathbf{t}\}(u,v)window_g(\sigma_f)
\end{equation} 

Therefore we can deduce the following expression from this:

\begin{align*}
\mathcal{F}_{dtft}^{-1}\{\mathbf{t}\}(u,v)
& = \int_{-\infty}^{\infty} \int_{-\infty}^{\infty} {F}_{fft}^{-1}\{\mathbf{t}\}(w_u,w_v) \phi(u-w_u, v-w_v) dw_u dw_v \\
& = \int_{-\infty}^{\infty} \int_{-\infty}^{\infty} \sum_i \sum_j {F}_{fft}^{-1}\{\mathbf{t}\}(w_u,w_v) \delta(w_u-w_i, w_v-w_j)\phi(u-w_u, v-w_v) dw_u dw_v \\
& = \sum_i \sum_j \int_{-\infty}^{\infty} \int_{-\infty}^{\infty}  {F}_{fft}^{-1}\{\mathbf{t}\}(w_u,w_v) \delta(w_u-w_i, w_v-w_j)\phi(u-w_u, v-w_v) dw_u dw_v \\
& = \sum_i \sum_j {F}_{fft}^{-1}\{\mathbf{t}\}(w_u,w_v) \phi(u-w_u, v-w_v)
\end{align*}

where $\phi(x,y) = \pi e^{-\frac{x^2 + y^2}{2\sigma_{f}^2}}$

\subsubsection{Aplitude smooting}
Let us consider the so called 1-dimensional Box-function with length $T$ which is defined as the following: 
ADD AN IMAGE OF BOXFUNCTION

$
Box(x) =
\left\{
	\begin{array}{ll}
		1  & \mbox{if } x \leq T \\
		0 & \mbox{if } else
	\end{array}
\right.
$

We assume, that our given heighfield can be represented as a 2-dimensional box-function. 
Note that we can use any explicit given constrainted 2-dimensional function and will get some identities like
we get from the box-function.
 
Further we are assuming that we can model the overall surface be assuming this heighfield being distributed in a periodic manor.
Therfore, the whole surface can be represented like this $f(x) = \sum_{n=0}^{N} Box(x+nT_1, y+mT_2)$ assuming the given heighfield has the dimensions $T_1$ by $T_2$. But let us first consider the 1-dimensional Box-function case before deriving an identity for the Fourier transform of our 2-dimensional Box-function, i.e. the fourier transform of our heighfield. 

Note: A function $f$ periodic with periode $T$ means: $\forall x \in \mathcal{R}: Box(x) = Box(x+T)$

A so called bump can be represented by our 1-dimensional Box-function. We assume periodicity which is equaivalent to:   
$f(x) = \sum_{n=0}^{N} Box(x+nT)$

We are insterested in the 1-dimensional inverse Fourier transform of the 1-dimensional Box-function:

\begin{align*}
\mathcal{F}^{-1}\{f\}(w)
& =\int f(x) e^{iwx}dx\\
& =\int_{-\infty}^{\infty} \sum_{n=0}^{N} Box(x+nT) e^{iwx}dx\\
& =\sum_{n=0}^{N} \int_{-\infty}^{\infty} Box(x+nT) e^{iwx}dx
\end{align*}

Next, apply the following substituation $x+nT = y$ which will lead us to:

\begin{gather*}
x=y-nT\\
dx=dy
\end{gather*} 

Plugging this substituation back to the equation from above we will get 

\begin{align*}
\mathcal{F}^{-1}\{f\}(w)
& =\int f(x) e^{iwx}dx\\
& =\sum_{n=0}^{N} \int_{-\infty}^{\infty} Box(y) e^{iw(y-nT)}dy \\
& =\sum_{n=0}^{N} e^{-iwnT} \int_{-\infty}^{\infty} Box(y) e^{iwy}dy \\
& =\sum_{n=0}^{N} e^{-iwnT} \mathcal{F}\{f\}(w) \\
& =\mathcal{F}^{-1}\{f\}(w) \sum_{n=0}^{N} e^{-iwnT}  
\end{align*}

We used the fact that the term $e^{-iwnT}$ is a constant when integrating along $dy$ and the identity for the inverse Fourier transform of the Box function. Next, let us consider $\sum_{n=0}^N e^{-uwnT}$ further:

\begin{align*}
\sum_{n=0}^N e^{-uwnT}
& =\sum_{n=0}^N (e^{-uwT})^n \\
& =\frac{1-e^{iwT(N+1)}}{1-e^{-iwT}}
\end{align*}

We recognize the geometric series identity for the left-handside of this equation. Since our series is bounded we can derive our right-handside.

Since $e^{-ix}$ is a complex number and every complex number can be written in its polar form, i.e. $e^{-ix} = cos(x) + i sin(x)$ we can go even further, using the trigonometric idententities that $cos(-x) = cos(x)$ and $sin(-x) = -sin(x)$:

\begin{align*}
\frac{1-e^{iwT(N+1)}}{1-e^{-iwT}}
& =\frac{1-cos(wT(N+1)) + i sin(wT(N+1)) }{1-cos(wT) + i sin(wT)}
\end{align*}

Which is still a complex number $(p+iq)$. Every complex number can be written as a fraction of two complex numbers. This means that the complex number $(p+iq)$ can be written as $(p+iq) = \frac{(a+ib)}{(c+id)}$ for any $(a+ib), (c+id) \neq 0$. 
For our case, let us use the follwoing substituations: 

\begin{align}
a& := 1 - cos(wT(N+1))&
b& =sin(wT(N+1))\\
c& =1-cos(wT)&
d& =sin(wT)
\end{align}

hence it follows $\frac{1-e^{iwT(N+1)}}{1-e^{-iwT}} = \frac{(a+ib)}{(c+id)}$.
By rearanging the terms it follows $(a+ib) = (c+id)(p+iq)$ and multiplying the right handside out we get the follwing system of equations:

\begin{align}
(cp-dq)& =a\\
(dp + cq)& =b
\end{align}

Which gives lead us we some further math (trick: mult first eq. by $c$ and 2nd by $d$, then adding them together. using distributivity and we have the identity for p for example, similar for q) to 

\begin{align}
p& =\frac{(ac+bd)}{c^2 + d^2}\\
q& =\frac{(bc+ad)}{c^2 + d^2}
\end{align}


Putting our substituation for $a, b, c, d$ back into the current representatio for $p$ and $q$ and using some trigonometric identites, this we then get:

\begin{align}
p& =\frac{1}{2}+\frac{1}{2}\left(\frac{cos(wTN)-cos(wT(N+1))}{1-cos(wT)}\right)\\
q& =\frac{sin(wT(N+1))-sin(wTN)-sin(wT)}{2(1-cos(wT))}
\end{align}

Since we have seen, that $\sum_{n=0}^N e^{-uwnT}$ is a complex number and can be written as $(p+iq)$ and we know now the explicit identity for those $p$ and $q$ we get for the 1-dimensional Fourier transform of the 1-dimensional Box-function the following final identity:

\begin{align*}
\mathcal{F}^{-1}\{f\}(w)
& =\mathcal{F}^{-1}\{f\}(w) \sum_{n=0}^{N} e^{-iwnT} \\
& = (p+iq) \mathcal{F}^{-1}\{Box\}(w)  
\end{align*}

In oder to derive next a identity for the Fourier transform for our 2-dim heighfield, we can proceed similarly, the only fact which changes is, that we are now in a 2-dimensional domain, i.e. we are about to compute a two-dimensional Fourier transform:
Let us again us again a Box-function, this time a 2-dimensional Box-function $Box(x,y)$ just for the sake of convenience.

\begin{align*}
\mathcal{F}^{-1}\{f\}(w_1,w_2)
& = \int_{-\infty}^{\infty}\int_{-\infty}^{\infty} \sum_{n_2=0}^{N_1} \sum_{n_2=0}^{N_2} Box(x_1 + n_1 T_1, x_2 + n_2 T_2) e^{iw(x_1 + x_2)}dx_1 dx_2 \\
& = \int_{-\infty}^{\infty}\int_{-\infty}^{\infty} \sum_{n_2=0}^{N_1} \sum_{n_2=0}^{N_2} Box(y_1, y_2) e^{iw((y_1 - n_1 T_1) + (y_2 + n_2 T_2))}dx_1 dx_2 \\
& =\sum_{n_2=0}^{N_1} \sum_{n_2=0}^{N_2} \int_{-\infty}^{\infty}\int_{-\infty}^{\infty} Box(y_1, y_2) e^{iw(y_1 + y_2)} e^{-iw(n_1 T_1 + n_2 T_2)}dy_1 dy_2 \\
& =\sum_{n_2=0}^{N_1} \sum_{n_2=0}^{N_2} e^{-iw(n_1 T_1 + n_2 T_2)} \int_{-\infty}^{\infty}\int_{-\infty}^{\infty} Box(y_1, y_2) e^{iw(y_1 + y_2)} dy_1 dy_2 \\
& =\left(\sum_{n_2=0}^{N_1} \sum_{n_2=0}^{N_2} e^{-iw(n_1 T_1 + n_2 T_2)}\right) \mathcal{F}^{-1}\{Box\}(w_1,w_2) \\
& =\left(\sum_{n_2=0}^{N_1} e^{-iw n_1 T_1}\right) \left(\sum_{n_2=0}^{N_2} e^{-iw n_2 T_2}\right) \mathcal{F}^{-1}\{Box\}(w_1,w_2) \\
& =(p_1 + i q_1)(p_2 + i q_2) \mathcal{F}^{-1}\{Box\}(w_1,w_2) \\
& =((p_1 p_2 - q_1 q_2) + i(p_1 p_2 + q_1 q_2)) \mathcal{F}^{-1}\{Box\}(w_1,w_2) \\
& =(p + iq) \mathcal{F}^{-1}\{Box\}(w_1,w_2)
\end{align*}

Where we define $p := (p_1 p_2 - q_1 q_2) $ and $q := (p_1 p_2 + q_1 q_2)$. For this identity we used green's integration rule which allowed us to split the double integral to the product of two single integrations. Also, we used the definition of the 2-dimensional inverse Fourer transform of the Box-function. We applied the same substituation like we did in for the 1 dimensional case, but this time twice, once for each variable seperately. The last step, substituting with $p$ and $q$ will be useful later in the implementation. The insight should be, that the product of two complex numbers is again a complex number. We will have to compute the absolute value of $\mathcal{F}^{-1}\{f\}(w_1,w_2)$ which will then be equal $(p^2 + q^2)^{\frac{1}{2}}\left|\mathcal{F}^{-1}\{Box\}(w_1,w_2)\right|$

\subsubsection{Final Expression}
As the last step of our series of derivations, we plug all our findings together to one big equation in order to compute the colors in the $CIE_XYZ$ colorspace:

For a given heigh-field $h(x,y)$, representing a small patch of the nano-structure of our surface, the resulting $CIE_XYZ$ caused by the effect of diffraction can be computed like the following: 

Let $P(u,v,\lambda) = {F}_{fft}^{-1}\{i^n h^n\}(\frac{2\pi u}{\lambda},\frac{2\pi v}{\lambda})$

\begin{equation}
\begin{split}
\colvec[X]{X}{Z}& = C(w_i,w_r) \int_{\lambda} \sum_{n=0}^N  \frac{(wk)^n}{n!} \sum_r \sum_s \left| P(u-w_r,v-w_s,\lambda) \right|^2 \\
& \quad \quad  \phi(u-w_r, v-w_s) \colvec[S_x(\lambda)]{S_y(\lambda)}{S_z(\lambda)}d\lambda
\end{split}
\end{equation}

where $\phi(x,y) = \pi e^{-\frac{x^2 + y^2}{2\sigma_{f}^2}}$ is the gaussian window.
where $w_s$ and $w_r$ are ... explain them

